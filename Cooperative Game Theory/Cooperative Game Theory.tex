\documentclass{article}
\input{"preamble.tex"}
\numberwithin{equation}{section}
%\pagenumbering{gobble}

\RenewTcbTheorem[use counter from=dfn]{prob}{연습문제}{
	enhanced, breakable, frame hidden, left=0mm, right=0mm, top=4mm, bottom=2mm,
	colback=white, coltitle=white, colframe=RedOrange,
	borderline south={.5mm}{0pt}{RedOrange},
	before title={\,}, title={#1}, after title={\,}, fonttitle=\bfseries,
	attach boxed title to top left={yshift=-3.45mm},
	boxed title style={colback=RedOrange, sharp corners, left=0mm,right=0mm, top=0mm, bottom=0mm,},
	separator sign none, description delimiters parenthesis
}{prob}

\geometry{a4paper, total={6.4in, 10in}}
\title{협력 게임 이론}

\begin{document}
\setstretch{1.3}
\maketitle
\tableofcontents

\newpage
\section{협력 게임과 코어}

\subsection{협력 게임의 할당}
$n$을 자연수라 하고, 전체 참가자의 집합을 $N=\{1,2,\cdots,n\}$으로 두자. $N$의 부분집합, 즉 참가자 중 일부를 모은 집합을 \textbf{연합(coalition)}이라 하고, $N$을 \textbf{대연합(grand coalition)}이라 하자. $N$의 모든 연합의 집합은 $N$의 멱집합(power set)이므로 이를 $2^N$으로 표기하자.

\begin{dfn}{}{1.1}
	$v(\varnothing)=0$을 만족하는 함수 $v:2^N\to\mathbb R$을 \textbf{특성함수(characteristic function)}라 한다. 각 연합 $S\subseteq N$에 대하여 $v(S)$를 연합 $S$의 \textbf{가치(worth)}라 한다.
\end{dfn}

특성함수는 각 연합이 생산하는 성과를 측정한다. 연합과 특성함수가 주어진 경우의 게임을 생각하자.

\begin{dfn}{}{1.2}
	\textbf{효용 양도 가능 협력 게임(cooperative game with transferable utility)}은 참가자의 집합 $N$과 특성함수 $v$의 쌍 $(N,v)$이다.
\end{dfn}

\noindent 효용 양도 가능 협력 게임을 간단히 \textbf{협력 게임}이라 하자. 참가자의 집합이 맥랑상 분명할 때, 협력 게임 $(N,v)$를 간단히 $v$로 표기하자. 연합 $\{i,j,\cdots,k\}$는 중괄호를 생략하여 $i,j,\cdots,k$로 표기하자.

\begin{dfn}{}{1.3}
	연합 $S$의 \textbf{보수 분배(payoff distribution)}는 실벡터 $(x_i)_{i\in S}$이다.
\end{dfn}

\noindent 보수 분배 $x=(x_1,x_2,\cdots,x_n)\in\mathbb R^N$과 연합 $S\subseteq N$에 대하여 $x(S)=\sum_{i\in S}x_i$라 하자. $x(\varnothing)=0$으로 정하자.

협력 게임에서, 각 참가자의 보수는 보수 분배의 요소로 표현되므로 특성함수과 관계없이 임의로 정의할 수 있다. 그러나 (게임의 설계자인) 우리와 협력 게임의 참가자는 연합과 개인의 성과를 비교하여 보수를 구성하고자 하는 욕망을 갖고 있다. 협력 게임 $(N,v)$에서 다음 두 조건을 만족하는 보수 분배를 생각하자.
\begin{enumerate}[(a)]
	\item $x$는 \textbf{개인에게 합리적(individually rational)}이다: 모든 $i\in N$에 대하여 $x_i\geq v(i)$.
	\item $x$는 \textbf{효율적(efficient)}이다: $x(N)=v(N)$.
\end{enumerate}
위의 두 조건을 만족하는 보수 분배를 할당(imputation)이라 하자. 할당은 대연합의 성과 $v(N)$을 단독 행동의 성과보다 크도록 분배하는 보수 분배이다. 협력 게임 $v$에서 모든 할당의 집합을 $I(v)$라 하자.

\begin{example}{}{1.4}
	서로소(disjoint)인 두 연합 $S$와 $T$에 대하여 $v(S\cup T)=v(S)+v(T)$를 만족하는 협력 게임 $v$를 \textbf{가법적(additive)}이라 한다. 가법적 협력 게임에서는 모든 연합 $S$에 대하여
	\[
		v(S)=\sum_{i\in S}v(i)
	\]
	이므로 연합의 성과는 $1$인 연합의 성과의 합으로 특정된다. 어떤 할당 $x$에 대하여 $x_j>v(j)$인 $j\in N$가 존재한다면
	\[
		v(N)=x(N)=\sum_{i\in N}x_i=x_j+\sum_{i\in N\setminus\{j\}}x_i>v(j)+\sum_{i\in N\setminus\{j\}}v(i)=\sum_{i\in N}v(i)=v(N)
	\]
	이므로 모순이다. 따라서 모든 $i\in N$에 대하여 $x_i=v(i)$이므로 $I(v)=\{(v(1),v(2),\cdots,v(n))\}$이다.
\end{example}

협력 게임 $v$에서 $I(v)\ne\varnothing$일 필요충분조건은
\begin{equation}
	\label{eq:1.1}
	v(N)\geq\sum_{i\in N}v(i)
\end{equation}
이다. 부등식 \ref{eq:1.1}을 만족하는 협력 게임을 \textbf{본질적(essential)}이라 한다. 부등식 \ref{eq:1.1}의 부정이 성립한다고 가정하고 $x\in I(v)$라 하면
\[
	x(N)=v(N)<\sum_{i\in N}v(i)\leq\sum_{i\in N}x_i=x(N)
\]
이므로 $I(v)=\varnothing$이다. 역으로 부등식 \ref{eq:1.1}이 성립한다고 가정하자. 각 $i\in N$에 대하여 보수 분배 $f^i$를
\[
	f_i=\begin{cases}
		v(j) & (j\ne i\text{인 경우})\\
		v(N)-\sum_{k\in N\setminus\{i\}}v(k) & (j=i\text{인 경우})
	\end{cases}
\]
로 정의하자. $f^i$가 효율적임은 자명하다. $f^i$가 개인에게 합리적임은 $f_i^i\geq v(i)$임을 보이면 충분하다:
\[
	f_i^i=v(N)-\sum_{k\in N\setminus\{i\}}v(k)\geq\sum_{k\in N}v(k)-\sum_{k\in N\setminus\{i\}}v(k)=v(i)
\]

다음 정리에서 본질적 협력 게임의 할당 집합 $I(v)$가 $f^i$에 의해 특정됨을 알 수 있다.

\begin{thm}{}{1.5}
  본질적 협력 게임 $v$에서 $I(v)$는 $n$개의 점 $f^1,f^2,\cdots,f^n$의 볼록 껍질(convex hull)이다.
\end{thm}

\begin{proof}
	먼저 $I(v)$가 볼록집합(convex set)임을 보이자. $x,y\in I(v)$, $0\leq\lambda\leq1$이라 하자. $z=\lambda x+(1-\lambda)y$라 하면
	\begin{enumerate}[(a)]
		\item $z_i=\lambda x_i+(1-\lambda)y_i\geq\lambda v(i)+(1-\lambda)v(i)=v(i)$
		\item $z(N)=\lambda x(N)+(1-\lambda)y(N)=\lambda v(N)+(1-\lambda)v(N)=v(N)$
	\end{enumerate}
	이므로 $I(v)$는 볼록집합이다. $v$가 본질적 협력 게임이므로 $I(v)$는 집합 $\{f^i\mid i\in N\}$의 볼록 껍질을 포함한다. 이제 $x\in I(v)$라 하자. 각 $i\in N$에 대하여 $\alpha_i=x_i-v(i)\geq0$라 하면
	\[
		x=(v(1),v(2),\cdots,v(n))+(\alpha_1,\alpha_2,\cdots,\alpha_n)
	\]
	이고
	\[
		\alpha:=\sum_{j\in N}\alpha_j=x(N)-\sum_{j\in N}v(j)=v(N)-\sum{j\in N}v(j)\geq0
	\]
	이다. 만약 $\alpha=0$이면 임의의 $i\in N$에 대하여 $x=f^i$이다. 따라서 $\alpha\ne0$이라 가정하고 $\lambda_j=\alpha_j/\alpha$라 하자. 그러면 $0\leq\lambda_j\leq1$, $\sum_{j\in N}\lambda_j=1$이고
	\[
		\begin{aligned}
			\sum_{j\in N}\lambda_jf^j_i&=\sum_{j\in N\setminus\{i\}}\lambda_jv(i)+\lambda_i\biggl[v(N)-\sum_{j\in N\setminus\{i\}}v(j)\biggr]\\[3pt]
			&=v(i)\sum_{j\in N\setminus\{i\}}\lambda_j+\lambda_i[\alpha+v(i)]\\[3pt]
			&=v(i)\sum_{j\in N}\lambda_j+\lambda_i\alpha=v(i)+\alpha_i=x_i
		\end{aligned}
	\]
	이므로 $x=\sum_{j\in N}\lambda_jf^j$이다.
\end{proof}

\begin{example}{}{1.6}
  $(\{1,2,3\},v)$를 다음과 같은 $3$인 협력 게임이라 하자.
  \[
  	v(1)=v(3)=0,\quad v(2)=3,\quad v(1,2,3)=5
  \]
  그러면 $f^1=(2,3,0)$, $f^2=(0,5,0)$, $f^3=(0,3,2)$이고 $I(v)$는 $\mathbb R^3$에서 세 점 $f^1,f^2,f^3$을 잇는 삼각형이다.
\end{example}

협력 게임에서 한 연합이 구성되었을 때, 그 연합의 일원은 자신의 성과 내에서 최대한 많은 보상을 받고 싶어한다. 이에 연합 내에서 보수를 비교하는 기준을 다음과 같이 정의하자.

\begin{dfn}{}{1.7}
	$(N,v)$를 협력 게임이라 하자. $y,z\in I(v)$이고 $S\in 2^N\setminus\{\varnothing\}$이라 하자. 다음 두 조건을 만족하면 연합 $S$에서 \textbf{$y$가 $z$를 압도한다($y$ dominates $z$ via coalition $S$)}고 하고, 이를 $y\succ_Sz$로 표기한다.
	\begin{enumerate}[(i)]
		\item 모든 $i\in S$에 대하여 $y_i>z_i$이다.
		\item $y(S)\leq v(S)$
	\end{enumerate}
\end{dfn}

\noindent 연합 $S$에서 할당 $y$가 $z$를 압도함은 $S$의 각 참가자가 받는 보수가 더 크면서, $S$ 내에서의 협력만으로 보수를 충당할 수 있음을 의미한다. 연합 $S$에 대하여 집합 $D(S)$를
\[
	D(S)=\{z\in I(v)\mid y\succ_Sz\text{인 }y\in I(v)\text{가 존재한다.}\}
\]
로 정의하자. 이 집합에 속하는 할당은 항상 어떤 다른 할당에 비해 우월하지 않으므로 연합 $S$에 속한 참가자는 $D(S)$의 할당을 거부한다. 집합 $I(V)\setminus\bigcup_{S\in 2^N\setminus\{\varnothing\}}D(S)$에 속한 할당을 \textbf{압도당하지 않는다(undominated)}고 한다.

\begin{example}{}{1.8}
	협력 게임 $(N=\{1,2,3\},v)$를
	\[
		v(S)=\begin{cases}
			2 & (S=\{1,2\})\\
			1 & (S=N)\\
			0 & (S\ne\{1,2\},N)
		\end{cases}
	\]
	으로 정의하자. 그러면 $I(v)$는 세 점 $(1,0,0)$, $(0,1,0)$, $(0,0,1)$을 꼭짓점으로 하는 삼각형이고, $D(\{1,2\})=\{x\in I(v)\mid x_3>0\}$이다. $\{1,2\}$가 아닌 모든 연합 $S$에 대해서는 $D(S)=\varnothing$이다.
\end{example}

\begin{prob}{}{1.9}
  협력 게임 $(N,v)$에서 $\vert S\vert=1$이거나 $S=N$이면 $D(S)=\varnothing$임을 보여라.
\end{prob}

\begin{sol}
  일반성을 잃지 않고 $S=\{1\}$이라 하자. 할당 $y$, $z$에 대하여 $z\in D(S)$이고 $y\succ_Sz$라 가정하자. $y$는 $S$에서 $z$를 압도하는 할당이므로 $y_1\geq v(1)$이고 $y_1\leq v(1)$이다. 따라서
  \[
  		v(1)=y_1>z_1\geq z_1\geq v(1)
  \]
  이 성립하므로 모순이다. 이제 $S=N$이라 하자. 할당 $y$와 $z$가 앞선 상황과 같은 조건을 만족하면 모든 $i\in N$에 대하여 $y_i\geq v(i)$, $z_i\geq v(i)$, $y_i>z_i$이므로
  \[
  	v(N)=y(N)=\sum_{i\in N}y_i>\sum_{i\in N}z_i=z(N)=v(N)
  \]
  에서 모순이다. 그러므로 $\vert S\vert=1$이거나 $S=N$이면 $D(S)=\varnothing$이다.\qed
\end{sol}

\subsection{코어와 D--코어}
협력 게임 이론에서는 대연합을 형성하도록 하는 조건을 모색하는데 중점을 둔다. (이 조건을 협력 게임의 \textbf{해(solution)}라 한다.) 특정 참가자들이 대연합보다 작은 연합, 즉 파벌을 형성하지 않도록 하는 보수 분배의 집합을 \textbf{코어}라 한다.

\begin{dfn}{}{1.10}
	협력 게임 $(N,v)$의 \textbf{D--코어(unDominated core)}는 집합
	\[
		DC(v):=I(v)\setminus\bigcup_{S\in 2^N\setminus\{\varnothing\}}D(S),
	\]
	즉 압도당하지 않는 할당의 집합이다. 협력 게임 $(N,v)$의 \textbf{코어(core)}는 다음과 같이 정의한다.
	\[
		C(v):=\{x\in I(v)\mid\text{모든 }S\subseteq N\text{에 대하여 }x(S)\geq v(S).\}
	\]
\end{dfn}

\noindent $C(v)\ne\varnothing$이면 대연합 $N$이 아닌 연합(파벌) $S$가 형성될 이유는 없다. $x\in C(v)$에 대하여 연합 $S$에 배정된 보수의 합 $x(S)$가 연합 $S$를 형성하여 생산하는 가치 $v(S)$보다 부족하지 않기 때문이다.

\begin{example}{}{1.11}
	예시 \ref{ex:1.8}에서 정의한 협력 게임의 D--코어는 공집합이 아니지만 코어는 공집합이다.
\end{example}

\begin{example}{}{1.12}
	실수 $\alpha\in[0,1]$에 대하여 $3$인 협력 게임 $(N=\{1,2,3\},v)$를 다음과 같이 정의하자.
	\[
		v(S)=\begin{cases}
			1 & (\vert S\vert=3)\\
			\alpha & (\vert S\vert=2)\\
			0 & (\vert S\vert\leq1)
		\end{cases}
	\]
	협력 게임 $(N,v)$의 코어 $C(v)$는
	\[
		x_i\geq0,\quad x_1+x_2+x_3=1,\quad x_1+x_2\geq\alpha,\quad x_2+x_3\geq\alpha,\quad x_3+x_1\geq\alpha
	\]
	를 만족하는 벡터 $(x_1,x_2,x_3)$의 집합이다. 만약 $\alpha>2/3$이면
	\[
		x_1+x_2+x_3=\frac{1}{2}[(x_1+x_2)+(x_2+x_3)+(x_3+x_1)]\geq3\alpha>2
	\]
	이다. 따라서 $C(v)\ne\varnothing$일 필요충분조건은 $\alpha\leq2/3$이다.
\end{example}

\newpage
코어와 D--코어 사이에는 다음 관계가 성립한다.

\begin{thm}{}{1.13}
	협력 게임 $v$에서 $C(v)\subseteq DC(v)$이다.
\end{thm}

\begin{proof}
	협력 게임 $v$에 대하여 $x\in I(v)\setminus DC(v)$라 하자. 그러면 연합 $S\ne\varnothing$과 할당 $y$가 존재하여 $y\succ_Sx$이다. 따라서 $v(S)\geq y(S)>x(S)$가 성립하므로 $x\notin C(v)$이다.
\end{proof}

협력 게임의 코어는 선형 연립부등식으로부터 구할 수 있는 다포체이다. 특히 D--코어는 볼록 집합이다. (연습문제 \ref{prob:1.16}) 코어와 D--코어가 같아지도록 하는 충분조건은 다음과 같다.

\begin{thm}{}{1.14}
	협력 게임 $(N,v)$가
	\begin{equation}
		\label{eq:1.2}
		\text{모든 연합 }S\subseteq N\text{에 대하여 }v(N)\geq v(S)+\sum_{i\in N\setminus S}v(i)
	\end{equation}
	를 만족하면 $DC(v)=C(v)$이다.
\end{thm}

\begin{proof}
	정리 \ref{thm:1.13}으로부터 $DC(v)\subseteq C(v)$임을 보이면 충분하다. 먼저 다음을 보이자.
	\begin{lemma*}{}{}
  	할당 $x$가 어떤 연합 $S$에 대하여 $x(S)<v(S)$를 만족하면 $y\succ_Sx$인 할당 $y$가 존재한다.
	\end{lemma*}
	$y$를 다음과 같이 정의하자.
	\[
		y_i=\begin{cases}
			x_i+\dfrac{v(S)-x(S)}{\vert S\vert} & (i\in S\text{인 경우})\\[9pt]
			v(i)+\dfrac{v(N)-v(S)-\sum_{i\in N\setminus S}v(i)}{\vert N\setminus S\vert} & (i\notin S\text{인 경우})
		\end{cases}
	\]
	그러면 조건 \ref{eq:1.2}로부터 모든 $i\in N\setminus S$에 대하여 $y_i\geq v(i)$이므로 $y$는 $y\succ_Sx$를 만족하는 할당이다. 이제 $DC(v)\subseteq C(v)$임을 보이기 위해 $x\in DC(v)$라 하자. $x$를 압도하는 할당은 존재하지 않으므로 모든 $S\in 2^N\setminus\{\varnothing\}$에 대하여 $x(S)\geq v(S)$이 되어 $x\in C(v)$이다.
\end{proof}

\begin{remark}
  협력 게임 $v$의 코어가 공집합이 아니면 조건 \ref{eq:1.2}를 만족한다. $x\in C(v)$이면 모든 $S\in 2^N\setminus\{\varnothing\}$에 대하여 $x(S)\geq v(S)$이므로
  \[
  	v(N)=x(N)=x(S)+\sum_{i\in N\setminus S}x(i)\geq v(S)+\sum_{i\in N\setminus S}v(i)
  \]
  가 성립한다. 그러므로 이 경우에 D--코어와 코어는 서로 같다.
\end{remark}

현실적인 상황에서 유도되는 협력 게임 $v$는 다음 조건을 만족한다.
\begin{equation}
	\label{eq:1.3}
	\text{임의의 서로소인 두 연합 }S\text{와 }T\text{에 대하여 }v(S\cup T)\geq v(S)+v(T).
\end{equation}
조건 \ref{eq:1.3}을 만족하는 협력 게임을 \textbf{초가법적(superadditive)}라 한다. 조건 \ref{eq:1.3}은 조건 \ref{eq:1.2}를 함의하므로 정리 \ref{thm:1.14}는 초가법적 협력 게임에서도 성립한다.\qed

\begin{prob}{}{1.15}
  정리 \ref{thm:1.14}의 역은 성립하는가?
\end{prob}

\begin{sol}
	$DC(v)\ne\varnothing$이면 역도 성립한다. (\cite{3} Proposition 2.1) $C(v)=DC(v)$라 가정하자. $DC(v)\ne\varnothing$이므로 $x\in C(v)$라 하자. 그러면 임의의 연합 $S\subseteq N$에 대하여
	\[
		v(N)=x(N)=x(S)+\sum_{i\in N\setminus S}x(i)\geq v(S)+\sum_{i\in N\setminus S}v(i)
	\]
	이 성립한다.\qed
\end{sol}

\begin{prob}{}{1.16}
	협력 게임 $(N,v)$의 D--코어가 볼록 집합임을 보여라.
\end{prob}

\begin{sol}
	(\cite{3} 참고) 협력 게임 $(N,v)$에 대하여 특성함수 $v^*$를 $v^*(\varnothing)=0$이고, $S\in 2^N\setminus\{\varnothing\}$에 대하여
	\[
		v^*(S)=\min\biggl\{v(S),v(N)-\sum_{i\notin S}v(i)\biggr\}
	\]
	로 정의하자. 다음을 먼저 보이자.
	\begin{lemma*}{}{}
		$DC(v)\ne\varnothing$이면 $DC(v)=C(v^*)$이다.
	\end{lemma*}
	$v^*(N)=v(N)$이고, 정의로부터 $v^*(i)$는 $v(i)$와 $v(N)-\sum_{j\ne i}v(j)$의 임의의 볼록 결합보다 작거나 같으므로
	\[
		\begin{aligned}
			\sum_{i\in N}v^*(i)&\leq\frac{\vert N\vert-1}{\vert N\vert}\sum_{i\in N}v(i)+\frac{1}{\vert N\vert}\sum_{i\in N}\biggl[v(N)-\sum_{j\ne i}v(j)\biggr]\\[3pt]
			&=\frac{\vert N\vert-1}{\vert N\vert}\sum_{i\in N}v(i)+v(N)-\frac{\vert N\vert-1}{\vert N\vert}\sum_{i\in N}v(i)=v(N)=v^*(N)
		\end{aligned}
	\]
	이 성립한다. 협력 게임 $v^*$는 본질적이므로 $I(v^*)\ne\varnothing$이다. 또한 $v^*(i)=v(i)$일 필요충분조건은 $I(v)\ne\varnothing$이다:
	\[
		v^*(i)=v(i)~\Longleftrightarrow~v(i)\leq v(N)-\sum_{j\ne i}v(j)~\Longleftrightarrow~v(N)\geq\sum_{j\in N}v(j)~\Longleftrightarrow~I(v)\ne\varnothing.
	\]
	이로부터 $I(v)\ne\varnothing$일 필요충분조건은 $I(v^*)=I(v)$임을 알 수 있다. 이제 $DC(v)\ne\varnothing$이라 가정하면 $I(v)\ne\varnothing$이므로 $I(v)=I(v^*)$이다. $x\notin DC(v)$라 하자. 그러면 연합 $S$와 할당 $y$가 존재하여 $y\succ_Sx$이다. $y$가 할당이므로
	\[
		y(S)=y(N)-\sum_{i\notin S}y_i\leq v(N)-\sum_{i\notin S}v(i)
	\]
	이 성립하고, $y(S)\leq v(S)$이므로 $y(S)\leq v^*(S)$이다. 따라서 $y\in I(v^*)$이고 협력 게임 $(N, v^*)$에서 $y\succ_Sx$이다. 따라서 $x\notin DC(v^*)$이므로 $DC(v^*)\subseteq DC(v)$가 성립한다. $DC(v)\subseteq DC(v^*)$임은 자명하므로 정리 \ref{thm:1.14}에 의해
	\[
		DC(v)=DC(v^*)=C(v^*)
	\]
	이다. 협력 게임의 코어가 볼록 집합임은 쉽게 확인할 수 있으므로 $DC(v)$ 역시 볼록 집합이다.\qed
\end{sol}

\subsection{단순 협력 게임}
\begin{dfn}{}{1.17}
	\textbf{단순 협력 게임(simple game)}은 다음 두 조건을 만족하는 협력 게임 $(N,v)$이다.
	\begin{enumerate}[(i)]
		\item 임의의 연합 $S$에 대하여 $v(S)\in\{0,1\}$이다.
		\item $v(N)=1$
	\end{enumerate}
	$v(S)=1$을 만족하는 연합 $S$를 \textbf{승리 연합(winning coalition)}, 그렇지 않은 연합을 \textbf{패배 연합(losing coalition)}이라 한다.
\end{dfn}

\begin{dfn}{}{1.18}
	단순 협력 게임 $(N,v)$에 대하여
	\begin{itemize}
		\item 모든 진부분집합이 패배 연합인 승리 연합을 \textbf{최소 승리 연합(minimal winning coalition)}이라 한다.
		\item $v(S)=1\Longleftrightarrow i\in S$를 만족하는 참가자 $i$를 \textbf{독재자(dictator)}라 한다.
		\item 모든 승리 연합에 속해있는 참가자 $i$를 \textbf{거부권자(veto player)}라 하고, 모든 거부권자의 집합을 $\operatorname{veto}(v)$로 나타낸다. 즉 $\operatorname{veto}(v)=\bigcap\{S\in 2^N\mid v(S)=1\}$이다.
	\end{itemize}
\end{dfn}

각 $i\in N$에 대하여 $i$번째 성분만 $1$이고 나머지 성분은 모두 $0$인 $\mathbb R^N$의 벡터를 $e^i$라 하자.

\begin{example}{}{1.19}
	$i\in N$이라 하자. \textbf{독재 게임(dictator game)} $\delta_i$를
	\[
		\delta_i(S)=1~\Longleftrightarrow~i\in S
	\]
	인 단순 협력 게임이라 하자. 그러면 $\operatorname{veto}(\delta_i)=\{i\}$이다. $x\in I(v)$이면 모든 $j\ne i$에 대하여 $x_j\geq v(j)=0$이고
	\[
		1=v(i)\leq x_i\leq x(N)=v(N)=1
	\]
	이므로 $x=e^i$이다. 따라서 $I(\delta_i)=\{e^i\}$이고, $C(\delta_i)=DC(\delta_i)=\{e^i\}$이다.
\end{example}

\begin{example}{}{1.20}
	$\vert N\vert$이 홀수일 때, \textbf{다수결 게임(majority game)} $(N,v)$를
	\[
		v(S)=1~\Longleftrightarrow~\vert S\vert\geq\frac{\vert N\vert}{2}
	\]
	인 단순 협력 게임이라 하자. $x\in C(v)$라 하자. $\vert S\vert=\vert N\vert-1$이면 $v(S)=1$이므로 $x(S)=1$이다. 크기가 $\vert N\vert-1$인 연합은 $\vert N\vert$개 있으므로 $\sum_{S:\vert S\vert=\vert N\vert-1}x(S)=\vert N\vert$이다. 한편,
	\[
		\sum_{S:\vert S\vert=\vert N\vert-1}x(S)=\sum_{S:\vert S\vert=\vert N\vert-1}\sum_{i\in S}x_i=(\vert N\vert-1)x(N)=\vert N\vert-1
	\]
	이므로 이는 모순이다. 따라서 다수결 게임의 코어는 공집합이다.
\end{example}

\newpage
단순 협력 게임의 코어는 거부권자가 존재할 때에만 공집합이 아니다. 게다가 코어에 속하는 보수 분배는 대연합의 가치 $1$을 거부권자에게만 분배한다.

\begin{thm}{}{1.21}
	$(N,v)$를 단순 협력 게임이라 하자.
	\begin{enumerate}[(1)]
		\item $C(v)$는 집합 $\{e^i\mid i\in\operatorname{veto}(v)\}$의 볼록 껍질이다.
		\item $\operatorname{veto}(v)=\varnothing$이고 $\{i\in N\mid v(i)=1\}=\{k\}$이면 $C(v)=\varnothing$이고 $DC(v)=\{e^k\}$이다. \\ 그렇지 않으면, $DC(v)=C(v)$이다.
	\end{enumerate}
\end{thm}

\begin{proof}
	\phantom{}
	\begin{enumerate}[(1)]
		\item $i\in\operatorname{veto}(v)$이고 $S\in 2^N\setminus\{\varnothing\}$이라 하자. $i\in S$이면 $e^i(S)=1\geq v(S)$이고, $i\notin S$이면 $e^i(S)=0=v(S)$이다. $e^i(N)=1=v(N)$이므로 $e^i\in C(v)$이다. $C(v)$는 볼록집합이므로 집합 $\{e^i\mid i\in\operatorname{veto}(v)\}$의 볼록 껍질을 포함한다. 역으로 $x\in C(v)$라 하자. $i\notin\operatorname{veto}(v)$이면 $x_i=0$임을 보이면 충분하다. 어떤 $i\notin\operatorname{veto}(v)$에 대하여 $x_i>0$이라 가정하자. $i$는 거부권자가 아니므로 $i\notin S$이고 $v(S)=1$인 연합 $S$가 존재한다. 그러면
			\[
				x(S)=x(N)-x(N\setminus S)\leq 1-x_i<1=v(S)
			\]
			이므로 $x\in C(v)$임에 모순이다. 따라서 $C(v)$는 집합 $\{e^i\mid i\in\operatorname{veto}(v)\}$의 볼록 껍질이다.
		\item $\operatorname{veto}(v)=\varnothing$이면 (1)에 의해 $C(v)=\varnothing$이다. 여기에 $k$가 집합 $\{i\in N\mid v(i)=1\}$의 유일한 원소이면 $I(v)=\{e^k\}$이므로 $DC(v)=\{e^k\}$이다. $\operatorname{veto}(v)=\varnothing$이고 $\{i\in N\mid v(i)=1\}=\varnothing$이면 조건 \ref{eq:1.2}를 만족하므로 정리 \ref{thm:1.14}에 의해 코어와 D--코어는 같다. $\operatorname{veto}(v)=\varnothing$이고 $\vert\{i\in N\mid v(i)=1\}\vert\geq2$이면 $I(v)=\varnothing$이므로 $C(v)=DC(v)=\varnothing$이다. 이제 $\operatorname{veto}(v)\ne\varnothing$이면 (1)에 의해 $C(v)\ne\varnothing$이므로 $C(v)=DC(v)$이다.\qedhere
	\end{enumerate}
\end{proof}

\begin{example}{}{1.22}
	$T$를 공집합이 아닌 연합이라 하자. \textbf{$T$--만장일치 게임($T$--unanimity game)} $u_T$를
	\[
		u_T(S)=1~\Longleftrightarrow~T\subseteq S
	\]
	인 단순 협력 게임이라 하자. $\operatorname{veto}(u_T)=T$이므로 정리 \ref{thm:1.21}로부터 $C(u_T)$와 $DC(u_T)$ 모두 집합 $\{e^i\mid i\in T\}$의 볼록 껍질과 같다.
\end{example}

\begin{example}{}{1.23}
	$3$인 협력 게임 $(\{1,2,3\},v)$를
	\[
		v(S)=\begin{cases}
			1 & (S=\{1\},\{2,3\},\{1,2,3\})\\
			0 & (\text{그 외의 경우})
		\end{cases}
	\]
	로 정의하자. 그러면 $\operatorname{veto}(v)=\varnothing$, $C(v)=\varnothing$, $DC(v)=\{(1,0,0)\}$이다. $(N,v)$는 초가법적이 아니고 조건 \ref{eq:1.2}를 만족하지 않는다.
\end{example}

\newpage
\subsection{안정 집합}
존 폰 노이만과 모겐스턴은 협력 게임에서 다음과 같은 해를 고려했다.

\begin{dfn}{}{1.24}
	협력 게임 $v$에 대하여 집합 $A\subseteq I(v)$가 다음 두 조건을 만족하면 $A$를 \textbf{안정 집합(stable set)}이라 한다.
	\begin{enumerate}[(i)]
		\item \textbf{내적 안정성(internal stability)}: $x,y\in A$이면 $x$는 $y$를 압도하지 않는다.
		\item \textbf{외적 안정성(external stability)}: $x\in I(v)\setminus A$이면 $x$를 압도하는 $y\in A$가 존재한다.
	\end{enumerate}
\end{dfn}

\noindent 안정 집합은 할당의 비교 관점에서 `안정적'이다. 한 협력 게임에 여러 안정 집합이 존재할 수 있으며, 안정성은 집합이 갖는 성질이므로 안정 집합의 적절한 선택이 필요하다. 코어가 존재하는 협력 게임은 \ref{subsection:1.5}에서 완벽하게 특정되지만, 안정 집합의 존재성은 부분적으로만 해결되었다.

\begin{thm}{}{1.25}
	$v$를 단순 협력 게임이라 하고, $S$를 최소 승리 연합이라 하자. 집합 $\Delta^S$를
	\[
		\Delta_S:=\{x\in I(v)\mid\text{모든 }i\notin S\text{에 대하여 }x_i=0\}
	\]
	이라 하자. $\Delta^S\ne\varnothing$이면 $\Delta^S$는 안정 집합이다.
\end{thm}

\begin{proof}
	$\Delta^S\ne\varnothing$이면 모든 $i\notin S$에 대하여 $v(i)=0$임은 쉽게 확인할 수 있다.
	\begin{enumerate}[(i)]
		\item $x,y\in\Delta^S$와 연합 $T$에 대하여 $x\succ_Ty$라 하자. $i\notin S$에 대하여 $x_i=y_i=0$이므로 $T\subseteq S$여야 한다. 만약 $T\subsetneq S$이면 $S$의 최소성에 의해 $v(T)=0$이므로 $x(T)=0$이다. 따라서 모든 $i\in T$에 대하여 $x_i=0$이므로 $x\succ_Ty$임에 모순이다. 이제 $T=S$라 하자. $x(N\setminus S)=\sum_{i\notin S}x_i=0$이므로 $x(S)=x(S)+x(N\setminus S)=x(N)=1$이고, 같은 논리로 $y(S)=1$이다. 이는 $x\succ_Ty$임에 모순이다.
		\item $x\in I(v)\setminus\Delta^S$라 하자. 그러면 어떤 $i\notin S$에 대하여 $x_i>0$이므로 $\alpha:=x(N\setminus S)>0$이다. 보수 분배 $y$를
		\[
			y_i=\begin{cases}
				x_i+\dfrac{\alpha}{\vert S\vert} & (i\in S)\\
				0 & (i\notin S)
			\end{cases}
		\]
		로 정의하자. 그러면 $y\in\Delta^S$이고 $y(S)=\sum_{i\in S}x_i+\alpha=x(N)=1$이므로 $y\succ_Sx$이다.
	\end{enumerate}
	(i), (ii)로부터 $\Delta^S$가 안정 집합임을 알 수 있다.
\end{proof}

D--코어와 안정 집합 사이의 관계는 다음과 같다.

\begin{thm}{}{1.26}
	협력 게임 $(N,v)$에 대하여
	\begin{enumerate}[(1)]
		\item D--코어는 모든 안정 집합의 부분집합이다.
		\item $A$와 $B$가 서로 다른 안정 집합이면 $A\nsubseteq B$이다.
		\item D--코어가 안정 집합이면 이는 유일한 안정 집합이다.
	\end{enumerate}
\end{thm}

\begin{proof}
	(1)과 (2)는 외적 안정성에 의해 성립한다. (3)은 (1)과 (2)에 의해 성립한다.
\end{proof}

\newpage
안정 집합은 참가자의 행동 기준(standard of behavior)를 규정한다. 다음 예시를 보자.

\begin{example}{}{1.27}
	$(\{1,2,3\},v)$를 $3$인 다수결 게임이라 하자. (예시 \ref{ex:1.20}) 집합 $X$를
	\[
		X:=\biggl\{\biggl(\frac{1}{2},\frac{1}{2},0\biggr),\biggl(\frac{1}{2},0,\frac{1}{2}\biggr),\biggl(0,\frac{1}{2},\frac{1}{2}\biggr)\biggr\}
	\]
	라 하자. $X$의 내적 안정성은 쉽게 확인할 수 있다. $X$의 외적 안정성을 보이기 위해 $z\in I(v)\setminus X$라 하자. 그러면 어느 두 참가자 $i$와 $j$에 대하여 $z_i<1/2$이므로 연합 $\{i,j\}$에서 $z$를 압도하는 $x\in X$를 택할 수 있다. 따라서 $X$는 안정 집합이다. $X$은 두 참가자가 연합을 이루어 보상을 얻은 후 이를 서로 반씩 가지는 행동 양식으로 해석 가능하다. 이제 $c\in[0,1/2)$이라 하고 정해진 차가자 $i$에 대하여
	\[
		Y_{i,c}:=\{y\in I(v)\mid y_i=c\}
	\]
	라 하자. $Y_{i,c}$의 내적 안정성 역시 쉽게 확인할 수 있다. 일반성을 잃지 않고 $i=3$이라 하자. $Y_{i,c}$의 외적 안정성을 보이기 위해 $z\in I(v)\setminus Y_{i,c}$라 하자. $z_3>c$이면 $z_1+z_2<1-c$이므로 할당 $y=(y_1,y_2,y_3)$을
	\[
		y_1=z_1+\frac{(1-c)-(z_1+z_2)}{2},\quad y_2=z_2+\frac{(1-c)-(z_1+z_2)}{2},\quad y_3=c
	\]
	라 하면 $y\succ_{\{1,2\}}z$이다. $z_3<c$이고 $z_1\leq z_2$이면 $(1-c,0,c)\succ_{\{1,3\}}z$이다. 따라서 $Y_{i,c}$ 역시 안정 집합이다. $Y_{i,c}$는 참가자 $i$를 정해 무조건적으로 $c$만큼의 보상을 받게끔 하는 행동 양식으로 해석 가능하다.
\end{example}

\begin{prob}{글러브 게임 각색}{1.28}
	어느 보물 상자는 두 열쇠 $\alpha$와 $\beta$가 적어도 하나씩 있어야 열리는 구조이다. 세 사람 $A$, $B$, $C$는 각각 열쇠 $\alpha$, $\beta$, $\alpha$를 갖고 있다. 그러므로 보물 상자를 열기 위해서는 $B$가 $A$ 또는 $C$와 같은 편이 되어야 한다. 이를 다음과 같이 정의된 $3$인 협력 게임 $(\{1,2,3\},v)$로 기술하자.
	\[
		v(S)=\begin{cases}
			1 & (S=\{1,2\},\{2,3\},\{1,2,3\})\\
			0 & (\text{그 외의 경우})
		\end{cases}
	\]
	\begin{enumerate}[(a)]
		\item $e^2=(0,1,0)$이 아닌 할당은 어떤 다른 할당에 압도됨을 보여라.
		\item 이 협력 게임의 코어와 D--코어를 구하여라.
		\item D--코어가 안정 집합이 아님을 보여라.
		\item 집합 $B=\{(x,1-2x,x)\mid0\leq x\leq1/2\}$이 안정 집합임을 보여라.
	\end{enumerate}
\end{prob}

\begin{sol}
	\begin{enumerate}[(a)]
		\item 할당 $x=(x_1,x_2,x_3)$가 $e^2$이 아니면 $x_2<1$이다.
			\begin{enumerate}[(i)]
				\item $x_3>0$인 경우: $(x_1+x_3/2,x_2+x_3/2,0)\succ_{\{1,2\}}x$이다.
				\item $x_3=0$인 경우: $(0,x_2+x_1/2,x_1/2)\succ_{\{2,3\}}x$이다.
			\end{enumerate}
		\item $\operatorname{veto}(v)=\{2\}$이므로 정리 \ref{thm:1.14}와 정리 \ref{thm:1.21}로부터 $C(v)=DC(v)=\{e^2\}$이다.
		\item $v$의 정의에 의해 어떤 할당이 연합 $S$에서 다른 할당을 압도한다면 $S$는 $2$를 반드시 포함해야 한다. 그러므로 D--코어는 외적 안정성을 갖지 않는다.
		\item 편의를 위해 실수 $x\in[0,1/2]$에 대하여 $b(x)=(x,1-2x,x)\in B$라 하자.
			\begin{enumerate}[(i)]
				\item 내적 안정성: (c)와 같은 논리로 어떤 연합 $S$에 대하여 $b(x)\succ_Sb(y)$이면
					\[
							b(x)_2>b(y)_2~\Longrightarrow~1-2x>1-2y~\Longrightarrow~x<y~\Longrightarrow~ S=\{2\}
					\]
					이다. $v(2)=0$이므로 이는 불가능하다.
				\item 외적 안정성: 할당 $x=(x_1,x_2,x_3)$가 $B$에 속하지 않을 필요충분조건은 $x_1\ne x_3$이다. 그러한 할당 $x$에 대하여 $y=(x_1+x_3)/2$라 하면 $b(y)\in B$이고 $b(y)\succ_{\{1,2\}}x$ ($x_1<x_3$인 경우) 또는 $b(y)\succ_{\{2,3\}}x$ ($x_1>x_3$인 경우)이다.\qed
			\end{enumerate}
	\end{enumerate}
\end{sol}

\subsection{코어의 존재성}
\label{subsection:1.5}
여기서는 협력 게임의 코어가 공집합이 아닐 필요충분조건을 논한다. 코어는 단순히 선형 연립부등식의 해로 정의되지만, 협력 게임의 사고 방식으로부터 구체적인 필요충분조건을 유도하고자 한다. 협력 게임 $(N,v)$의 연합 $S$에 대하여 \textbf{특성벡터(characteristic vector)} $e^S\in\mathbb R^N$을 다음과 같이 정의하자.
\[
	e^S_i=\begin{cases}
		1 & (i\in S)\\
		0 & (i\notin S)
	\end{cases}
\]
함수 $\lambda:2^N\setminus\{\varnothing\}\to[0,\infty)$가
\[
	\sum_{S\subseteq N}\lambda(S)e^S=e^N
\]
을 만족하면 $\lambda$를 \textbf{균형 사상(balanced map)}이라 한다. 공집합이 아닌 연합의 모임 $B$가 어떤 균형 사상 $\lambda$에 대하여
\[
	B=\{S\in 2^N\setminus\{\varnothing\}\mid\lambda(S)>0\}
\]
을 만족하면 $B$를 \textbf{균형 연합 모임(balanced collection)}이라 한다.

균형 사상은 참가자의 시간 분배의 관점에서 해석할 수 있다. 각 참가자마다 정해진 단위 시간이 주어진다고 생각하자. 참가자 $i$는 자신이 속할 수 있는 연합(파벌)마다 시간을 투자할 수 있다. 이러한 `시간 배분'이 어떤 균형 사상 $\lambda$로 기술되면 적절한 배분으로 보는 것이다. $\lambda(S)$는 연합 $S$가 형성되어 존재하는 시간으로 생각할 수 있고, 균형 사상은 각 참가자가 자신이 가진 한 단위의 시간을 제각기 다른 연합에 정확하게 배분함을 의미한다.

\begin{example}{}{1.29}
	\begin{enumerate}[(1)]
		\item $B=\{N_1,N_2,\cdots,N_k\}$를 공집합이 아닌 연합으로 이루어진 $N$의 분할이라 하자. 사상 $\lambda$를
			\[
				\lambda(S)=\begin{cases}
					1 & (S\in B)\\
					0 & (\text{그 외의 경우})
				\end{cases}
			\]
			라 하면 $\lambda$는 균형 사상이고, $B$는 $\lambda$에 대해 균형 연합 모임이 된다.
		\item $N=\{1,2,3\}$에 대하여 집합 $B=\{\{1,2\},\{1,3\},\{2,3\}\}$은
			\[
				\lambda(S)=\begin{cases}
					1/2 & (\vert S\vert=2)\\
					0 & (\text{그 외의 경우})
				\end{cases}
			\]
			로 정의된 균형 사상 $\lambda$에 대해 균형 연합 모임을 이룬다.
	\end{enumerate}
\end{example}

\begin{dfn}{}{1.}
	협력 게임 $(N,v)$가 모든 균형 사상 $\lambda:2^N\setminus\{\varnothing\}\to[0,\infty)$에 대하여
	\[
		\sum_{S\subseteq N}\lambda(S)v(S)\leq v(N)
	\]
	을 만족하면 $(N,v)$를 \textbf{균형 게임(balanced game)}이라 한다.
\end{dfn}

시간 배분의 관점을 고려하여 균형 게임을 해석하자. `균형잡힌' 시간 배분 $\lambda$에 대하여 각 참가자가 각 연합에 시간을 투자하여 얻는 가치의 총합이 부등식의 좌변으로 표현된다. 한편 우변은 참가자 전원이 자신의 시간을 대연합에 투자하여 얻는 가치이다. 그러므로 균형 게임은 대연합에 온전히 집중하지 않고 더 작은 연합(파벌)을 만드는 행위가 비효율적임을 의미한다. 여기에 코어의 의미를 덧붙여 생각하면 다음 정리를 자연스럽게 떠올릴 수 있다.

\begin{thm}{Bondareva--Shapley}{1.31}
	협력 게임 $(N,v)$에 대하여 다음은 서로 동치이다.
	\begin{enumerate}[(i)]
		\item $C(v)\ne\varnothing$
		\item $(N,v)$는 균형 게임이다.
	\end{enumerate}
\end{thm}

정리 \ref{thm:1.31}은 선형계획법의 쌍대성으로부터 유도된다.

\begin{lemma}{}{1.32}
	$A\in\mathbb R^{n\times p}$, $b\in\mathbb R^p$, $c\in\mathbb R^n$이라 하고, 두 집합 $X=\{x\in\mathbb R^n\mid xA\geq b\}$와 $Y=\{y\in\mathbb R^p\mid Ay=c,y\geq0\}$이 모두 공집합이 아니라고 하자. 그러면 다음이 성립한다:
	\[
		\min\{x\cdot c\mid xA\geq b\}=\max\{b\cdot y\mid Ay=c,y\geq0\}.
	\]
	만약 $X$와 $Y$ 중 하나가 공집합이면 위의 최댓값와 최솟값은 달성할 수 없다.
\end{lemma}

\begin{remark}[정리 \ref{thm:1.31}의 증명]
	$(N,v)$를 협력 게임이라 하자. $n=\vert N\vert$에 대하여 $p=2^n$이라 하고, $N$의 모든 부분집합을 
	\[
		S_1=\varnothing,S_2,\cdots,S_p=N
	\]
	이라 하자. $k=1,2,\cdots,p$에 대하여 $e^{S_k}$를 $k$번째 열로 갖는 $n\times p$ 행렬을 $A$라 하자. $b=(v(S_k))_{k=1}^p\in\mathbb R^p$라 하면
	\[
		xA\geq b~\Longleftrightarrow~x\cdot e^{S_k}\geq v(S_k)~\Longleftrightarrow~x(S_k)\geq v(S_k)
	\]
	이므로 $\{x\in\mathbb R^n\mid xA\geq b\}\ne\varnothing$일 필요충분조건은 $C(v)\ne\varnothing$이다. 한편
	\[
		Ay=c~\Longleftrightarrow~\sum_{k=1}^py_ke^{S_k}=e^N
	\]
	이므로 집합 $\{y\in\mathbb R^p\mid Ay=c,y\geq0\}\ne\emptyset$이다. (벡터 $(0,0,\cdots,0,1)\in\mathbb R^p$이 이 집합에 속한다.) $C(v)\ne\varnothing$일 필요충분조건이
	\[
		v(N)=\min\{x(N)\mid x\in\mathbb R^N\text{이고 모든 }S\subseteq N\text{에 대하여 }x(S)\geq v(S).\}=\min\{x\cdot c\mid xA\geq b\}
	\]
	이다. 보조정리 \ref{lemma:1.32}로부터 이는
	\[
		v(N)=\max\biggl\{\sum_{S\subseteq N}\lambda(S)v(S)\mid\sum_{S\subseteq N}\lambda(S)e^S=e^N,\lambda\geq0\biggr\}=\{b\cdot y\mid Ay=c,y\geq0\}
	\]
	와 동치이고, 이는 $(N,v)$가 균형 게임임을 의미한다.\qed
\end{remark}

\begin{prob}{}{1.33}
	$N=\{1,2,3,4\}$이라 하자. 다음과 같이 정의된 협력 게임 $(N,v)$의 코어가 공집합임을 보여라.
	\[
		v(S)=\begin{cases}
			1 & (S=N)\\
			\frac{3}{4} & (S=\{1,2\},\{1,3\},\{1,4\},\{2,3,4\})\\
			0 & (\text{그 외의 경우})
		\end{cases}
	\]
\end{prob}

\begin{sol}
	사상 $\lambda:2^N\setminus\{\varnothing\}\to[0,\infty)$를 다음과 같이 정의하자.
	\[
		\lambda(\{1,2\})=\lambda(\{1,3\})=\lambda(\{1,4\})=\frac{1}{3},\quad\lambda(\{2,3,4\})=\frac{2}{3},\quad\text{그 외의 경우에는 }\lambda(S)=0
	\]
	그러면
	\[
		\sum_S\lambda(S)e^S=\frac{1}{3}(1,1,0,0)+\frac{1}{3}(1,0,1,0)+\frac{1}{3}(1,0,0,1)+\frac{2}{3}(0,1,1,1)=e^N
	\]
	이므로 $\lambda$는 균형 사상이고,
	\[
		\sum_S\lambda(S)v(S)=\frac{3}{4}\biggl(\frac{1}{3}+\frac{1}{3}+\frac{1}{3}+\frac{2}{3}\biggr)=\frac{5}{4}>1=v(N)
	\]
	이므로 $(N,v)$는 균형 게임이 아니다. 따라서 정리 \ref{thm:1.31}에 의해 $C(v)=\varnothing$이다.
\end{sol}

\section{섀플리 값}

\subsection{섀플리 값의 정의}
지금까지는 여러 해 집합의 성질을 확인했다. 이제 각 협력 게임에 하나의 보수 체계를 대응시키는 상황, 즉 단일점 해(one-point solution)를 생각하자. 참가자의 집합 $N$에 대한 모든 협력 게임의 집합을 $\mathcal G^N$이라 하자.

\begin{dfn}{}{2.1}
	$\mathcal G^N$ 위의 \textbf{값(value)}은 함수 $\psi:\mathcal G^N\to\mathbb R^N$이다. $\psi$는 각 협력 게임 $v$를 보수 분배 $\psi(v)$에 대응시킨다.
\end{dfn}

협력 게임이 현실적인 협력과 분열의 상황을 반영함을 고려하면, 우리는 협력 게임의 값을 일관적인 규칙으로 정하고 싶다. 그 중 참가자의 기여도에 따라 보수를 분배하는 방법이 가장 직관적이다. $(N,v)$를 협력 게임이라 하고, $\sigma:N\to N$을 $N$의 순열(일대일대응)이라 하자. 이제 하나의 연합에 참가자 $\sigma(1),\sigma(2),\cdots,\sigma(n)$이 차례대로 합류한다고 생각하자. 참가자가 합류할 때마다 연합이 확장되고, 추가 참가자에 따른 성과 변화를 특성함수로 측정할 수 있다. 참가자 $i$보다 먼저 합류한 참가자의 연합을
\[
	P_\sigma(i)=\{j\in N\mid\sigma^{-1}(j)<\sigma^{-1}(i)\}
\]
로 정의하자. 예를 들어 $N=\{1,2,3,4,5\}$이고 순열 $\sigma:N\to N$이
\[
	\sigma(1)=2,\quad\sigma(2)=5,\quad\sigma(3)=4,\quad\sigma(4)=1,\quad\sigma(5)=3
\]
이면, 참가자 $2$가 먼저 연합에 합류하고, 그 다음에 참가자 $5,4,1,3$의 순서로 합류한다. 따라서 $P_\sigma(1)=\{2,5,4\}$이다. 이제 \textbf{한계 기여도 벡터(marginal vector)} $m^\sigma=(m^\sigma_1,m^\sigma_2,\cdots,m^\sigma_n)$을
\begin{equation}
	\label{eq:2.1}
	m^\sigma_i=v(P_\sigma(i)\cup\{i\})-v(P_\sigma(i))
\end{equation}
로 정의하자. 한계 기여도 벡터 $m^\sigma$는 합류 순서 $\sigma$가 주어질 때, 참가자 $i$의 합류에 의한 가치 변화(참가자 $i$의 기여도)를 나타낸다. 이제 가능한 모든 순열 $\sigma$에 대한 평균치를 택하면 처음에 목표했던 보수 분배를 구성할 수 있다.

\begin{dfn}{}{2.2}
	$\Pi(N)$을 $N$ 위의 모든 순열의 집합이라 하자. 협력 게임 $(N,v)\in\mathcal G^N$의 \textbf{섀플리 값(Shapley value)} $\Phi(v)$를 다음과 같이 정의한다:
	\[
		\Phi(v)=\frac{1}{n!}\sum_{\sigma\in\Pi(N)}m^\sigma.
	\]
\end{dfn}

\begin{example}{}{2.3}
	\begin{enumerate}[(1)]
		\item $2$인 협력 게임 $(\{1,2\},v)$의 섀플리 값은 다음과 같다.
			\[
				\Phi(v)=\biggl(v(1)+\frac{v(N)-v(1)-v(2)}{2},\quad v(2)+\frac{v(N)-v(1)-v(2)}{2}\biggr)
			\]
		\item 가법적 협력 게임 $(N,v)$의 섀플리 값은 $\Phi(v)=(v(1),v(2),\cdots,v(n))$이다. (예시 \ref{ex:1.4} 참고) 
	\end{enumerate}
\end{example}

식 \ref{eq:2.1}을 대입하여 섀플리 값의 $i$ 성분을 생각하자.
\begin{equation}
	\label{eq:2.2}
	\Phi_i(v)=\frac{1}{n!}\sum_{\sigma\in\Pi(N)}[v(P_\sigma(i)\cup\{i\})-v(P_\sigma(i))]
\end{equation}
$\sigma$가 $N$의 순열이므로 $P_\sigma(i)$는 결국 \ref{eq:2.2}의 합에서 $i$를 포함하지 않는 연합 $S$를 대표하게 되고, 그러한 모든 $S$에 대해 $v(S\cup\{i\})-v(S)$를 더하는 것으로 생각할 수 있다. $P_\sigma(i)=S$가 되기 위한 순열 $\sigma$의 수는 $i$ 앞의 $S$를 나열하는 경우의 수 $\vert S\vert!$와 $i$ 뒤의 남은 참가자를 나열하는 경우의 수 $(n-1-\vert S\vert)!$의 곱이다.
\[
	\underbrace{\fbox{\phantom{\{1,1}$S$\phantom{1,1\}}}}_{S\text{를 먼저 나열하고,}}
	\underbrace{\fbox{\phantom{\{}$i$\phantom{\}}}}_{i\text{를 배치한 후,}}\underbrace{\fbox{\phantom{\{1,1}$N\setminus(S\cup i)$\phantom{1,1\}}}}_{\text{남은 참가자를 배치한다.}}
\]
그러므로 식 \ref{eq:2.2}는
\begin{equation}
	\label{eq:2.3}
	\begin{aligned}
		\Phi_i(v)&=\sum_{S\subseteq N:i\notin S}\frac{\vert S\vert!(n-1-\vert S\vert)!}{n!}[v(S\cup i)-v(S)]\\
		&=\sum_{S\subseteq N:i\notin S}\frac{1}{n}\binom{n-1}{\vert S\vert}^{-1}[v(S\cup i)-v(S)]
	\end{aligned}
\end{equation}
로 다시 쓸 수 있다. 식 \ref{eq:2.3}은 한계 기여도의 기댓값으로 정의된 섀플리 값의 새로운 확률론적 해석을 제시한다. 특정 참가자 $i$를 포함하지 않는 연합 $S$를 다음 과정을 따라 구성하자.
\begin{enumerate}[1.]
	\item 연합 $S$의 크기 $\vert S\vert$를 숫자 $0,1,2,\cdots,n-1$를 선택하여 정한다. (균등 확률 $1/n$)
	\item $i$를 제외한 참가자의 집합 $N\setminus\{i\}$에서 $S$에 속할 인원을 정한다. (균등 확률 $\binom{n-1}{\vert S\vert}^{-1}$)
\end{enumerate}
이렇게 택한 연합 $S$에 대하여 $i$에게 $v(S\cup\{i\})-v(S)$ 만큼의 보수를 제시하자. 이 확률 과정에서 $i$가 받게 되는 보수의 기댓값은 정확히 섀플리 값과 일치한다.

\begin{prob}{}{2.4}
	연습문제 \ref{prob:1.28}의 $3$인 협력 게임의 섀플리 값을 구하여라.
\end{prob}

\begin{sol}
	순열 $\sigma\in\Pi(\{1,2,3\}$에 따른 한계 기여도 벡터를 계산하자.
	\begin{table}[H]
		\centering
		\begin{tabular}{c|cccccc}
			\toprule
			$\sigma$ & $1\rightarrow 2\rightarrow 3$ & $1\rightarrow 3\rightarrow 2$ & $2\rightarrow 1\rightarrow 3$ & $2\rightarrow 3\rightarrow 1$ & $3\rightarrow 1\rightarrow 2$ & $3\rightarrow 2\rightarrow 1$\\
			\midrule
			$m^\sigma_1$ & $0$ & $0$ & $1$ & $0$ & $0$ & $0$\\
			\midrule
			$m^\sigma_2$ & $1$ & $1$ & $0$ & $0$ & $1$ & $1$\\
			\midrule
			$m^\sigma_3$ & $0$ & $0$ & $0$ & $1$ & $0$ & $0$\\
			\bottomrule
		\end{tabular}
	\end{table}
	따라서 구하는 섀플리 값은 $(1/6,2/3,1/6)$이다.
\end{sol}

\begin{prob}{}{2.5}
	$N$의 공집합이 아닌 부분집합 $T$에 대하여 \textbf{$T$--표준 게임($T$--standard game)} $1_T$를 다음과 같이 정의하자.
	\[
		1_T(S)=\begin{cases}
			1 & (S=T)\\
			0 & (\text{그 외의 경우})
		\end{cases}
	\]
	섀플리 값 $\Phi(1_T)$을 구하여라.
\end{prob}

\begin{sol}
	식 \ref{eq:2.3}을 이용하자. 먼저, $i\in T$이면 $S=T\setminus\{i\}$일 때에만 $1_T(S\cup i)=1$이므로
	\[
		\Phi_i(1_T)=\sum_{S\ni i}\frac{\vert S\vert!(n-1-\vert S\vert)!}{n!}[1_T(S\cup i)-1_T(S)]=\frac{(\vert T\vert-1)!(n-\vert T\vert)!}{n!}
	\]
	이다. 역으로 $i\notin T$이면 $S=T$일 때에만 $1_T(S)=1$이므로
	\[
		\Phi_i(1_T)=\sum_{S\ni i}\frac{\vert S\vert!(n-1-\vert S\vert)!}{n!}[1_T(S\cup i)-1_T(S)]=-\frac{\vert T\vert!(n-1-\vert T\vert)!}{n!}
	\]
	이다. 섀플리 값은 $T$에 속하지 않은 참가자에게 음의 보수를 제시한다.\qed
\end{sol}


\subsection{섀플리 값의 성질과 특정 (1)}
여기서는 단일점 해가 가졌으면 하는 좋은 성질(공리)을 제시하고, 섀플리 값이 몇몇 공리를 동시에 만족하는 유일한 단일점 해임을 보인다. $\psi:\mathcal G^N\to\mathbb R^N$을 $\mathcal G^N$ 위의 값이라 하자. 효율성은 할당의 정의에서 이미 논하였다.
\begin{itemize}
	\item \textbf{효율성(efficiency, EFF)}: 모든 $v\in\mathcal G^N$에 대하여 $\sum_{i=1}^n\psi_i(v)=v(N)$이다.
\end{itemize}
효율성은 다른 문헌에서 \textbf{파레토 최적성(Pareto optimality)} 또는 \textbf{파레토 효율성(Pareto efficiency)} 등으로 불린다.

모든 연합 $S\subseteq N$에 대하여 $v(S\cup i)-v(S)=0$을 만족하는 참가자 $i\in N$를 \textbf{무임승차자(null-player)}라 하자. 무임승차자는 어떤 연합에 합류시켜도 추가적인 기여를 하지 않는다. 그러므로 우리가 원하는 단일점 해는 무임승차자에게 보수를 지급하지 않아야 한다.
\begin{itemize}
	\item \textbf{무임승차자 배제(null-player property, NP)}: 모든 $v\in\mathcal G^N$의 무임승차자 $i\in N$에 대하여 $\psi_i(v)=0$이다.
\end{itemize}

두 참가자 $i$와 $j$가 모든 연합 $S\subseteq N\setminus\{i,j\}$에 대하여 $v(S\cup i)=v(S\cup j)$를 만족하면 $i$와 $j$가 \textbf{대칭(symmetric)}이라 하자. 즉 $i$와 $j$가 대칭이면 한 연합에 각각 합류할 때의 기여도는 서로 같다. 대칭인 두 참가자에게 같은 보수를 지급함이 바람직하다.
\begin{itemize}
	\item \textbf{대칭성(symmetry, SYM)}: 모든 $v\in\mathcal G^N$의 대칭인 두 참가자 $i,j\in N$에 대하여 $\psi_i(v)=\psi_j(v)$이다.
\end{itemize}

마지막 성질은 둘 이상의 협력 게임을 진행할 때를 위한 성질이다. 가령 어느 하루의 오전에 협력 게임 $(N,v_1)$을 진행하고, 오후에 다른 협력 게임 $(N,v_2)$를 진행한다고 하자. 각각의 게임을 통해 참가자들이 얻는 보수는 $\psi(v_1)+\psi(w_2)$이다. 이는 협력 게임 $(N,v_1+v_2)$를 하루에 진행할 때의 보수 $\psi(v_1+v_2)$와 일치하는 것이 합리적이다.
\begin{itemize}
	\item \textbf{가법성(additivity, ADD)}: 모든 $v_1,v_2\in\mathcal G^N$에 대하여 $\psi(v_1+v_2)=\psi(v)+\psi(w)$이다.
\end{itemize}

이제 섀플리 값이 위의 네 성질을 만족함을 보이자.

\begin{thm}{}{2.6}
	섀플리 값 $\Phi$는 EFF, NP, SYM, ADD를 만족한다.
\end{thm}

\begin{proof}
	$N=\{1,2,\cdots,n\}$, $(N,v)\in\mathcal G^N$이라 하자. $P_\sigma(\sigma(i))=\{j\in N\mid\sigma^{-1}(j)<i\}$이므로
	\[
		\begin{aligned}
			P_\sigma(\sigma(i+1))&=\{j\in N\mid\sigma^{-1}(j)<i+1\}\\
			&=\{j\in N\mid\sigma^{-1}(j)<i\text{ 또는 }\sigma^{-1}(j)=i\}\\
			&=P_\sigma(\sigma(i))\cup\{\sigma(i)\}
		\end{aligned}
	\]
	이다. $\sigma:N\to N$이 일대일대응임을 이용하면
	\[
		\begin{aligned}
			\sum_{i=1}^n\Phi_i(v)&=\sum_{i=1}^n\frac{1}{n!}\sum_{\sigma\in\Pi(N)}[v(P_\sigma(i)\cup i)-v(P_\sigma(i))]\\
			&=\frac{1}{n!}\sum_{\sigma\in\Pi(N)}\sum_{i=1}^n[v(P_\sigma(i)\cup i)-v(P_\sigma(i))]\\
			&=\frac{1}{n!}\sum_{\sigma\in\Pi(N)}\sum_{i=1}^n[v(P_\sigma(\sigma(i))\cup\sigma(i))-v(P_\sigma(\sigma(i)))]\\
			&=\frac{1}{n!}\sum_{\sigma\in\Pi(N)}\sum_{i=1}^n[v(P_\sigma(\sigma(i+1)))-v(P_\sigma(\sigma(i)))]\\
			&=\frac{1}{n!}\sum_{\sigma\in\Pi(N)}[v(N)-v(\varnothing)]=v(N)
		\end{aligned}
	\]
	이므로 $\Phi$는 EFF를 만족한다. 무임승차자 $i\in N$에 대해 $v(P_\sigma(i)\cup\{i\})-v(P_\sigma(i))=0$이므로 $\Phi$는 NP를 만족한다. 이제 $i,j\in N$가 대칭이라 하자. 순열 $\sigma\in\Pi(N)$에 대하여 $i$와 $j$의 합류 순서를 바꾼 순열을 $\sigma^\prime$이라 하자. 즉 $i$와 $j$의 자리를 바꾸는 호환(transposition) $\tau$에 대하여 $\sigma^\prime=\tau\sigma\tau^{-1}$이다.  $P_\sigma(i)=P_{\sigma^\prime}(j)$이고 두 집합은 $i$와 $j$를 포함하지 않으므로 대칭성에 의해
	\[
		\begin{aligned}
			m^\sigma_i-m^{\sigma^\prime}_j&=[v(P_\sigma(i)\cup i)-v(P_\sigma(i))]-[v(P_{\sigma^\prime}(j)\cup j)-v(P_{\sigma^\prime}(j))]\\
			&=v(P_\sigma(i)\cup i) -v(P_{\sigma^\prime}(j)\cup j)=0
		\end{aligned}
	\]
	이고, 그러므로 $\Phi$는 SYM를 만족한다. 마지막으로 $v=v_1+v_2$라 하면
	\[
		v(S\cup i)-v(S)=[v_1(S\cup i)-v_1(S)]+[v_2(S\cup i)-v_2(S)]
	\]
	이므로 $\Phi$는 ADD를 만족한다.
\end{proof}

사실, 섀플리 값은 EFF, NP, SYM, ADD를 만족하는 유일한 단일점 해이다. 이를 보이기 위해 $\mathcal G^N$의 벡터 공간으로서의 성질을 먼저 보이자. 연합 $T\in 2^N\setminus\{\varnothing\}$에 대하여 연습문제 \ref{prob:2.5}의 표준 게임 $1_T$를 고려하자. 임의의 $v\in\mathcal G^N$에 대하여 $v=\sum_{T\ne\varnothing}v(T)1_T$이므로 집합 $\mathcal B_1=\{1_T\mid T\in 2^N\setminus\{\varnothing\}\}$은 $\mathcal G^N$의 기저이다. 이 기저는 간단하지만 섀플리 값의 성질을 규명하는데 도움이 되지 않는다. $1_T$에는 무임승차자가 없기 때문이다. 그 대신 $T$--만장일치 게임의 모임 $\mathcal B_u=\{u_T\mid T\in 2^N\setminus\{\varnothing\}\}$을 고려하자. (예시 \ref{ex:1.22} 참고) $\vert\mathcal B_1\vert=\vert\mathcal B_2\vert=2^{\vert N\vert}-1$이므로 $\mathcal B_u$가 $\mathcal G^N$의 기저임을 보이기 위해서는 $\mathcal B_u$가 선형 독립(linearly independent)임을 보이면 충분하다. $\sum_{T\ne\varnothing}\beta_Tu_T=0$이라 하고, 어떤 $S$에 대하여 $\beta_S\ne0$이라 하자. 그러한 $S$ 중에서 모든 $T\subsetneq S$에 대하여 $\beta_T=0$인 $S$를 택하면 $\sum_{T\ne\varnothing}\beta_Tu_T(S)=\beta_S\ne0$이므로 모순이다. 따라서 $\mathcal B_u$는 선형 독립이다. 이제 다음을 보이자.

\begin{thm}{}{2.7}
	$\mathcal G^N$ 위의 값 $\psi$가 EFF, NP, SYM, ADD를 모두 만족할 필요충분조건은 $\psi=\Phi$이다.
\end{thm}

\begin{proof}
	정리 \ref{thm:2.6}에 의해 $\psi$가 네 성질을 만족한다고 가정하고 $\psi=\Phi$임을 보이면 충분하다. 임의의 협력 게임 $v\in\mathcal G^N$는 $T$--만장일치 게임의 선형 결합으로 (유일하게) 표현할 수 있고, $\psi$와 $\Phi$ 모두 ADD를 만족하므로 모든 $T\in 2^N\setminus\{\varnothing\}$과 $c\in\mathbb R$에 대하여 $\psi(cu_T)=\Phi(cu_T)$임을 보이면 충분하다. 임의의 참가자 $i\in N\setminus T$와 $S\subseteq N$에 대하여 $cu_T(S\cup i)-cu_T(S)=0$이므로 $i$는 무임승차자이다. $\psi$와 $\phi$는 NP를 만족하므로 모든 $i\in N\setminus T$에 대하여
	\begin{equation}
		\label{eq:2.4}
		\psi_i(cu_T)=\Phi_i(cu_T)=0
	\end{equation}
	이다. 이제 $i$와 $j$를 $T$에 속하는 서로 다른 참가자라 하자. $\{i,j\}$를 포함하지 않는 연합 $S$에 대하여 
	\[
		cu_T(S\cup i)=cu_T(S\cup j)=0
	\]
	이므로 $i$와 $j$는 $cu_T$에서 대칭이다. $\psi$와 $\Phi$는 SYM을 만족하므로 모든 $i,j\in T$에 대하여
	\[
		\psi_i(cu_T)=\psi_j(cu_T),\quad\Phi_i(cu_T)=\Phi_j(cu_T)
	\]
	이다. 따라서 EFF로부터 모든 $i\in T$에 대하여
	\begin{equation}
		\label{eq:2.5}
		\psi_i(cu_T)=\Phi_i(cu_T)=\frac{c}{\vert T\vert}
	\end{equation}
	이므로 $\psi=\Phi$이다.
\end{proof}

섀플리 값이 갖는 성질을 더 알아보자. 협력 게임 $(N,v)$의 참가자 $i$가 모든 $S\subseteq N\setminus\{i\}$에 대하여 $v(S\cup i)-v(S)=v(i)$을 만족하면 $i$를 \textbf{더미(dummy player})라 하자. 더미 $i$는 연합에 참가하더라도 추가적인 시너지 효과를 내지 못하므로 기여분 $v(i)$의 보수를 받는 것이 이상적이다.
\begin{itemize}
	\item \textbf{더미 성질(dummy player property, DUM)}: 모든 $v\in\mathcal G^N$의 더미 $i\in N$에 대하여 $\psi_i(v)=v(i)$이다.
\end{itemize}

DUM은 NP를 함의한다. 다음 연습문제를 참고하라.

\begin{prob}{}{2.8}
	$\text{DUM}\Longrightarrow\text{NP}$가 성립함을 보여라.
\end{prob}

\begin{sol}
	무임승차자 $i$는 $v(i)=0$인 더미이므로 DUM은 NP를 함의한다.\qed
\end{sol}

\newpage
\begin{thm}{}{2.9}
	섀플리 값 $\Phi$는 DUM을 만족한다. 
\end{thm}

\begin{proof}
	$i\in N$을 협력 게임 $(N,v)$의 더미라 하면 $P_\sigma(i)$는 $i$를 포함하지 않으므로
	\[
		\Phi_i(v)=\frac{1}{n!}\sum_{\sigma\in\Pi(N)}[v(P_\sigma(i)\cup i)-v(P_\sigma(i))]=\frac{1}{n!}\sum_{\sigma\in\Pi(N)}v(i)=v(i)
	\]
	이므로 섀플리 값은 DUM을 만족한다.
\end{proof}

\begin{remark}
	식 \ref{eq:2.3}을 이용하면
	\[
		\begin{aligned}
			\Phi_i(v)&=\sum_{S\subseteq N\setminus\{i\}}\frac{1}{n}\binom{n-1}{\vert S\vert}^{-1}[v(S\cup i)-v(S)]=\frac{v(i)}{n}\sum_{S\subseteq N\setminus\{i\}}\binom{n-1}{\vert S\vert}^{-1}=v(i)
		\end{aligned}
	\]
	이므로 같은 결과를 얻는다. ($i$를 포함하지 않으면서 크기가 $k$인 연합의 개수가 $\binom{n-1}{k}$임을 이용하라.)
\end{remark}

순열 $\sigma\in\Pi(N)$에 대하여 협력 게임 $v^\sigma\in\mathcal G^N$을
\[
	v^\sigma(S)=v(\sigma^{-1}(S)),
\]
함수 $\sigma^*:\mathbb R^N\to\mathbb R^N$를 $\sigma^*_i(x)=x_{\sigma^{-1}(i)}$로 정의하자.
\begin{itemize}
	\item \textbf{익명성(anonimity, AN)}: 모든 $v\in\mathcal G^N$와 $\sigma\in\Pi(N)$에 대하여 $\psi(v^\sigma)=\sigma^*(\psi(v))$이다.
\end{itemize}
순열 $\sigma:N\to N$이 참가자의 이름(또는 번호)을 바꾼다고 생각하자. 협력 게임 $v^\sigma$에서의 연합 $S$와 협력 게임 $v$에서의 연합 $\sigma^{-1}(S)$는 구성원의 이름만 다를 뿐 서로 같다. 따라서 $v^\sigma$는 연합 $S$에게 $v(\sigma^{-1}(S))$ 만큼의 가치를 책정한다. $\mathcal G^N$ 위의 값 $\psi$가 익명성을 가지면 각 $i\in N$에 대하여 $\psi_i(v^\sigma)=\sigma^*_i(\psi(v))=\psi_{\sigma^{-1}(i)}(v)$이므로 참가자 $i$는 $v^\sigma$에서 과거의 이름 $\sigma^{-1}(i)$으로서 $v$에 대한 보수를 받는다. 즉 익명성은 참가자를 이름, 번호만으로 차별하지 아니함을 의미한다.

AN은 SYM을 함의한다. 다음 연습문제를 참고하라.

\begin{prob}{}{2.10}
	$\text{AN}\Longrightarrow\text{SYM}$가 성립함을 보여라.
\end{prob}

\begin{sol}
	$\psi$를 AN을 만족하는 $\mathcal G^N$ 위의 값이라 하자. 두 참가자 $i$와 $j$가 서로 대칭이라 하고, 순열 $\sigma$를 $i$와 $j$를 교환하는 호환이라 하자. 연합 $S$가 $S\subset N\setminus\{i,j\}$이거나 $\{i,j\}\subseteq S$를 만족하면 $\sigma^{-1}(S)=S$이므로 $v^\sigma(S)=v(\sigma^{-1}(S))=v(S)$이다. 만약 $i\in S$이고 $j\notin S$이면 $S\setminus\{i\}\subseteq N\setminus\{i,j\}$이므로
	\[
		v^\sigma(S)=v(\sigma^{-1}(S))=v(S\setminus\{i\}\cup\{j\})=v(S\setminus\{i\}\cup\{i\})=v(S)
	\]
	이다. 같은 논리로 $i\notin S$이고 $j\in S$이면 $v^\sigma(S)=v(S)$이다. 즉 $v^\sigma=v$이므로
	\[
		\psi_i(v)=\psi_{\sigma^{-1}(j)}(v)=\psi_j(v^\sigma)=\psi_j(v)
	\]
	이 되어 $\psi$는 SYM을 만족한다.\qed
\end{sol}

\newpage
\begin{thm}{}{2.11}
	섀플리 값 $\Phi$는 AN을 만족한다.
\end{thm}

\begin{proof}
	먼저 모든 $v\in\mathcal G^N$과 $\tau,\sigma\in\Pi(N)$에 대하여
	\[
		\tau^*(m^\sigma(v))=m^{\tau\sigma}(v^\tau)
	\]
	임은 다음 계산을 통해 확인할 수 있다.
	
	\[
		\begin{aligned}
			\tau^*(m^\sigma(v))_{\tau\sigma(i)}&=m^\sigma(v)_{\sigma(i)}\\
			&=v(P_\sigma(\sigma(i))\cup\sigma(i))-v(P_\sigma(i))\\
			&=v(\sigma(\{1,2,\cdots,i\}))-v(\sigma(\{1,2,\cdots,i-1\}))\\
			&=v(\tau^{-1}(\tau\sigma(\{1,2,\cdots,i\})))-v(\tau^{-1}(\tau\sigma(\{1,2,\cdots,i-1\})))\\
			&=v^\tau(\tau\sigma(\{1,2,\cdots,i\}))-v^\tau(\tau\sigma(\{1,2,\cdots,i-1\}))\\
			&=v^\tau(P_{\tau\sigma}(\tau\sigma(i))\cup\tau\sigma(i))-v^\tau(P_{\tau\sigma}(\tau\sigma(i)))\\
			&=m^{\tau\sigma}(v^\tau)_{\tau\sigma(i)}
		\end{aligned}
	\]
	이제 $v\in\mathcal G^N$, $\tau\in\Pi(N)$이라 하자. 사상 $\tau\mapsto\tau\sigma$는 $\Pi(N)$ 위의 전사 사상이므로
	\[
		\Phi(v^\tau)=\frac{1}{n!}\sum_{\sigma\in\Pi(N)}m^\sigma(v^\tau)=\frac{1}{n!}\sum_{\sigma}m^{\tau\sigma}(v^\tau)=\frac{1}{n!}\sum_\sigma\tau^*(m^\sigma(v))=\tau^*\biggl(\frac{1}{n!}\sum_\sigma m^\sigma\biggr)=\tau^*(\Phi(v))
	\]
	이 성립한다. 따라서 $\Phi$는 AN을 만족한다.
\end{proof}

\begin{prob}{}{2.12}
	이 연습문제에서는 조건 EFF, DUM, SYM, ADD 중 하나라도 누락되면 정리 \ref{thm:2.7}이 성립하지 않음을 보인다. $(N,v)\in\mathcal G^N$이라 하자.
	\begin{enumerate}[(a)]
		\item 각 $i\in N$에 대하여
			\[
				\psi_i(v)=v(i)
			\]
			라 하면 $\psi$는 DUM, SYM, ADD를 만족하지만 EFF는 만족하지 않음을 보여라.
		\item 각 $i\in N$에 대하여 
			\[
				\psi_i(v)=\sum_{\sigma:\sigma(1)=1}\frac{1}{(\vert N\vert-1)!}m^\sigma_i
			\]
			라 하면 $\psi$는 EFF, DUM, ADD를 만족하지만 SYM은 만족하지 않음을 보여라.
		\item 각 $i\in N$에 대하여
			\[
				\psi_i(v)=\frac{v(N)}{\vert N\vert}
			\]
			이라 하면 $\psi$는 EFF, SYM, ADD를 만족하지만 DUM은 만족하지 않음을 보여라.
		\item $D(v)$를 $v$의 모든 더미의 집합이라 하자. 각 $i\in N$에 대하여
			\[
				\psi_i(v)=\begin{cases}
					v(i) & (i\in D(v))\\[3pt]
					\displaystyle\frac{1}{\vert N\setminus D(v)\vert}\biggl(v(N)-\sum_{j\in D(v)}v(j)\biggr) & (i\in N\setminus D(v))
				\end{cases}
			\]
			라 하면 $\psi$는 EFF, DUM, SYM을 만족하지만 ADD는 만족하지 않음을 보여라.
	\end{enumerate}
\end{prob}

\begin{sol}
	\begin{enumerate}[(a)]
		\item $\psi$가 DUM, ADD를 만족함은 자명하다. 두 참가자 $i,j\in N$가 대칭이면 $v(i)=v(j)$이므로 $\psi$는 SYM을 만족한다. (정의에 $S=\varnothing$을 대입하라.) $\psi$가 EFF를 만족하지 않음은 표준 게임 $1_N$으로 보일 수 있다:
			\[
				\sum_{i\in N}\psi_i(v)=\sum_{i\in N}v(i)=0\ne 1=v(N).
			\]
		\item $\psi$가 EFF, DUM, ADD를 만족함은 정리 \ref{thm:2.6}, \ref{thm:2.9}와 같은 방법으로 확인할 수 있다. $\psi$가 SYM을 만족하지 않음을 보이기 위해 $N=\{1,2,3\}$, $T=\{1,2\}\subseteq N$이라 하고 $(N,1_T)$를 $T$--표준 게임이라 하자.
			\[
				v(\varnothing\cup i)=v(\varnothing\cup j)=0,\quad v(1,3)=v(2,3)=0
			\]
			이므로 두 참가자 $1$과 $2$는 대칭이다. 그러나 $\psi_1(1_T)=0$, $\psi_2(1_T)=1/2$이므로 $\psi$는 SYM을 만족하지 않는다.
		\item $\psi$가 EFF, SYM, ADD를 만족함은 자명하다. 연습문제 \ref{prob:2.8}로부터 $\psi$가 NP를 만족하지 않음을 보이면 충분하다. $N=\{1,2,3\}$, $T=\{1,2\}\subseteq N$에 대하여 $(N,u_T)$를 $T$--만장일치 게임이라 하자. 모든 $S\subseteq\{1,2\}$에 대하여 $v(S\cup 3)-v(S)=0$이므로 참가자 $3$은 무임승차자이다. 그러나 $\psi_3(u_T)=1/3$이므로 $\psi$는 DUM을 만족하지 않는다.
		\item $\psi$가 EFF, DUM을 만족함은 자명하다. $\psi$가 SYM을 만족함은 다음을 보이면 충분하다.
			\begin{lemma*}{}{}
				두 참가자가 대칭이면, 둘은 모두 더미이거나 모두 더미가 아니다.
			\end{lemma*}
			대칭인 두 참가자 $i,j\in N$에 대하여 $i\in D(v)$라 하자. $i$와 $j$의 대칭성에 의해 $v(i)=v(j)$이다. 임의의 $S\subseteq N\setminus\{j\}$에 대하여 $S\setminus\{i\}\subseteq N\setminus\{i,j\}$이므로
			\[
				\begin{aligned}
					v(S\cup j)&=v([S\setminus\{i\}\cup i]\cup j)=v([S\setminus\{i\}\cup j]\cup i)\\
					&=v(S\setminus\{i\}\cup j)+v(i)=v(S\setminus\{i\}\cup i)+v(j)\\
					&=v(S)+v(j)
				\end{aligned}
			\]
			이 성립한다. 즉 $j\in D(v)$이다. 이제 $\psi$가 ADD를 만족하지 않음을 보이자. $N=\{1,2,3\}$에 대하여 $u_{\{1,2\}}$와 $u_{\{2,3\}}$을 각각 만장일치 게임이라 하고, $u=u_{\{1,2\}}+u_{\{2,3\}}$이라 하자.
			\[
				D(u_{\{1,2\}})=\{3\},\quad D(u_{\{2,3\}})=\{1\},\quad D(u)=\varnothing
			\]
			이므로 $\psi(u_{\{1,2\}})$, $\psi(u_{\{2,3\}})$, $\psi(u)$는 다음과 같다.
			\begin{table}[H]
				\centering
				\begin{tabular}{c|ccc}
					\toprule
					$v\in\mathcal G^N$ & $\psi_1(v)$ & $\psi_2(v)$ & $\psi_2(v)$\\
					\midrule
					$u_{\{1,2\}}$ & $1/2$ & $1/2$ & $0$\\
					$u_{\{2,3\}}$ & $0$ & $1/2$ & $1/2$\\
					$u$ & $2/3$ & $2/3$ & $2/3$\\
					\bottomrule
				\end{tabular}
			\end{table}
			따라서 $\psi$는 ADD를 만족하지 않는다.
	\end{enumerate}
\end{sol}

\begin{thebibliography}{9}
	\bibitem{1}
		Peters, Hans. Game Theory: A Multi-Leveled Approach. 2nd ed., Springer (2015).\\
		\url{https://doi.org/10.1007/978-3-662-46950-7}
	\bibitem{2}
		Osborne, Martin J., Ariel Rubinstein. A Course in Game Theory. MIT Press (1994).
	\bibitem{3}
		Rafels, C., Tijs, S. On the Cores of cooperative games and the stability of the Weber set. Int J Game Theory 26, 491–499 (1997). \url{https://doi.org/10.1007/BF01813887}
\end{thebibliography}
\end{document}
