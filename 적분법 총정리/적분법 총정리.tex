\documentclass{article}
\input{"D:/OneDrive/LaTeX/preamble/preamble.tex"}

\geometry{a4paper, total={6.4in, 10in}}
\title{적분법 총정리}

\begin{document}
\setstretch{1.3}
\maketitle

\noindent 고등학교 수준에서 배우는 적분법을 크게 치환적분과 부분적분의 두 가지이다. 두 방법 모두 미분법에서 파생하여 이해하기 쉬우나, 문제에서 실제로 적용하기는 어렵다. 특히 적분 공식을 다양하게 이용해야 하는 문제에서는 헤맬 수 있다. 이 칼럼에서는 적분법의 사용 기준을 정립하고 미분법과 적분법에서 파생하는 잡기술을 정리할 것이다. $\langle$수학II$\rangle$에서 학습하는 부정적분/정적분의 성질을 충분히 숙지하고 있음을 가정하고 $\langle$미적분$\rangle$에서 추가로 학습하는 적분법에 초점을 맞추어 설명한다.

\section{기본 적분 공식}
함수 $f(x)$의 한 부정적분을 $F(x)$라 하면, $f(x)$의 모든 부정적분은 어떤 상수 $C$에 대해
\[
	\int f(x)\,dx=F(x)+C
\]
로 표현된다. 부정적분은 미분의 역과정이므로 미분 공식을 이용하여 적분 공식을 도출할 수 있다.

\begin{thm}{$x^n$의 부정적분}{1.1}
	$x^n$의 부정적분은 $-1$을 기준으로 공식이 다르다.
	\[
		\int x^n\,dx=\frac{x^n}{n+1}+C~(n\ne-1)\quad\int\frac{1}{x}\,dx=\ln\vert x\vert+C
	\]
\end{thm}

\begin{thm}{지수함수의 부정적분}{1.2}
	지수함수의 부정적분은 자기 자신의 상수배이다.
	\[
		\int_e^x\,dx=e^x+C,\quad\int a^x\,dx=\frac{a^x}{\ln a}+C~(a>0,a\ne1)
	\]
\end{thm}

\begin{thm}{삼각함수의 부정적분}{1.3}
	삼각함수의 부정적분에서는 순환 형태가 등장한다.
	\[
		\begin{array}{ll}
			\displaystyle\int\sin x\,dx=-\cos x+C & \displaystyle\int\cos x\,dx=\sin x+C \\[9pt]
			\displaystyle\int\sec^2x\,dx=\tan x+C & \displaystyle\int \csc^2x\,dx=-\cot x+C \\[9pt]
			\displaystyle\int\sec x\tan x\,dx=\sec x+C & \displaystyle\int\csc x\cot x\,dx=-\csc x+C
		\end{array}
	\]
\end{thm}

\begin{exc}{2019학년도 수능특강}{1.4}
	$\displaystyle\int_1^{\sqrt2}\biggl(\sqrt x-\frac{1}{x\sqrt x}\biggr)^2\,dx$의 값은?
\end{exc}

\begin{sol}
	\[
		\int_1^{\sqrt2}\biggl(\sqrt x-\frac{1}{x\sqrt x}\biggr)^2\,dx=\int_1^{\sqrt2}\biggl(x-\frac{2}{x}+\frac{1}{x^3}\biggr)\,dx=\biggl[\frac{x^2}{2}-2\ln\vert x\vert-\frac{1}{2x^2}\biggr]_1^{\sqrt2}=\frac{3}{4}-\ln2\qed
	\]
\end{sol}

\begin{exc}{2015년 5월 전북교육청 B형 3번}{1.5}
	$\displaystyle\int_0^{\ln2}\frac{e^{2x}}{e^x+1}\,dx-\int_0^{\ln2}\frac{1}{e^x+1}\,dx$의 값은? (단, $e$는 자연로그의 밑이다.) [2점]
\end{exc}

\begin{sol}
	\begin{align*}
		\int_0^{\ln2}\frac{e^{2x}}{e^x+1}\,dx-\int_0^{\ln2}\frac{1}{e^x+1}\,dx&=\int_0^{\ln2}\frac{e^{2x}-1}{e^x+1}\,dx=\int_0^{\ln2}\frac{(e^x+1)(e^x-1)}{e^x+1}\,dx\\[3pt]
		&=\int_0^{\ln2}(e^x-1)\,dx=\biggl[e^x-x\biggr]_0^{\ln2}=1-\ln2\qed
	\end{align*}
\end{sol}

\begin{exc}{2024학년도 수능특강}{1.6}
	$0<\theta<\dfrac{\pi}{4}$인 실수 $\theta$에 대하여 $\displaystyle\int_0^{\tfrac{\pi}{2}-\theta}\frac{1}{\sin^2\cos^2x}\,dx=3$일 때, $\sin^2\theta$의 값은?
\end{exc}

\begin{sol}
	먼저 주어진 정적분을 계산하면
	\begin{align*}
		\int_0^{\tfrac{\pi}{2}-\theta}\frac{1}{\sin^2\cos^2x}\,dx&=\int_0^{\tfrac{\pi}{2}-\theta}\frac{\sin^2x+\cos^2x}{\sin^2x\cos^2x}\,dx=\int_0^{\tfrac{\pi}{2}-\theta}\biggl(\frac{1}{\cos^2x}+\frac{1}{\sin^2x}\biggr)\,dx\\[3pt]
		&=\int_0^{\tfrac{\pi}{2}-\theta}(\sec^2x+\csc^2x)\,dx=\biggl[\tan x-\cot x\biggr]_0^{\tfrac{\pi}{2}-\theta}\\[3pt]
		&=\tan\biggl(\frac{\pi}{2}-\theta\biggr)-\cot\biggl(\frac{\pi}{2}-\theta\biggr)-\tan\theta+\cot\theta\\[3pt]
		&=\cot\theta-\tan\theta-\tan\theta+\cot\theta\\[3pt]
		&=\frac{2}{\tan\theta}-2\tan\theta.
	\end{align*}
	그러므로
	\[
		\frac{2}{\tan\theta}-2\tan\theta=3~\longrightarrow~2\tan^2\theta+3\tan\theta-2=0~\longrightarrow~(2\tan\theta-1)(\tan\theta+2)=0
	\]
	에서 $0<\theta<\dfrac{\pi}{4}$이므로 $\tan\theta=\dfrac{1}{2}$이고 $\sin^2\theta=1-\cos^2\theta=1-\dfrac{1}{1+\tan^2\theta}=\dfrac{1}{5}$이다.\qed
\end{sol}

\begin{remark}
	삼각함수가 포함된 적분은 삼각함수의 성질을 적극적으로 이용한다.
\end{remark}

\section{치환적분법 -- 도함수가 곱해져 있다}
함수 $f(x)$의 한 부정적분을 $F(x)$라 하면
\[
	\frac{dF}{dx}=f,\quad\int f(x)\,dx=F(x)+C
\]
가 성립한다. 이때 미분가능하고 일대일대응인 함수 $g$에 대하여 $x=g(t)$이면, $F(x)=F(g(t))$이므로 합성함수의 미분법에 의하여
\[
	\frac{d}{dt}F(x)=\frac{d}{dt}(F\circ g)(t)=\frac{dF}{dx}(g(t))\times\frac{dg}{dt}(t)=f(g(t))g^{\prime}(t)
\]
이므로 $f(g(t))g^{\prime}(t)$는 $t$에 대한 $F$의 한 부정적분이다. 따라서 
\[
	\int f(g(t))g^{\prime}(t)\,dt=F(x)+C=F(g(t))+C
\]
으로부터 다음을 얻는다.
\begin{thm}{치환적분법}{2.1}
	\[
		\int f(x)\,dx=\int f(g(t))g^{\prime}(t)\,dt
	\]
\end{thm}

정적분을 계산할 때는 적분 범위가 함수 $g$에 의해 바뀐다. $a=g(\alpha)$, $b=g(\beta)$라 할 때,
\[
	\int_a^bf(x)\,dx=F(b)-F(a)=F(g(\beta))-F(g(\alpha))=\int_{g(\alpha)}^{g(\beta)}f(g(t))g^{\prime}(t)\,dt
\]
가 성립한다. 공식의 유도 과정에서는 좌변에서 치환 $x=g(t)$을 이용하였으나, 이를 적분법으로 이용하기 위해서는 우변의 형태에서 $x=g(t)$로 치환하여 좌변의 간단한 형태를 만든다.
\begin{align*}
	\int f(x)\,dx~&\xrightarrow[\text{적분법의 유도}]{x=g(t):~t\text{의 도입}}~\int f(g(t))g^{\prime}(t)\,dt\\[6pt]
\int f(x)\,dx~&\xleftarrow[\text{적분법의 적용}]{x=g(t):~x\text{의 도입}}~\int f(g(t))g^{\prime}(t)\,dt
\end{align*}

따라서 치환적분법을 적용하기 위한 적절한 타이밍은 어떤 함수 $f$의 속에 들어있는 함수 $g$의 도함수 $g^{\prime}$이 식 전체에 곱해져 있는 것을 발견할 때이다. $g$가 일차함수인 경우에는 특히 도함수 $g^{\prime}$이 상수이므로 도함수가 항상 곱해져 있다고 해석할 수 있다. 이와 더불어 정적분에서 적분 구간이 바뀌는 점을 이용하면 적절한 일차함수 $g$를 이용하여 적분 범위를 임의로 조작할 수 있다.
\[
	\int_a^bf(x)\,dx~\xrightarrow{t=\tfrac{\beta-\alpha}{b-a}(x-a)+\alpha}~\frac{b-a}{\beta-\alpha}\int_\alpha^\beta f\biggl(\frac{b-a}{\beta-\alpha}(t-\alpha)+a\biggr)dt
\]

\begin{exc}{2022학년도 수능특강}{2.2}
	$\displaystyle\int_{\tfrac{\pi}{4}}^{\tfrac{\pi}{2}}\frac{\sin x-\cos x}{\sin x+\cos x}\,dx$의 값은?
\end{exc}

\begin{sol}
	$(\sin x +\cos x)^{\prime}=\cos x-\sin x$이므로 $t=\sin x+\cos x$라 하면
	\[
		\int_{\tfrac{\pi}{4}}^{\tfrac{\pi}{2}}\frac{\sin x-\cos x}{\sin x+\cos x}\,dx=\int_{\sqrt2}^1\biggl(-\frac{1}{t}\biggr)dt=\int_1^{\sqrt2}\frac{1}{t}dt=\biggl[\ln\vert t\vert\biggr]_1^{\sqrt2}=\ln\sqrt2.\qed
	\]
\end{sol}

\begin{remark}
	분모에 있는 함수의 도함수가 분자에 곱해져 있는 형태는 자주 출제된다.
\end{remark}

\begin{exc}{2013학년도 수능 가형 12번}{2.3}
	연속함수 $f(x)$가
	\[
		f(x)=e^{x^2}+\int_0^1tf(t)\,dt
	\]
	를 만족시킬 때, $\displaystyle\int_0^1xf(x)\,dx$의 값은? [3점]
\end{exc}

\begin{sol}
	$\displaystyle\int_0^1xf(x)\,dx=k$라 하자. 그러면 $f(x)=e^{x^2}+k$이므로
	\[
		\int_0^1xf(x)\,dx=\int_0^1(xe^{x^2}+kx)\,dx=\biggl[\frac{1}{2}e^{x^2}+\frac{k}{2}x^2\biggr]_0^1=\frac{e+k-1}{2}
	\]
	이므로 $k=\dfrac{e+k-1}{2}$에서 $k=e-1$이다.\qed
\end{sol}

\begin{remark}
	도함수가 곱해져 있다는 점을 기억하면 치환하는 과정없이 암산으로 치환적분을 계산할 수 있다. 부정적분의 암산이 가능하면 정적분에서 적분 범위를 바꾸는 행위를 생략할 수 있다.
\end{remark}

\begin{exc}{2012학년도 4월 학력평가 가형 16번}{2.4}
	정적분 $\displaystyle\int_0^{\tfrac{\pi}{2}}\sin 2x(\sin x+1)\,dx$의 값은? [3점]
\end{exc}

\begin{sol}
	\[
		\int_0^{\tfrac{\pi}{2}}\sin 2x(\sin x+1)\,dx=\int_0^{\frac{\pi}{2}}2\cos x\sin x(\sin x+1)\,dx=\biggl[\frac{2}{3}\sin^3x+\sin x\biggr]_0^{\tfrac{\pi}{2}}=\frac{5}{3}\qed
	\]
\end{sol}

\section{부분적분법 -- 미분/적분하기 쉬운 함수}
미분가능한 두 함수 $f(x)$, $g(x)$의 곱 $f(x)g(x)$의 도함수는
\[
	\{f(x)g(x)\}^{\prime}=f^{\prime}(x)g(x)+f(x)g^{\prime}(x)
\]
이다. 이를 이용하면 함수의 곱을 적분할 수 있는 방법인 부분적분법을 얻게 된다.

\begin{thm}{부분적분법}{3.1}
	\['
		\int f(x)g^{\prime}(x)\,dx=f(x)g(x)-\int f^{\prime}(x)g(x)\,dx
	\]
\end{thm}

정적분을 계산하기 위해서는 적분 범위만 추가로 생각하면 된다.
\[
	\int_a^b f(x)g^{\prime}(x)\,dx=\biggl[f(x)g(x)\biggr]_a^b-\int_a^b f^{\prime}(x)g(x)\,dx
\]

좌변에서 우변으로 넘어가는 과정에서, 함수 $f(x)$는 미분을, 함수 $g^{\prime}(x)$는 적분을 해야 한다. 일반적으로 부정적분의 계산이 도함수의 계산보다 어려우므로 $g^{\prime}(x)$의 역할을 할 함수를 먼저 정해야 한다. 이때 곱의 각 부분에서 적분하기 쉬운 부분을 택해야 한다. 정리하면 두 함수 곱에서 (상대적으로) 미분하기 쉬운 부분과 적분하기 쉬운 부분으로 구분하여 부분적분법을 적용한다. 다항함수와 초월함수(지수/로그/삼각함수)가 서로 곱해진 함수의 적분에서는 일반적으로 [로다삼지]를 암기하여 부분적분을 할 부분을 정한다. 그러나 ‘미분하기 쉬운 부분, 적분하기 쉬운 부분’만 생각하고 있으면 [로다삼지]를 암기할 필요가 없다. 특히 다항함수와 지수함수는 도함수와 부정적분이 모두 간단하게 계산되어 부분적분에서 역할이 다소 자유롭기 때문에, [로다삼지]가 무조건 통하지는 않는다.

\begin{exc}{2011년 10월 대전교육청 가형 29번}{3.2}
	다음 그림은 미분가능한 함수 $f(x)$의 그래프이다. $f(0)=0$, $f(2)=3$이고, $y=f(x)$와 $x$축 및 직선 $y=2$로 둘러싸인 부분의 넓이가 $1$일 때, 정적분 $\displaystyle\int_0^22xf^{\prime}(x)\,dx$의 값을 구하시오. [4점]
	\begin{figure}[H]
		\centering
		\includegraphics[scale=0.6]{2011년 10월 대전교육청 가형 29번}
	\end{figure}
\end{exc}

\begin{sol}
	넓이 조건에서 $\displaystyle\int_0^2f(x)\,dx=1$이다.
	\[
		\int_0^22xf^{\prime}(x)\,dx=\biggl[2xf(x)\biggr]_0^2-\int_0^22f(x)\,dx=4f(2)-\int_0^22f(x)\,dx=10\qed
	\]
\end{sol}

\begin{remark}
	$f^{\prime}(x)$는 적분하면 $f(x)$임을 곧바로 알 수 있는 반면, $f^{\prime\prime}(x)$는 문제의 조건에서 알 수 없으므로 $2x\times f^{\prime}(x)$에서 $f^{\prime}(x)$를 적분하기 쉬운 함수로 분류해야 한다.
\end{remark}

\begin{exc}{2017학년도 사관학교 가형 18번}{3.3}
	함수 $f(x)=\displaystyle\int_1^xe^{t^3}dt$에 대하여 $\displaystyle\int_0^1xf(x)\,dx$의 값은? [4점]
\end{exc}

\begin{sol}
	함수 $f(x)$의 정의에 의하여 $f(1)=0$, $f^{\prime}(x)=e^{x^3}$이다.
	\[
		\int_0^1xf(x)\,dx=\biggl[\frac{1}{2}x^2f(x)\biggr]_0^1-\int_0^1\frac{1}{2}x^2f^{\prime}(x)\,dx=-\frac{1}{2}\int_0^1x^2e^{x^3}dx=-\frac{1}{2}\biggl[\frac{1}{3}e^{x^3}\biggr]_0^1=\frac{1-e}{6}\qed
	\]
\end{sol}

\begin{remark}
	함수 $f(x)$는 정적분으로 정의되어 있으므로 적분보다 미분이 편하다. 따라서 $f(x)$를 미분하기 쉬운 함수로 분류해야 한다.
\end{remark}

\section{부분적분법의 연속 적용 -- Tabular Integration}
부분적분법의 공식을 다시 보자.
\[
		\int\underbrace{f(x)}_{\text{미분하기 쉬운 함수}}\quad\underbrace{g^{\prime}(x)}_{\text{적분하기 쉬운 함수}}\,dx\quad=\quad\underbrace{f(x)}_{\text{그대로}}\quad\underbrace{g(x)}_{\text{적분}}\quad-\quad\int \quad\underbrace{f^{\prime}(x)}_{\text{미분}}\quad\underbrace{g(x)}_{\text{적분}}\quad dx
\]
여기서 우변의 $\displaystyle\int f^{\prime}(x)g(x)\,dx$ 역시 두 함수의 곱으로 이루어져 있으므로 부분적분법의 적용을 고려할 수 있다. 이와 같이 부분적분법을 두 번 이상 적용해야 할 때 위의 공식에만 의존하기에는 식이 길어지므로 부분적분법의 과정을 도표로 만들어 연속적인 적용까지 고려한다. 이를 Tabular Integration이라 한다. 먼저 위의 공식을 참고하여 다음과 같이 표를 그려넣는다. $D$는 Differentiation(미분)을 의미하고, $I$는 Integration(적분)을 의미한다.
\[
	\begin{tikzcd}
		D  & I \\[-12pt]
		f(x) & g^\prime(x)
	\end{tikzcd}
\]
바로 아래에 $f$의 도함수 $f^{\prime}$과 $g^{\prime}$의 한 부정적분 $g$를 적고 다음과 같이 화살표를 그려넣고 부호를 번갈아 적는다.
\[
	\begin{tikzcd}
		D & I \\[-12pt]
		f(x) \arrow[rd, "+"] & g(x) \\
		f^\prime(x) \arrow[r, "-"] & g(x)
	\end{tikzcd}
\]
대각선 아래 방향 화살표는 부호와 함께 곱한다. 가로 화살표는 부호와 함께 곱하고 적분한다.
\[
	\begin{tikzcd}
		D & I & \\[-12pt]
		f(x) \arrow[rd, "+"] & g(x) & \\
		f^\prime(x) \arrow[r, "-"] & g(x) \arrow[r] \arrow[rd] & -\displaystyle\int f^\prime(x)g(x)\,dx \\
		& & f(x)g(x)
	\end{tikzcd}
\]
이제 곱한 것들을 모두 더하면 부분적분법을 시행한 것이 된다.
\[
	\int f(x)g^{\prime}(x)\,dx=f(x)g(x)-\int f^{\prime}(x)g(x)\,dx
\]
Tabular Integration의 장점을 이 도표를 계속 연장하여 연속 적용할 수 있다는 점이다. 위의 도표에서 함수 $g(x)$의 한 부정적분을 $G(x)$라 하자. 이제 $\displaystyle\int f^\prime(x)g(x)\,dx$에도 부분적분을 적용하고 싶다면 위의 도표를 그대로 연장하면 된다. 이때 화살표의 방향과 부호가 번갈아 나옴에 주의한다.

\[
	\begin{tikzcd}
		D & I \\[-12pt]
		f(x) \arrow[rd, "+"] & g^\prime(x) \\
		f^\prime(x) \arrow[rd, "-"] & g(x) \\
		f^{\prime\prime}(x) \arrow[r, "+"] & G(x)
	\end{tikzcd}
\]
같은 방법으로 화살표와 부호에 따라 적절히 곱하면 다음을 얻는다.
\[
	\int f(x)g^{\prime}(x)\,dx=f(x)g(x)-f^{\prime}(x)G(x)+\int f^{\prime\prime}(x)G(x)\,dx
\]
이제 다양한 예제를 보면서 도표적분법을 적용해보자.

\begin{exc}{다항함수가 $D$에 있는 경우 [1]}{4.1}
	$\displaystyle\int x\cos x\,dx$를 구하시오.
\end{exc}

\begin{sol}
	다항함수와 삼각함수의 곱으로 이루어져 있다. 이때 다항함수는 미분하면 차수가 낮아지고 결국 $0$이 되는 반면, 삼각함수는 미분/적분에 관계없이 항상 삼각함수의 형태로 남는다. 따라서 $D$에 $x$, $I$에 $\cos x$를 배치하여 Tabular Integration을 시행하자.
	\[
		\begin{tikzcd}
			D & I \\[-12pt]
			x \arrow[rd, "+"] & \cos x \\
			1 \arrow[rd, "-"] & \sin x \\
			0 & -\cos x
		\end{tikzcd}
	\]
	미분하는 쪽에서 결국 $0$이 나왔으므로 가로 화살표 $\xrightarrow{+}$는 생략 가능하다. 화살표를 따라 곱해도 $0$으로 무의미하기 때문이다. 이처럼 $D$에 다항함수가 배치되는 경우에는 마지막의 가로 화살표를 생략할 수 있다. 이제 도표의 화살표에 따라 곱하면 구하는 부정적분은
	\[
		\int x\cos x\,dx = x\sin x+\cos x +C.\qed
	\]
\end{sol}

\newpage
\begin{exc}{다항함수가 $D$에 있는 경우 [2]}{4.2}
	$\displaystyle\int x^2e^{2x}\,dx$를 구하시오.
\end{exc}

\begin{sol}
	\[
		\begin{tikzcd}
			D & I \\[-12pt]
			x^2 \arrow[rd, "+"] & e^{2x} \\
			2x \arrow[rd, "-"] & \dfrac{1}{2}e^{2x} \\
			2 \arrow[rd, "+"] & \dfrac{1}{4}e^{2x} \\
			0 & \dfrac{1}{8}e^{2x}
		\end{tikzcd}\qquad\int x^2e^{2x}\,dx=\frac{1}{2}x^2e^{2x}-\frac{1}{2}e^{2x}+\frac{1}{4}e^{2x}+C\qed
	\]
\end{sol}

\begin{exc}{로그함수가 $D$에 있는 경우 [1]}{4.3}
	$\displaystyle\int\ln x\,dx$를 구하시오.
\end{exc}

\begin{sol}
	로그함수의 부정적분을 구하기 위해 $1\times\ln x$의 형태로 보고 $D$에 $\ln x$, $I$에 $1$을 두어 계산한다. $D$에 다항함수가 배치되는 것이 아니므로 미분을 계속해도 $0$이 나오지 않는다. 따라서 도표도 다음에 그친다.
	\[
		\begin{tikzcd}
			D & I \\[-12pt]
			\ln x \arrow[rd, "+"] & 1 \\
			\dfrac{1}{x} \arrow[r, "-"] & x
		\end{tikzcd}
	\]
	이제 부분적분법에 따라 $1/x$와 $x$의 곱인 $1$을 적분해야 한다. 이를 식으로 쓰지 않고 다음과 같이 오른쪽에 곱만 미리 표시해두자. (가로 화살표는 잠시 생략한다.)
	\[
		\begin{tikzcd}
			D & I & \\[-12pt]
			\ln x \arrow[rd, "+"] & 1 & \\
			\dfrac{1}{x} \arrow[r, "-"] & x \arrow[r] & 1
		\end{tikzcd}
	\]
	이제 $1$을 적분하기 위해 위의 도표를 연장해서 이용하자. $D$에  $1$, $I$에 $1$을 배치하자. 화살표와 부호 교대는 그대로 유지한다.
	\[
		\begin{tikzcd}
			D & I & \\[-12pt]
			\ln x \arrow[rd, "+"] & 1 & \\
			\dfrac{1}{x} & x \arrow[r] & 1 \\[-12pt] \hline \\[-12pt]
			1 \arrow[rd, "-"] & 1 & \\
			0 & x &
		\end{tikzcd}
	\]
	따라서 구하는 부정적분은
	\[
		\int \ln x\,dx =x\ln x-x+C.\qed
	\]
\end{sol}

\begin{remark}
	$\ln x$의 부정적분은 암기하면 편하다.
\end{remark}

\begin{exc}{로그함수가 $D$에 있는 경우 [2]}{4.4}
	$\displaystyle\int x\ln x\,dx$를 구하시오.
\end{exc}

\begin{sol}
	\[
		\begin{tikzcd}
			D & I \\[-12pt]
			\ln x \arrow[rd, "+"] & x & \\
			\dfrac{1}{x} & \dfrac{1}{2}x^2 \arrow[r] & \dfrac{1}{2}x \\[-12pt] \hline \\[-12pt]
			\dfrac{1}{2} \arrow[rd, "-"] & x & \\
			0 & \dfrac{1}{2}x^2 &
		\end{tikzcd}\qquad\int x\ln x\,dx=\frac{1}{2}x^2\ln x-\frac{1}{4}x^2+C\qed
	\]
\end{sol}

\newpage
\begin{exc}{로그함수가 $D$에 있는 경우 [3]}{4.5}
	$\displaystyle\int x(\ln x)^2\,dx$를 구하시오.
\end{exc}

\begin{sol}
	\[
		\begin{tikzcd}
			D & I \\[-12pt]
			(\ln x)^2 \arrow[rd, "+"] & x & \\
			\dfrac{2\ln x}{x} & \dfrac{1}{2}x^2 \arrow[r] & x\ln x \\[-12pt] \hline \\[-12pt]
			\ln x \arrow[rd, "-"] & x & \\
			\dfrac{1}{x} & \dfrac{1}{2}x^2 \arrow[r] & \dfrac{1}{2}x \\[-12pt] \hline \\[-12pt]
			\dfrac{1}{x} \arrow[rd, "+"] & x & \\
			0 & \dfrac{1}{2}x^2 &
		\end{tikzcd}\qquad\int x(\ln x)^2\,dx=\frac{1}{2}x^2(\ln x)^2-\frac{1}{2}x^2\ln x+\frac{1}{4}x^2+C\qed
	\]
\end{sol}

\begin{exc}{삼각함수가 $D$에 있는 경우}{4.6}
	$\displaystyle\int e^x\sin x\,dx$를 구하시오.
\end{exc}

\begin{sol}
	\[
		\begin{tikzcd}
			D & I \\[-12pt]
			\sin x \arrow[rd, "+"] & e^x \\
			\cos x \arrow[rd, "-"] & e^x \\
			-\sin x \arrow[r, "+"] & e^x
		\end{tikzcd}
	\]
	도표의 마지막 행을 보면 $e^x\sin x$가 다시 등장하므로 도표 채우기를 멈추고 식을 쓴다.
	\[
		\int e^x\sin x\,dx =e^x\sin x-e^x \cos x-\int e^x\sin x\,dx
	\]
	양변을 정리하고 적분상수를 붙이면 부정적분을 얻는다.
	\[
		\int e^x\sin x\,dx =\frac{1}{2}e^x(\sin x-\cos x)+C\qed
	\]
\end{sol}

\newpage
\section{기본 넓이}
이 단락부터는 알아두면 문제 해결의 속도를 높일 수 있는 잡기술을 소개한다.

\begin{thm}{기본 넓이}{5.1}
	\begin{align}
		&\int_0^1e^x\,dx=e-1\\[6pt]
		&\int_1^e\ln x\,dx=1\\[6pt]
		&\int_0^{\tfrac{\pi}{2}}\sin x\,dx=\int_0^{\tfrac{\pi}{2}}\cos x\,dx=1\\[6pt]
		&\int_0^{\tfrac{\pi}{2b}}a\sin bx\,dx=\int_0^{\tfrac{\pi}{2b}}a\cos bx\,dx=\frac{a}{b}\quad(a>0,~b>0)\\[6pt]
		&\int_0^1x^n\,dx=\frac{1}{n+1}\quad (n\ne-1)
	\end{align}
\end{thm}

(1)--(3)는 실수 $1$과 관련이 있으며, (4)는 (3)을 일반화한 삼각함수에서의 한 칸($1/4$주기)의 넓이이다. 그래프를 그려보면서 숙지한자.

\begin{exc}{2014학년도 사관학교 B형 21번}{5.2}
	함수 $f(x)$가 다음 조건을 만족시킨다.
	\begin{hint}
		\begin{itemize}
			\item[(가)] $0\leq x<1$일 때, $f(x)=e^x-1$이다.
			\item[(나)] 모든 실수 $x$에 대하여 $f(x+1)=-f(x)+e-1$이다.
		\end{itemize}
	\end{hint}
	$\displaystyle\int_0^3f(x)\,dx$의 값은? [4점]
\end{exc}

\begin{sol}
	(나)를 이용하여 적분 범위를 조정하자.
	\begin{align*}
		&\int_0^1f(x)\,dx=\int_0^1(e^x-1)\,dx=e-2\\[6pt]
		&\int_1^2f(x)\,dx=\int_0^1f(x+1)\,dx=-\int_0^1f(x)\,dx+e-1=1\\[6pt]
		&\int_2^3f(x)\,dx=\int_1^2f(x+1)\,dx=-\int_1^2f(x)\,dx+e-1=e\\[6pt]
		&\int_0^3f(x)\,dx=(e-2)+1+e=2e-1\qed
	\end{align*}
\end{sol}

\begin{remark}
	$x$축과 평행한 방향으로의 평행이동은 적분 범위를 평행이동 시킨다.
\end{remark}

\newpage
\section{유리함수의 적분 -- Heaviside Cover-up Method}
유리함수를 적분할 때에는 기본적으로 차수를 낮추는 것에 집중한다. 이때 다음의 루틴을 적용한다.
\begin{enumerate}
	\item 간단한 부분분수 분해 공식을 적용할 수 있는 경우: 부분분수 분해를 적용한다.
		\[
			\frac{1}{AB}=\frac{1}{B-A}\biggl(\frac{1}{A}-\frac{1}{B}\biggr)
		\]
	\item 분자의 차수가 분모의 차수보다 큰 경우: 다항식의 나눗셈을 이용한다. 특히 분모가 일차식인 경우 조립제법, 나머지정리를 이용할 수 있다.
		\[
			f(x)=g(x)Q(x)+R(x)~\longrightarrow~\frac{f(x)}{g(x)}=Q(x)+\frac{R(x)}{g(x)}
		\]
	\item 분자의 차수가 분모의 차수보다 작은 경우:
		\begin{enumerate}[(a)]
			\item 치환적분법을 적용할 수 있는지를 판단한다.
			\item 분모가 일차식의 곱으로 인수분해되는 경우에는 ‘Heaviside Cover-up Method’을 이용한다.
			\item 이외에는 적절히 부분분수를 설정하고 부분분수의 계수를 미지수로 하여 계수비교법 또는 수치대입법을 통해 부분분수 분해를 시행한다.
		\end{enumerate}
\end{enumerate}
(iii)의 (b)에서 Heaviside Cover-up Method는 (i)의 부분분수 분해 공식을 일반화한 것이다.

\begin{thm}{Heaviside Cover-up Method}{6.1}
	다음 형태의 부분분수 분해를 생각하자.
	\[
		\frac{p(x)}{(x-\alpha_1)(x-\alpha_2)\cdots(x-\alpha_n)}=\sum_{i=1}^n\frac{c_i}{x-\alpha_i}
	\]
	이때 계수 $c_i$는 다음과 같다.
	\[
		c_i=\frac{p(\alpha_i)}{\prod\limits_{j\ne i}(\alpha_i-\alpha_j)}=\frac{p(\alpha_i)}{(\alpha_i-\alpha_1)\cdots(\alpha_i-\alpha_{i-1})(\alpha_i-\alpha_{i+1})\cdots(\alpha_i-\alpha_n)}
	\]
\end{thm}

위의 식에 따르면 부분분수 분해에서 계수 $c_i$는 원래 유리식의 분모에서 $(x-\alpha_i)$만 ‘지우고’ 남은 부분에 $\alpha_i$를 대입한 값이 된다.
\begin{align*}
	&\frac{p(x)}{(x-\alpha_1)\cdots(x-\alpha_{i-1})\xcancel{(x-\alpha_i)}(x-\alpha_{i+1})\cdots(x-\alpha_n)}\\[3pt]
	\longrightarrow~&c_i=\frac{p(\alpha_i)}{(\alpha_i-\alpha_1)\cdots(\alpha_i-\alpha_{i-1})(\alpha_i-\alpha_{i+1})\cdots(\alpha_i-\alpha_n)}
\end{align*}
특히 $f(x)=(x-\alpha_1)(x-\alpha_2)\cdots(x-\alpha_n)$이라 하면
\[
	f^{\prime}(\alpha_i)=(\alpha_i-\alpha_1)\cdots(\alpha_i-\alpha_{i-1})(\alpha_i-\alpha_{i+1})\cdots(\alpha_i-\alpha_n)
\]
이므로 위의 식은 다음과 같이 축약할 수 있다.
\[
	c_i=\frac{p(\alpha_i)}{f^{\prime}(\alpha_i)}~\longrightarrow~\frac{p(x)}{f(x)}=\sum_{i=1}^n\frac{p(\alpha_i)}{f^{\prime}(\alpha_i)(x-\alpha_i)}
\]

\begin{exc}{2024학년도 수능특강}{6.2}
	공차가 $2$인 등차수열 $\{a_n\}$에 대하여 함수 $f(x)$를 $f(x)=(x-a_1)(x-a_2)(x-a_3)$이라 하자. 함수 $g(x)$가
	\[
		g(x)=\sum_{k=1}^3\frac{f(x)}{f^{\prime}(a_k)\times(x-a_k)}
	\]
	일 때, $g(a_4)$의 값은?
\end{exc}

\begin{sol}
	정리 \ref{thm:6.1}에서 $p(x)=1$인 경우에 해당하므로 $g(x)=f(x)\times\dfrac{1}{f(x)}=1$이다.\qed
\end{sol}

\begin{exc}{2019학년도 수능특강}{6.3}
	$\displaystyle\int_e^{e^2}\frac{1+\ln x}{x(\ln x)(2+\ln x)}\,dx$의 값은?
\end{exc}

\begin{sol}
	$\ln x=t$라 하면 $dx=\dfrac{1}{x}\,dt$이므로
	\begin{align*}
		\int_e^{e^2}\frac{1+\ln x}{x(\ln x)(2+\ln x)}\,dx&=\int_1^2\frac{1+t}{t(2+t)}\,dt=\int_1^2\biggl(\frac{1/2}{t}+\frac{1/2}{(t+2)}\biggr)\,dt\\[3pt]
		&=\biggl[\frac{1}{2}\ln\vert t\vert +\frac{1}{2}\ln\vert t+2\vert\biggr]_1^2=\frac{1}{2}\ln\frac{8}{3}.
	\end{align*}
	여기서 $\dfrac{1}{t}$의 계수 $\dfrac{1}{2}$은 $\alpha_1=0$에서 $\dfrac{1+0}{2+0}$으로 구하고, $\dfrac{1}{t+2}$의 계수 $\dfrac{1}{2}$은 $\alpha_2=-2$에서 $\dfrac{1+(-2)}{-2}$로 구한다.\qed
\end{sol}

\begin{remark}
	$t(2+t)=u$라 하면 $du=(2t+2)\,dt$이므로 $u$로 다시 치환하는 방법도 가능하나, 이것이 쉽게 보이지 않는 경우에는 Heaviside Cover-up Method 같은 방법이 도움이 될 수 있다.
\end{remark}

\section{곱/몫의 미분법 형태의 처리}
함수의 곱의 미분법, 몫의 미분법은 각각 다음과 같다.
\begin{align*}
	\{f(x)g(x)\}^{\prime}&=f^{\prime}(x)g(x)+f(x)g^{\prime}(x)\\
	\biggl\{\frac{f(x)}{g(x)}\biggr\}^{\prime}&=\frac{f^{\prime}(x)g(x)-f(x)g^{\prime}(x)}{\{g(x)\}^2}
\end{align*}
문제에서 각 식의 우변의 형태 그대로 조건이 등장하는 경우가 많다. 특히  $f^{\prime}(x)g(x)-f(x)g^{\prime}(x)$의 형태가 보일 경우, $\{g(x)\}^2$으로 적절히 나누어 몫의 미분법의 형태를 이끌어내면 편하다.

\newpage
\begin{exc}{2014년 7월 학력평가 B형 9번}{7.1}
	$x>0$에서 미분가능한 함수 $f(x)$가 다음 조건을 만족시킨다.
	\begin{hint}
		\begin{itemize}
			\item[(가)] $f\biggl(\dfrac{\pi}{2}\biggr)=1$
			\item[(나)] $f(x)+xf^{\prime}(x)=x\cos x$
		\end{itemize}
	\end{hint}
	$f(\pi)$의 값은? [3점]
\end{exc}

\begin{sol}
	$f(x)+xf^{\prime}(x)=\{xf(x)\}^{\prime}$이므로 $\{xf(x)\}^{\prime}=x\cos x$에서
	\[
		\begin{tikzcd}
			D & I \\[-12pt]
			x \arrow[rd, "+"] & \cos x \\
			1 \arrow[rd, "-"] & \sin x \\
			0 & -\cos x
		\end{tikzcd}\qquad xf(x)=x\sin x+\cos x+C
	\]
	이고 $f\biggl(\dfrac{\pi}{2}\biggr)=1$에서 $C=0$이다. 따라서 $f(\pi)=-\dfrac{1}{\pi}$이다.\qed
\end{sol}

\begin{exc}{2019년 7월 학력평가 가형 26번}{7.2}
	실수 전체의 집합에서 미분가능한 함수 $f(x)$가 다음 조건을 만족시킨다.
	\begin{hint}
		\begin{itemize}
			\item[(가)] $f(1)=0$
			\item[(나)] $0$이 아닌 모든 실수 $x$에 대하여 $\dfrac{xf^{\prime}(x)-f(x)}{x^2}=xe^x$이다.
		\end{itemize}
	\end{hint}
	$f(3)\times f(-3)$의 값을 구하시오.
\end{exc}

\begin{sol}
	$\dfrac{xf^{\prime}(x)-f(x)}{x^2}=\biggl\{\dfrac{f(x)}{x}\biggr\}^{\prime}$이므로 $\biggl\{\dfrac{f(x)}{x}\biggr\}^{\prime}=xe^x$에서
	\[
		\begin{tikzcd}
			D & I \\[-12pt]
			x \arrow[rd, "+"] & e^x \\
			1 \arrow[rd, "-"] & e^x \\
			0 & e^x
		\end{tikzcd}\qquad\frac{f(x)}{x}=(x-1)e^x+C
	\]
	이고 $f(1)=0$에서 $C=0$이다. 따라서 $f(3)\times f(-3)=6e^3\times12e^{-3}=72$이다.\qed
\end{sol}

$f^{\prime}(x)+f(x)$, $f^{\prime}(x)-f(x)$ 형태의 식이 등장하는 경우,
\begin{align*}
	\{e^xf(x)\}^{\prime}&=e^x\{f^{\prime}(x)+f(x)\}\\
	\{e^{-x}f(x)\}^{\prime}&=e^{-x}\{f^{\prime}(x)-f(x)\}
\end{align*}
를 이용하기 위해 $e^x$ 또는 $e^{-x}$를 적절히 곱한다.

\begin{exc}{2023학년도 수능특강}{7.3}
	$f(-1)=0$인 일차함수 $f(x)$에 대하여
	\[
		\int_0^1e^x\{f(x)+f^{\prime}(x)\}\,dx=4e-2
	\]
	일 때, $f(2)$의 값은?
\end{exc}

\begin{sol}
	$f(-1)=0$이므로 상수 $m\ne0$에 대하여 $f(x)=m(x+1)$라 하면
	\[
		\int_0^1e^x\{f(x)+f^{\prime}(x)\}\,dx=\biggl[e^xf(x)\biggr]_0^1=ef(1)-f(0)
	\]
	에서 $(2e-1)m=4e-2$이므로 $m=2$이다. 따라서 $f(2)=6$이다.\qed
\end{sol}

\section{역함수의 정적분}
일대일대응이면서 연속인 함수 $f(x)$는 그  역함수 $f^{-1}(x)$ 역시 연속이므로 $f^{-1}(x)$의 정적분을 생각할 수 있다. 여기서 추가로 $f(x)$가 미분가능하다고 가정하고 정적분 $\displaystyle\int_{f(a)}^{f(b)}f^{-1}(x)\,dx$를 고려하자. $f^{-1}(x)=t$로 치환하면 $x=f(t)$, $\dfrac{dx}{dt}=f^{\prime}(t)$이므로
\[
	\int_{f(a)}^{f(b)}f^{-1}(x)\,dxt=\int_a^btf^{\prime}(t)\,dt=\biggl[tf(t)\biggr]_a^b-\int_a^bf(t)\,dt=bf(b)-af(a)-\int_a^bf(t)\,dt
\]
가 성립한다. 특히 마지막 등호의 식 $bf(b)-af(a)-\displaystyle\int_a^bf(t)\,dt$는 다음 그래프로 시각화할 수 있다.
\begin{figure}[H]
	\centering
	\includegraphics[scale=0.35]{section 8}
\end{figure}

역함수의 정적분을 그래프를 이용하여 빠르게 구할 수 있는 경우에는 $bf(b)-af(a)-\displaystyle\int_a^bf(t)\,dt$를, 식으로 계산해야 할 경우에는 $\displaystyle\int_a^btf^{\prime}(t)\,dt$를 이용한다. 이 등식은 항상 성립하므로 문제에서 역함수와 $xf^{\prime}(x)$, 또는 역함수와 그래프가 동시에 등장하는 경우 등 상황에 맞게 적용하면 된다.  물론 직선 $y=x$에 대한 대칭성 등 미분/적분과 관계없이 성립하는 성질들은 기본적으로 숙지해야 한다.

\begin{exc}{2012년 7월 학력평가 나형 21번}{8.1}
	함수 $f(x)=x^3+x-1$의 역함수를 $g(x)$라 할 때, $\displaystyle\int_1^9g(x)\,dx$의 값은? [4점]
\end{exc}

\begin{sol}
	$f(1)=1$, $f(2)=9$이므로
	\[
		\int _1^9g(x)\,dx=\int_1^2xf^{\prime}(x)\,dx=\int_1^2(3x^3+x)\,dx=\biggl[\frac{3}{4}x^4+\dfrac{1}{2}x^2\biggr]_1^2=\frac{51}{4}.\qed
	\]
\end{sol}

\section{대칭성의 활용 -- King Property}
정적분의 계산에서 대칭성을 활용하여 계산을 줄일 수 있다. 일반적으로 적분의 범위가 $0$을 중심으로 대칭이면 $y$축 대칭과 원점 대칭을 이용한다. 여기서는 대칭성의 아이디어를 일반화한 King Property를 다룬다.

\begin{thm}{King Property [1]}{9.1}
	구간 $[a$, $b]$에서 연속인 함수 $f(x)$에 대하여 다음이 성립한다.
	\[
		\int_a^bf(x)\,dx=\int_a^bf(a+b-x)\,dx
	\]
\end{thm}

\begin{proof}
	$x=a+b-t$라 하면 $\dfrac{dx}{dt}=-1$이므로
	\[
		\int_a^bf(x)\,dx=-\int_b^af(a+b-t)\,dt=\int_a^bf(a+b-t)\,dt.\qedhere
	\]
\end{proof}

\begin{remark}
	두 함수 $f(x)$와 $f(a+b-x)$는 직선 $x=\dfrac{a+b}{2}$에 대하여 대칭이다.
\end{remark}

\begin{thm}{King Property [2]}{9.2}
	구간 $[a$, $b]$에서 연속이고 $f(a+b-x)+f(x)\ne0$인 함수 $f(x)$에 대하여 다음이 성립한다.
	\[
		\int_a^b\frac{f(x)}{f(a+b-x)+f(x)}\,dx=\frac{b-a}{2}
	\]
\end{thm}

\begin{proof}
	정리 \ref{thm:9.1}에 의해 다음이 성립힌다.
	\[
		\int_a^b\frac{f(x)}{f(a+b-x)+f(x)}\,dx=\int_a^b\frac{f(a+b-x)}{f(x)+f(a+b-x)}\,dx
	\]
	따라서
	\begin{align*}
		\int_a^b\frac{f(x)}{f(a+b-x)+f(x)}\,dx&=\frac{1}{2}\biggl(\int_a^b\frac{f(x)}{f(a+b-x)+f(x)}\,dx+\int_a^b\frac{f(a+b-x)}{f(a+b-x)+f(x)}\,dx\biggr)\\[3pt]
		&=\frac{1}{2}\int_a^b\frac{f(x)+f(a+b-x)}{f(x)+f(a+b-x)}\,dx=\frac{1}{2}\int_a^b\,dx=\frac{b-a}{2}.\qedhere
	\end{align*}
\end{proof}

\begin{exc}{2016년 3월 학력평가 가형 28번}{9.3}
	함수 $f(x)=\dfrac{e^{\cos x}}{1+e^{\cos x}}$에 대하여
	\[
		a=f(\pi-x)+f(x),\quad b=\int_0^\pi f(x)\,dx
	\]
	일 때, $a+\dfrac{100}{\pi}b$의 값을 구하시오.
\end{exc}

\begin{sol}
	정리 \ref{thm:9.2}의 아이디어를 차용하면
	\begin{align*}
		a&=f(\pi-x)+f(x)=\frac{e^{-\cos x}}{1+e^{-\cos x}}+\frac{e^{\cos x}}{1+e^{\cos x}}=\frac{1}{e^{\cos x}+1}+\frac{e^{\cos x}}{1+e^{\cos x}}=1\\[3pt]
		b&=\int_0^\pi f(x)\,dx=\frac{1}{2}\bigg(\int_0^\pi f(x)\,dx+\int_0^\pi f(\pi-x)\,dx\biggr)=\frac{1}{2}\int_0^\pi1\,dx=\frac{\pi}{2}
	\end{align*}
	따라서 $a+\dfrac{100}{\pi}b=51$이다. \qed
\end{sol}

\newpage
\section*{연습문제}
\begin{prob}{2019학년도 6월 모의평가 가형 11번}{1}
	$\displaystyle\int_1^{\sqrt2}x^3\sqrt{x^2-1}\,dx$의 값은? [3점]
\end{prob}

\begin{sol}[풀이 1]
	$x^2-1=t$라 하면 $\dfrac{dt}{dx}=2x$이므로
	\[
		\int_1^{\sqrt2}x^3\sqrt{x^2-1}\,dx=\int_0^1\frac{1}{2}(t+1)\sqrt{t}\,dt=\int_0^1\frac{1}{2}(t\sqrt{t}+\sqrt{t})\,dt=\frac{1}{2}\times\biggl(\frac{1}{\frac{3}{2}+1}+\frac{1}{\frac{1}{2}+1}\biggr)=\frac{8}{15}.\qed
	\]
\end{sol}

\begin{sol}[풀이 2]
	$\sqrt{x^2-1}=t$라 하면 $\dfrac{dt}{dx}=\dfrac{x}{\sqrt{x^2-1}}$이므로
	\[
		\int_1^{\sqrt2}x^3\sqrt{x^2-1}\,dx=\int_0^1(t^2+1)t^2\,dt=\int_0^1(t^4+t^2)\,dt=\frac{1}{5}+\frac{1}{3}=\frac{8}{15}.\qed
	\]
\end{sol}

\newpage
\begin{prob}{2022학년도 수능완성}{2}
	열린구간 $\biggl(-\dfrac{\pi}{2}, \dfrac{\pi}{2}\biggr)$에서 정의된 함수 $f(x)$에 대하여 $f(0)=0$이고
	\[
		f^{\prime}(x)=\frac{\cos x-\sin x}{\cos^3x}
	\]
	이다. $f\bigg(\dfrac{\pi}{4}\biggr)$의 값은? [3점]
\end{prob}

\begin{sol}
	\[
		f(x)=\int\frac{\cos x-\sin x}{\cos^3x}\,dx=\int(\sec^2x-\sec ^2x\tan x)\,dx=\tan x-\frac{1}{2}\tan^2x+C
	\]
	이고 $f(0)=0$에서 $C=0$이다. 따라서 $f\bigg(\dfrac{\pi}{4}\biggr)=\dfrac{1}{2}$이다.\qed
\end{sol}

\newpage
\begin{prob}{2022학년도 수능특강}{3}
	$\displaystyle\int_0^{\tfrac{\pi}{3}}\dfrac{1}{\cos^4x}\,dx=k$일 때, $k^2$의 값을 구하시오.	
\end{prob}

\begin{sol}
	\[
		\begin{aligned}
			k&=\int_0^{\tfrac{\pi}{3}}\frac{1}{\cos^4x}\,dx=\int_0^{\tfrac{\pi}{3}}\frac{\sin^2x+\cos^2x}{\cos^4x}\,dx=\int_0^{\tfrac{\pi}{3}}\biggl(\frac{\sin^2x}{\cos^4x}+\frac{1}{\cos^2x}\biggr)\,dx\\[3pt]
			&=\int_0^{\tfrac{\pi}{3}}(\sec^2x\tan^2x+\sec^2x)\,dx=\biggl[\frac{1}{3}\tan^3x+\tan x\biggr]_0^{\tfrac{\pi}{3}}=2\sqrt3
		\end{aligned}
	\]
	이므로 $k^2=12$이다.\qed
\end{sol}

\newpage
\begin{prob}{2009학년도 9월 모의평가 가형 28번}{4}
	좌표평면에서 곡선 $y=\dfrac{xe^{x^2}}{e^{x^2}+1}$과 직선 $y=\dfrac{2}{3}x$로 둘러싸인 두 부분의 넓이의 합은? [3점]
\end{prob}

\begin{sol}
	$\dfrac{xe^{x^2}}{e^{x^2}+1}=\dfrac{2}{3}x$에서 $x\biggl(\dfrac{e^{x^2}}{e^{x^2}+1}-\dfrac{2}{3}\biggr)=0$이므로 $x=0$, $\pm\sqrt{\ln2}$이다. 두 도형은 서로 원점에 대하여 대칭이므로 구하는 넓이의 합은
	\[
		2\left\vert\int_0^{\sqrt{\ln2}}\biggl(\frac{xe^{x^2}}{e^{x^2}+1}-\frac{2}{3}x\biggr)\,dx\right\vert=\left\vert\biggl[\ln(e^{x^2}+1)-\frac{2}{3}x^2\biggr]_0^{\sqrt{\ln2}}\right\vert=\frac{5}{3}\ln2-\ln3.\qed
	\]
\end{sol}

\newpage
\begin{prob}{2022학년도 수능완성}{5}
	좌표평면 위의 점 $\rm P$의 시각 $t$ $(t>0)$에서의 위치 $(x$, $y)$가
	\[
		x=t(1-\ln t)^2+t+\frac{1}{t},\quad y=(\ln t)^2
	\]
	일 때, 시각 $t=1$에서 $t=e$까지 점 $\rm P$가 움직인 거리는? [4점]
\end{prob}

\begin{sol}
	각 식을 $t$에 대하여 미분하면
	\[
		\frac{dx}{dt}=(1-\ln t)^2-2(1-\ln t)+1-\frac{1}{t^2}=(\ln t)^2-\frac{1}{t^2},\quad\frac{dy}{dt}=\frac{2\ln t}{t}
	\]
	이므로
	\[
		\sqrt{\biggl(\frac{dx}{dt}\biggr)^2+\biggl(\frac{dy}{dt}\biggr)^2}=\sqrt{(\ln t)^4+\frac{1}{t^4}+\frac{2(\ln t)^2}{t^2}}=\sqrt{\biggl\{(\ln t)^2+\frac{1}{t^2}\biggr\}^2}=(\ln t)^2+\frac{1}{t^2}
	\]
	이고,
	\[
		\begin{tikzcd}
			D & I \\[-12pt]
			(\ln t)^2 \arrow[rd, "+"] & 1 & \\
			\dfrac{2\ln t}{t} & t \arrow[r] & 2\ln t \\[-12pt] \hline \\[-12pt]
			\ln t \arrow[rd, "-"] & 2 & \\
			\dfrac{1}{t} & 2t \arrow[r] & 2 \\[-12pt] \hline \\[-12pt]
			2 \arrow[rd, "+"] & 1 & \\
			0 & t &
		\end{tikzcd}\quad \begin{aligned}
			&\int_1^e(\ln t)^2\,dt=\biggl[t(\ln t)^2-2t\ln t+2t\biggr]_1^e=e-2\\[12pt]
			&\int_1^e\frac{1}{t^2}\,dt=\biggl[-\frac{1}{t}\biggr]_1^e=1-\frac{1}{e}
		\end{aligned}
	\]
	이므로 움직인 거리는 $(e-2)+\biggl(1-\dfrac{1}{e}\biggr)=e-\dfrac{1}{e}-1$이다.\qed
\end{sol}

\newpage
\begin{prob}{2019학년도 수능특강}{6}
	실수 전체의 집합에서 미분가능한 함수 $f(x)$가 다음 조건을 만족시킨다.
	\begin{hint}
		\begin{itemize}
			\item[(가)] $\lim\limits_{x\,\to\,0}\dfrac{f(x)}{x}=3$
			\item[(나)] $f^{\prime}(x)=(x+1)\sqrt{x^2+2x+a}$ (단, $a$는 상수이다.)
		\end{itemize}
	\end{hint}
	$a+f(\sqrt a)$의 값은?
\end{prob}

\begin{sol}
	(가)에서 $f(0)=0$, $f^{\prime}(0)=3$이다. (나)에서 $f^{\prime}(0)=\sqrt a$이므로 $a=9$이다.
	\[
		f(x)=\int(x+1)\sqrt{x^2+2x+9}\,dx=\frac{1}{2}\times\frac{2}{3}(x^2+2x+9)^{\tfrac{3}{2}}+C=\frac{1}{3}(x^2+2x+9)^{\tfrac{3}{2}}+C
	\]
	에서 $f(0)=9+C$이므로 $C=-9$이다. 따라서 $a+f(\sqrt a)=9+(16\sqrt6-9)=16\sqrt6$이다.\qed
\end{sol}

\newpage
\begin{prob}{2015년 4월 학력평가 B형 17번}{7}
	자연수 $n$에 대하여 함수 $f(n)=\displaystyle\int_1^nx^3e^{x^2}dx$라 할 때, $\dfrac{f(5)}{f(3)}$의 값은? [4점]
\end{prob}

\begin{sol}
	\[
		\begin{tikzcd}
			D & I & \\[-12pt]
			x^2 \arrow[rd, "+"] & xe^{x^2} & \\
			2x & \dfrac{1}{2}e^{x^2} \arrow[r] & xe^{x^2} \\[-12pt] \hline \\[-12pt]
			1 \arrow[rd, "-"] & xe^{x^2} & \\
			0 & \dfrac{1}{2}e^{x^2} &
		\end{tikzcd}\quad\begin{aligned}
			&f(n)=\int_1^nx^3e^{x^2}\,dx=\biggl[\frac{1}{2}x^2e^{x^2}-\frac{1}{2}e^{x^2}\biggr]_1^n=\frac{1}{2}(n^2-1)e^{n^2}\\[12pt]
			&\frac{f(5)}{f(3)}=\frac{24e^{25}}{8e^{9}}=3e^{16}\qed
		\end{aligned}
	\]
\end{sol}

\newpage
\begin{prob}{2023학년도 수능특강}{8}
	수열 $\{a_n\}$은 모든 자연수 $n$에 대하여
	\[
		\frac{a_n}{n+1}=\int_0^p(\tan^nx+\tan^{n+2}x)\,dx
	\]
	를 만족시킨다. $\displaystyle\sum_{n=1}^\infty a_n=\frac{1}{12}$일 때, $\tan p$의 값은? (단, $p$는 $0<p<\dfrac{\pi}{4}$인 상수이다.)
\end{prob}

\begin{sol}
	$a_n$을 먼저 정리하자.
	\begin{align*}
		a_n&=(n+1)\int_0^p(\tan^nx+\tan^{n+2}x)\,dx=(n+1)\int_0^p(1+\tan^2x)\tan^nx\,dx\\[3pt]
		&=(n+1)\int_0^p\sec^2x\tan^nx\,dx=(n+1)\biggl[\frac{\tan^{n+1}x}{n+1}\biggr]_0^p=\tan^{n+1}p
	\end{align*}
	$0<p<\dfrac{\pi}{4}$이므로 $0<\tan p<1$이고 급수 $\displaystyle\sum_{n=1}^\infty a_n$은 수렴한다.
	\[
		\sum_{n=1}^\infty a_n=\frac{\tan^2p}{1-\tan p}=\frac{1}{12}
	\]
	이므로 $12\tan^2p+\tan p-1=0$에서 $(3\tan p+1)(4\tan p-1)=0$, $\tan p=\dfrac{1}{4}$이다.\qed
\end{sol}

\newpage
\begin{prob}{2019년 4월 학력평가 가형 16번}{9}
	두 곡선 $y=(\sin x)\ln x$, $y=\dfrac{\cos x}{x}$와 두 직선 $x=\dfrac{\pi}{2}$, $x=\pi$로 둘러싸인 부분의 넓이는? [4점]
	\begin{figure}[H]
		\centering
		\includegraphics[scale=0.45]{2019년 4월 학력평가 가형 16번}
	\end{figure}
\end{prob}

\begin{sol}
	\[
		\begin{aligned}
			\int_{\tfrac{\pi}{2}}^\pi\biggl\{(\sin x)\ln x-\frac{\cos x}{x}\biggr\}\,dx&=\int_{\tfrac{\pi}{2}}^\pi\{(-\cos x)^{\prime}\ln x+(-\cos x)(\ln x)^{\prime}\}\,dx\\[3pt]
			&=\biggl[(-\cos x)\ln x\biggr]_{\tfrac{\pi}{2}}^\pi=\ln\pi\qed
		\end{aligned}
	\]
\end{sol}

\newpage
\begin{prob}{2020학년도 수능특강}{10}
	양의 실수 전체의 집합에서 정의된 연속함수 $f(x)$가 다음 조건을 만족시킨다.
	\begin{hint}
		\begin{itemize}
			\item[(가)] 모든 양의 실수 $x$에 대하여 $f(x)=f\biggl(\dfrac{1}{x}\biggr)$이다.
			\item[(나)] $\displaystyle\int_{\tfrac{1}{2}}^2f(x)\,dx=15+16\ln2$
		\end{itemize}
	\end{hint}
	$\displaystyle\int_{\tfrac{1}{2}}^2\frac{f(x)}{x^2}\,dx$의 값은?
\end{prob}

\begin{sol}
	함수 $f(x)$의 한 부정적분을 $F(x)$라 하면 (나)에서 $F(2)-F\biggl(\dfrac{1}{2}\biggr)=15+16\ln2$이다.
	\[
		\int_{\tfrac{1}{2}}^2\frac{f(x)}{x^2}\,dx=\int_{\tfrac{1}{2}}^2\frac{1}{x^2}f\biggl(\frac{1}{x}\biggr)\,dx=\biggl[-F\biggl(\frac{1}{x}\biggr)\biggr]_{\tfrac{1}{2}}^2=F(2)-F\biggl(\frac{1}{2}\biggr)=15+16\ln2\qed
	\]
\end{sol}

\newpage
\begin{prob}{2004학년도 사관학교 가형 9번}{11}
	$f(x)$의 부정적분 $F(x)$와 $g(x)$의 부정적분 $G(x)$가 다음 관계식을 만족한다.
	\[
		f(x)=\frac{G(x)+g(x)}{2},\quad g(x)=\frac{F(x)+f(x)}{2}
	\]
	$f(0)=0$, $g(0)=2$일 때, $f(1)+g(1)$의 값은? [3점]
\end{prob}

\begin{sol}
	\[
		\begin{aligned}
			&f(x)+g(x)=\frac{F(x)+G(x)+f(x)+g(x)}{2}\\[3pt]
			\longrightarrow~&f(x)+g(x)=F(x)+G(x)\\[3pt]
			\longrightarrow~&\{F(x)+G(x)\}^{\prime}-\{F(x)+G(x)\}=0\\[3pt]
			\longrightarrow~&e^{-x}[\{F(x)+G(x)\}^{\prime}-\{F(x)+G(x)\}]=0\\[3pt]
			\longrightarrow~&[e^{-x}\{F(x)+G(x)\}]^{\prime}=0\\[3pt]
			\longrightarrow~&F(x)+G(x)=Ce^x\\[3pt]
			\longrightarrow~&f(x)+g(x)=Ce^x
		\end{aligned}
	\]
	$f(0)+g(0)=2$이므로 $C=2$이다. 따라서 $f(1)+g(1)=2e$이다.\qed
\end{sol}

\newpage
\begin{prob}{2010학년도 9월 모의평가 가형 28번}{12}
	함수 $f(x)=\displaystyle\int_0^x\dfrac{1}{1+x^6}\,dt$에 대하여 상수 $a$가 $f(a)=\dfrac{1}{2}$을 만족시킬 때,
	\[
		\int_0^a\frac{e^{f(x)}}{1+x^6}\,dx
	\]
	의 값은? [3점]
\end{prob}

\begin{sol}
	주어진 조건으로부터 $f^{\prime}(x)=\dfrac{1}{1+x^6}$, $f(0)=0$이다.
	\[
		\int_0^a\frac{e^{f(x)}}{1+x^6}\,dx=\int_0^af^{\prime}(x)e^{f(x)}\,dx=\biggl[e^{f(x)}\biggr]_0^a=e^{f(a)}-e^{f(0)}=\sqrt e-1\qed
	\]
\end{sol}

\newpage
\begin{prob}{2022학년도 수능특강}{13}
	정의역이 $\{x\,\vert\,x>0\}$인 미분가능한 함수 $f(x)$가 모든 양의 실수 $x$에 대하여
	\[
		xf^{\prime}(x)+f(x)=4x^3\ln x
	\]
	를 만족시킨다. $f(1)=-\dfrac{1}{4}$일 때, $f(e)$의 값은?
\end{prob}

\begin{sol}
	\[
		\begin{tikzcd}
			D & I & \\[-12pt]
			\ln x \arrow[rd, "+"] & 4x^3 & \\
			\dfrac{1}{x} & x^4 \arrow[r] & x^3 \\[-12pt] \hline \\[-12pt]
			1 \arrow[rd, "-"] & x^3 & \\
			0 & \dfrac{1}{4}x^4 &
		\end{tikzcd}\qquad\begin{aligned}
			&xf^{\prime}(x)+f(x)=4x^3\ln x\\[3pt]
			\longrightarrow~&\{xf(x)\}^{\prime}=4x^3\ln x\\[3pt]
			\longrightarrow~&xf(x)=x^4\ln x-\dfrac{1}{4}x^4+C
		\end{aligned}
	\]
	$f(1)=-\dfrac{1}{4}$에서 $C=0$이다. 따라서 $f(e)=e^3-\dfrac{1}{4}e^3=\dfrac{3}{4}e^3$이다.\qed
\end{sol}

\newpage
\begin{prob}{2014학년도 6월 모의평가 B형 27번}{14}
	함수 $f(x)=\dfrac{1}{1+x}$에 대하여
	\[
		F(x)=\int_0^xtf(x-t)\,dt\quad(x\geq0)
	\]
	일 때, $F^{\prime}(a)=\ln10$을 만족시키는 상수 $a$의 값을 구하시오. [4점]
\end{prob}

\begin{sol}[풀이 1]
	$x-t=u$라 하면 $\dfrac{du}{dt}=-1$이므로
	\begin{align*}
		&F(x)=\int_0^xtf(x-t)\,dt=-\int_x^0(x-u)f(u)\,du=\int_0^x(x-u)f(u)\,du=x\int_0^xf(u)\,du-\int_0^xuf(u)\,du\\[3pt]
		&F^{\prime}(x)=\int_0^xf(u)\,du+xf(x)-xf(x)=\int_0^xf(u)\,du=\int_0^x\frac{1}{1+u}\,du=\biggl[\ln(1+u)\biggr]_0^x=\ln(1+x)
	\end{align*}
	따라서 $F^{\prime}(a)=\ln(1+a)=\ln10$으로부터 $a=9$이다.\qed
\end{sol}

\begin{sol}[풀이 2]
	\[
		\begin{aligned}
			F(x)&=\int_0^xtf(x-t)\,dt=\int_0^x\frac{t}{1+x-t}\,dt=\int_0^x\biggl(\frac{1+x}{1+x-t}-1\biggr)\,dt\\[3pt]
			&=\biggl[-(1+x)\ln(1+x-t)-t\biggr]_{t=0}^{t=x}=(1+x)\ln(1+x)-x\\[3pt]
			F^{\prime}(x)&=\ln(1+x)+1-1=\ln(1+x)
		\end{aligned}
	\]
	(이하 생략)\qed
\end{sol}

\newpage
\begin{prob}{2017년 5월 전북교육청 가형 14번}{15}
	미분가능한 함수 $f(x)$가 모든 양의 실수 $x$에 대하여 $f^{\prime}(e^x)=\ln x$를 만족시킬 때, $\displaystyle\int_e^{e^2}\dfrac{f^{\prime}(x)}{x}\,dx$의 값은? [4점]
\end{prob}

\begin{sol}[풀이 1]
	$x=e^t$라 하면 $\dfrac{dx}{dt}=e^t$이다.
	\[
		\int_e^{e^2}\frac{f^{\prime}(x)}{x}\,dx=\int_1^2\frac{f^{\prime}(e^t)}{e^t}e^t\,dt=\int_1^2\ln t\,dt=\biggl[t\ln t-t\biggr]_1^2=2\ln 2-1\qed
	\]
\end{sol}

\begin{sol}[풀이 2]
	$t=e^x$라 하면 $f^{\prime}(t)=\ln(\ln t)$이다.
	\[
		\int_e^{e^2}\frac{f^{\prime}(x)}{x}\,dx=\int_1^2\frac{\ln(\ln x)}{x}\,dx=\biggl[(\ln x)\ln(\ln x)-\ln x\biggr]_e^{e^2}=2\ln 2-1\qed
	\]
\end{sol}

\newpage
\begin{prob}{2016년 10월 전북교육청 가형 17번}{16}
	함수 $f(x)$의 도함수가 $f^{\prime}(x)=\dfrac{1}{1+\sin x}$일 때, 함수 $g(x)=2\sin x\cos x$에 대하여 함수 $h(x)$를 $h(x)=\displaystyle\int f(x)g^{\prime}(x)\,dx$라 하자. $h(0)=0$일 때, $h\biggl(\dfrac{\pi}{2}\biggr)$의 값은? [4점]
\end{prob}

\begin{sol}
	부분적분을 생각하여 $f^\prime(x)g(x)$의 부정적분을 계산하고 $h(x)$를 구하자.
	\begin{align*}
		\int f^{\prime}(x)g(x)\,dx&=\int\frac{2\sin x\cos x}{1+\sin x}\,dx\\[3pt]
		&=\int(\sin x)^{\prime}\biggl(2-\frac{2}{1+\sin x}\biggr)dx\\[3pt]
		&=2\sin x-2\ln\vert1+\sin x\vert+C\\[3pt]
		h(x)&=\int f(x)g^{\prime}(x)\,dx=f(x)g(x)-\int f^{\prime}(x)g(x)\,dx\\[3pt]
		&=2f(x)\sin x\cos x-2\sin x+2\ln\vert1+\sin x\vert+C
	\end{align*}
	$h(0)=0$에서 $C=0$이므로 $h\biggl(\dfrac{\pi}{2}\biggr)=-2+2\ln2$이다.\qed
\end{sol}

\begin{remark}
	$h(x)$에서 $C$의 부호를 바꾸지 않은 것은 $C$가 상수이기 때문이다.
\end{remark}

\newpage
\begin{prob}{2024학년도 수능특강}{17}
	일차함수 $f(x)$가 다음 조건을 만족시킨다.
	\begin{hint}
		\begin{itemize}
			\item[(가)] $\displaystyle\int_0^{\tfrac{\pi}{2}}\{f(x)\sin x-f^{\prime}(x)\cos x\}\,dx=1$
			\item[(나)] $\displaystyle\int_0^{\tfrac{\pi}{2}}\{f(x)\cos x+f^{\prime}(x)\sin x\}\,dx=\dfrac{3}{2}\pi+1$
		\end{itemize}
	\end{hint}
	$\displaystyle\int_0^1e^xf(x)\,dx$의 값은?
\end{prob}

\begin{sol}
	(가)에서
	\[
		\begin{aligned}
			\int_0^{\tfrac{\pi}{2}}\{f(x)\sin x-f^{\prime}(x)\cos x\}\,dx&=\int_0^{\tfrac{\pi}{2}}\{f(x)(-\cos x)^{\prime}+f^{\prime}(x)(-\cos x)\}\,dx\\[3pt]
			&=\biggl[-f(x)\cos x\biggr]_0^{\tfrac{\pi}{2}}=f(0)
		\end{aligned}
	\]
	이므로 $f(0)=1$이다. (나)에서
	\[
		\begin{aligned}
			\int_0^{\tfrac{\pi}{2}}\{f(x)\cos x+f^{\prime}(x)\sin x\}\,dx&=\int_0^{\tfrac{\pi}{2}}\{f(x)(\sin x)^{\prime}+f^{\prime}(x)\sin x\}\,dx\\[3pt]
			&=\biggl[f(x)\sin x\biggr]_0^{\tfrac{\pi}{2}}=f\bigg(\frac{\pi}{2}\biggr)
		\end{aligned}
	\]
	이므로 $f\biggl(\dfrac{\pi}{2}\biggr)=3\times\dfrac{\pi}{2}+1$이다. 따라서 $f(x)=3x+1$이고,
	\[
		\begin{tikzcd}
			D & I \\[-12pt]
			3x+1 \arrow[rd, "+"] & e^x \\
			3 \arrow[rd, "-"] & e^x \\
			0 & e^x
		\end{tikzcd}\quad\begin{aligned}
			\int_0^1e^xf(x)\,dx&=\int_0^1e^x(3x+1)\,dx\\[3pt]
			&=\biggl[(3x-2)e^x\biggr]_0^1\\[3pt]
			&=e+2.\qed
		\end{aligned}
	\]
\end{sol}

\newpage
\begin{prob}{2018학년도 수능 가형 15번}{18}
	함수 $f(x)$가
	\[
		f(x)=\int_0^x\frac{1}{1+e^{-t}}\,dt
	\]
	일 때, $(f\circ f)(a)=\ln5$를 만족시키는 실수 $a$의 값은? [4점]
\end{prob}

\begin{sol}
	$f(x)$를 간단히 하면
	\[
		f(x)=\int_0^x\frac{1}{1+e^{-t}}\,dt=\int_0^x\frac{e^t}{e^t+1}\,dt=\biggl[\ln(e^t+1)\biggr]_0^x=\ln(e^x+1)-\ln 2.
	\]
	따라서
	\[
		\begin{aligned}
			(f\circ f)(a)=\ln5~&\longrightarrow~\ln(e^{f(a)}+1)-\ln2 =\ln 5\\
			&\longrightarrow~e^{f(a)}=9\\
			&\longrightarrow~f(a)=\ln 9\\
			&\longrightarrow~\ln(e^a+1)-\ln 2=\ln 9\\
			&\longrightarrow~a=\ln 17.\qed
		\end{aligned}
	\]
\end{sol}

\newpage
\begin{prob}{2017년 10월 학력평가 가형 16번}{19}
	연속함수 $f(x)$가 다음 조건을 만족시킨다.
	\begin{hint}
		\begin{itemize}
			\item[(가)] $x\ne 0$인 실수 $x$에 대하여 $\{f(x)\}^2f^{\prime}(x)=\dfrac{2x}{x^2+1}$이다.
			\item[(나)] $f(0)=0$
		\end{itemize}
	\end{hint}
	$\{f(1)\}^3$의 값은? [4점]
\end{prob}

\begin{sol}
	\[
		\{f(x)\}^2f^{\prime}(x)=\dfrac{2x}{x^2+1}~\Longrightarrow~\frac{1}{3}\{f(x)\}^3=\ln(x^2+1)+C
	\]
	$f(0)=0$에서 $C=0$이므로 $\{f(1)\}^3=3\ln 2$이다.\qed
\end{sol}

\newpage
\begin{prob}{2018년 3월 학력평가 가형 20번}{20}
	함수 $f(x)=\displaystyle\int_0^x\sin(\pi\cos t)\,dt$에 대하여 $\langle$보기$\rangle$에서 옳은 것만을 있는 대로 고른 것은? [4점]
	\begin{hint}[보기]
		\begin{itemize}
			\item[ㄱ.] $f^{\prime}(0)=0$
			\item[ㄴ.] 함수 $y=f(x)$의 그래프는 원점에 대하여 대칭이다.
			\item[ㄷ.] $f(\pi)=0$
		\end{itemize}
	\end{hint}
\end{prob}

\begin{sol}
	\begin{itemize}
		\item[ㄱ.] $f^{\prime}(x)=\sin(\pi\cos x)$에서 $f^{\prime}(0)=\sin\pi=0$. (참)
		\item[ㄴ.] $t=-u$라 하면 $\dfrac{dt}{du}=-1$이므로
			\[
				f(-x)=\int_0^{-x}\sin(\pi\cos t)\,dt=-\int_0^x\sin(\pi\cos(-u))\,du=-\int_0^x\sin(\pi\cos u)\,du=-f(x)
			\]
			에서 함수 $y=f(x)$의 그래프는 원점에 대하여 대칭이다. (참)
		\item[ㄷ.] 정리 \ref{thm:9.1}을 이용하면
			\begin{align*}
				f(\pi)&=\int_0^\pi\sin(\pi\cos t)\,dt=\int_0^\pi\sin(\pi\cos(\pi-t))\,dt\\[3pt]
				&=\int_0^\pi\sin(-\pi\cos t)\,dt=-\int_0^\pi\sin(\pi\cos t)\,dt=-f(\pi)
			\end{align*}
			이므로 $f(\pi)=0$이다. (참)\qed
	\end{itemize}
\end{sol}

\newpage
\begin{prob}{2024학년도 수능특강}{21}
	양의 실수 전체의 집합에서 미분가능한 함수 $f(x)$가 모든 양의 실수 $x$에 대하여
	\[
		xf^{\prime}(x)-f(x)=\frac{x^3}{\sqrt{3x^2+1}}
	\]
	을 만족시킨다. $f(1)=1$일 때, $\displaystyle\int_1^4f(x)\,dx=\frac{q}{p}$이다. $p+q$의 값을 구하시오. (단, $p$와 $q$는 서로소인 자연수이다.)
\end{prob}

\begin{sol}
	주어진 식의 양변을 $x^2$으로 나누자.
	\begin{align*}
		xf^{\prime}(x)-f(x)=\frac{x^3}{\sqrt{3x^2+1}}~&\longrightarrow~\frac{xf^{\prime}(x)-f(x)}{x^2}=\frac{x}{\sqrt{3x^2+1}}\\[3pt]
		&\longrightarrow~\biggl(\frac{f(x)}{x}\biggr)^{\prime}=\frac{x}{\sqrt{3x^2+1}}\\[3pt]
		&\longrightarrow~\frac{f(x)}{x}=\frac{1}{3}\sqrt{3x^2+1}+C
	\end{align*}
	$f(1)=1$에서 $C=\dfrac{1}{3}$이다. 따라서
	\[
		\int_1^4f(x)\,dx=\int_1^4\biggl(\frac{1}{3}x\sqrt{3x^2+1}+\frac{1}{3}x\biggr)dx=\biggl[\frac{1}{27}(3x^2+1)^{\frac{3}{2}}+\frac{1}{6}x^2\biggr]_1^4=\frac{805}{54}
	\]
	에서 $p=54$, $q=805$이므로 $p+q=859$이다.\qed
\end{sol}

\newpage
\begin{prob}{2020학년도 9월 모의평가 가형 17번}{22}
	두 함수 $f(x)$, $g(x)$는 실수 전체의 집합에서 도함수가 연속이고 다음 조건을 만족시킨다.
	\begin{hint}
		\begin{itemize}
			\item[(가)] 모든 실수 $x$에 대하여 $f(x)g(x)=x^4-1$이다.
			\item[(나)] $\displaystyle\int_{-1}^1\{f(x)\}^2g^{\prime}(x)\,dx=120$
		\end{itemize}
	\end{hint}
	$\displaystyle\int_{-1}^1x^3f(x)$의 값은? [4점]
\end{prob}

\begin{sol}
	(나)에서
	\begin{align*}
		\int_{-1}^1\{f(x)\}^2g^{\prime}(x)\,dx&=\biggl[\{f(x)\}^2g(x)\biggr]_{-1}^1-\int_{-1}^12f(x)f^{\prime}(x)g(x)\,dx\\[3pt]
		&=\biggl[(x^4-1)f(x)\biggr]_{-1}^1-2\int_{-1}^1(x^4-1)f^{\prime}(x)\,dx
	\end{align*}
	이므로 $\displaystyle\int_{-1}^1(x^4-1)f^{\prime}(x)\,dx=-60$이다. 또한
	\[
		\int_{-1}^1(x^4-1)f^{\prime}(x)\,dx=\biggl[(x^4-1)f(x)\biggr]_{-1}^1-\int_{-1}^14x^3f(x)\,dx
	\]
	이므로 $\displaystyle\int_{-1}^1x^3f(x)=15$이다.\qed
\end{sol}

\newpage
\begin{prob}{2019학년도 수능 가형 16번}{23}
	$x>0$에서 정의된 연속함수 $f(x)$가 모든 양수 $x$에 대하여
	\[
		2f(x)+\frac{1}{x^2}f\biggl(\frac{1}{x}\biggr)=\frac{1}{x}+\frac{1}{x^2}
	\]
	을 만족시킬 때, $\displaystyle\int_{\tfrac{1}{2}}^2f(x)\,dx$의 값은? [4점]
\end{prob}

\begin{sol}[풀이 1]
	$\dfrac{1}{x}=t$라 하면 $\dfrac{dt}{dx}=-\dfrac{1}{x^2}$이므로
	\[
		\int_{\tfrac{1}{2}}^2\frac{1}{x^2}f\biggl(\frac{1}{x}\biggr)\,dx=-\int_2^{\tfrac{1}{2}}f(t)\,dt=\int_{\tfrac{1}{2}}^2f(t)\,dt.
	\]
	따라서 주어진 등식의 양변을 정적분하면
	\[
		2\int_{\tfrac{1}{2}}^2f(x)\,dx+\int_{\tfrac{1}{2}}^2f(x)\,dx=\int_{\tfrac{1}{2}}^2\biggl(\frac{1}{x}+\frac{1}{x^2}\biggr)dx=\biggl[\ln\vert x\vert-\frac{1}{x}\biggr]_{\tfrac{1}{2}}^2=2\ln 2+\frac{3}{2}
	\]
	이므로 $\displaystyle\int_{\tfrac{1}{2}}^2f(x)\,dx=\dfrac{2\ln 2}{3}+\frac{1}{2}$이다.\qed
\end{sol}	

\begin{sol}[풀이 2]
	주어진 등식의 양변에 $\dfrac{1}{x}$을 대입하면
	\[
		x^2f(x)+2f\bigg(\frac{1}{x}\biggr)=x+x^2.
	\]
	따라서 두 식을 적절히 연립하면 $f(x)=\dfrac{1}{3x}+\dfrac{2}{3x^2}-\dfrac{1}{3}$이므로
	\[
		\int_{\tfrac{1}{2}}^2f(x)\,dx=\int_{\tfrac{1}{2}}^2\frac{1}{3}\biggl(\frac{1}{x}+\frac{2}{x^2}-1\biggr)dx=\frac{1}{3}\biggl[\ln\vert x\vert-\frac{2}{x}-x\biggr]_{\tfrac{1}{2}}^2=\frac{2\ln 2}{3}+\frac{1}{2}.\qed
	\]
\end{sol}

\newpage
\begin{prob}{2012학년도 6월 모의평가 가형 19번}{24}
	정의역이 $\{x\,\vert\,x>-1\}$인 함수 $f(x)$에 대하여 $f^{\prime}(x)=\dfrac{1}{(1+x^3)^2}$이고, 함수 $g(x)=x^2$일 때,
	\[
		\int_0^1f(x)g^{\prime}(x)\,dx=\frac{1}{6}
	\]
	이다. $f(1)$의 값은? [4점]
\end{prob}

\begin{sol}
	\[
		\int_0^1f(x)g^{\prime}(x)\,dx=\biggl[f(x)g(x)\biggr]_0^1-\int_0^1f^{\prime}(x)g(x)\,dx=f(1)-\int_0^1\frac{x^2}{(1+x^3)^2}\,dx
	\]
	이고
	\[
		\int_0^1\frac{x^2}{(1+x^3)^2}\,dx=\biggl[-\frac{1}{3(1+x^3)}\biggr]_0^1=\frac{1}{6}
	\]
	이므로
	\[
		f(1)=\int_0^1f(x)g^{\prime}(x)\,dx+\int_0^1\frac{x^2}{(1+x^3)^2}\,dx=\frac{1}{3}.\qed
	\]
\end{sol}

\newpage
\begin{prob}{2020년 9월 경북교육청 가형 19번}{25}
	두 함수 $f(x)=xe^{x^2}$, $g(x)=\sin\sqrt x$에 대하여 $\displaystyle\int_{\tfrac{\pi^2}{4}}^{\pi^2}(f^{\prime}\circ g)(x)g(x)g^{\prime}(x)\,dx$의 값은? [4점]
\end{prob}

\begin{sol}
	곱의 순서를 적절히 바꾸어 부분적분을 적용하자.
	\begin{align*}
		\int_{\tfrac{\pi^2}{4}}^{\pi^2}(f^{\prime}\circ g)(x)g(x)g^{\prime}(x)\,dx&=\int_{\tfrac{\pi^2}{4}}^{\pi^2}g(x)\times(f^{\prime}\circ g)(x)g^{\prime}(x)\,dx\\[3pt]
		&=\biggl[g(x)f(g(x))\biggr]_{\tfrac{\pi^2}{4}}^{\pi^2}-\int_{\tfrac{\pi^2}{4}}^{\pi^2}g^{\prime}(x)f(g(x))\,dx\\[3pt]
		&=-e-\int_{\tfrac{\pi^2}{4}}^{\pi^2}g^{\prime}(x)f(g(x))\,dx\\[3pt]
		&=-e-\int_{\tfrac{\pi^2}{4}}^{\pi^2}g^{\prime}(x)g(x)e^{\{g(x)\}^2}\,dx\\[3pt]
		&=-e-\biggl[\frac{1}{2}e^{\{g(x)\}^2}\biggr]_{\tfrac{\pi^2}{4}}^{\pi^2}\\[3pt]
		&=-\frac{e+1}{2}\qed
	\end{align*}
\end{sol}

\newpage
\begin{prob}{2006년 10월 학력평가 가형 28번}{26}
	$a_n=\displaystyle\int_0^{\tfrac{\pi}{4}}\tan^nx\,dx$ $(n=1,2,3,\cdots)$으로 정의할 때, 옳은 내용을 $\langle$보기$\rangle$에서 모두 고른 것은? [4점]
	\begin{hint}[보기]
		\begin{itemize}
			\item[ㄱ.] $a_1+a_3=\dfrac{1}{2}$
			\item[ㄴ.] $a_1+a_2+a_3+a_4=\dfrac{1}{2}+\dfrac{1}{3}$
			\item[ㄷ.] $\displaystyle\sum_{k=1}^{100}a_k=\dfrac{1}{2}+\dfrac{1}{3}+\dfrac{1}{4}+\cdots+\dfrac{1}{51}$
		\end{itemize}
	\end{hint}
\end{prob}

\begin{sol}
	$\langle$보기$\rangle$에서 $a_n+a_{n+2}$가 반복적으로 등장하므로 이를 먼저 생각하자.
	\begin{align*}
		a_n+a_{n+2}&=\int_0^{\tfrac{\pi}{4}}(\tan^nx+\tan^{n+2}x)\,dx=\int_0^{\tfrac{\pi}{4}}(1+\tan^2x)\tan^nx\,dx\\[3pt]
		&=\int_0^{\tfrac{\pi}{4}}\sec^2x\tan^nx\,dx=\biggl[\frac{\tan^{n+1}x}{n+1}\biggr]_0^{\tfrac{\pi}{4}}\\[3pt]
	&=\frac{1}{n+1}
	\end{align*}
	\begin{itemize}
		\item[ㄱ.] $a_1+a_3=\dfrac{1}{2}$ (참)
		\item[ㄴ.] $a_1+a_2+a_3+a_4=(a_1+a_3)+(a_2+a_4)=\dfrac{1}{2}+\dfrac{1}{3}$ (참)
		\item[ㄷ.] 자연수 $k$에 대하여 $a_{4k-3}+a_{4k-2}+a_{4k-1}+a_{4k}=\dfrac{1}{4k-2}+\dfrac{1}{4k-1}$이다. 따라서
			\begin{align*}
				\sum_{k=1}^{100}a_k&=\sum_{k=1}^{25}(a_{4k-3}+a_{4k-2}+a_{4k-1}+a_{4k})\\[3pt]
				&=\sum_{k=1}^{25}\biggl(\frac{1}{4k-2}+\frac{1}{4k-1}\biggr)\\[3pt]
				&=\frac{1}{2}+\frac{1}{3}+\frac{1}{6}+\frac{1}{7}+\cdots+\frac{1}{98}+\frac{1}{99}.\quad\text{(거짓)}\qed
			\end{align*}
	\end{itemize}
\end{sol}

\newpage
\begin{prob}{2011학년도 9월 모의평가 가형 28번}{27}
	실수 전체의 집합에서 연속인 함수 $f(x)$가 모든 실수 $t$에 대하여
	\[
		\int_0^2xf(tx)\,dx=4t^2
	\]
	을 만족시킬 때, $f(2)$의 값은? [4점]
\end{prob}

\begin{sol}
	$tx=u$라 하면 $\dfrac{du}{dx}=t$이므로
	\begin{align*}
		\int_0^2xf(tx)\,dx=4t^2~&\longrightarrow~\int_0^{2t}\frac{u}{t}f(u)\times\frac{1}{t}\,du=4t^2\\[3pt]
		&\longrightarrow~\int_0^{2t}uf(u)\,du=4t^4\\[3pt]
		&\longrightarrow~4tf(2t)=16t^3\\[3pt]
		&\longrightarrow~f(2t)=4t^2,\quad f(2)=4.\qed
	\end{align*}
\end{sol}

\newpage
\begin{prob}{2014학년도 9월 모의평가 B형 30번}{28}
	두 연속함수 $f(x)$, $g(x)$가
	\[
		g(e^x)=\begin{cases}
			f(x) & (0\leq x<1)\\
			g(e^{x-1})+5 & (1\leq x\leq2)
		\end{cases}
	\]
	를 만족시키고, $\displaystyle\int_1^{e^2}g(x)\,dx=6e^2+4$이다. $\displaystyle\int_1^ef(\ln x)\,dx=ae+b$일 때, $a^2+b^2$의 값을 구하시오. \\ (단, $a$, $b$는 정수이다.) [4점]
\end{prob}

\begin{sol}
	주어진 관계식에 $\ln x$를 대입하면
	\[
		g(x)=\begin{cases}
			f(\ln x) & (1\leq x<e)\\[3pt]
			g\biggl(\dfrac{x}{e}\biggr)+5 & (e\leq x\leq e^2)
		\end{cases}
	\]
	이므로
	\[
		\int_1^{e^2}g(x)\,dx=\int_1^ef(\ln x)\,dx+\int_e^{e^2}\biggl\{g\biggl(\frac{x}{e}\biggr)+5\biggr\}\,dx.
	\]
	$\dfrac{x}{e}=t$라 하면 $\dfrac{dt}{dx}=\dfrac{1}{e}$이므로
	\begin{align*}
		\int_e^{e^2}\biggl\{g\biggl(\frac{x}{e}\biggr)+5\biggr\}\,dx&=e\int_1^e\{g(t)+5\}\,dt=e\int_1^eg(t)\,dt+5e^2-5e\\[3pt]
		&=e\int_1^e\{g(t)+5\}\,dt=e\int_1^ef(\ln t)\,dt+5e^2-5e.
	\end{align*}
	따라서
	\begin{align*}
		6e^2+4=(e+1)\int_1^ef(\ln x)\,dx+5e^2-5e~\longrightarrow~&(e+1)\int_1^ef(\ln x)\,dx=e^2+5e+4=(e+1)(e+4)\\[3pt]
		\longrightarrow~&\int_1^ef(\ln x)\,dx=e+4
	\end{align*}
	에서 $a=1$, $b=4$이므로 $a^2+b^2=17$이다.\qed
\end{sol}

\newpage
\begin{prob}{2014학년도 수능 B형 21번}{29}
	연속함수 $y=f(x)$의 그래프가 원점에 대하여 대칭이고, 모든 실수 $x$에 대하여
	\[
		f(x)=\frac{\pi}{2}\int_1^{x+1}f(t)\,dt
	\]
	이다. $f(1)=1$일 때,
	\[
		\pi^2\int_0^1xf(x+1)\,dx
	\]
	의 값은? [4점]
\end{prob}

\begin{sol}
	주어진 관계식에 의해 함수 $f(x)$는 미분가능하고, $f^{\prime}(x)=\dfrac{\pi}{2}f(x+1)$, $f(0)=0$이다. 원점 대칭에 의해
	\[
		f(-1)=-f(1)=-1~\longrightarrow~\frac{\pi}{2}\int_1^0f(t)\,dt=-1~\longrightarrow~\int_0^1f(t)\,dt=\frac{2}{\pi}.
	\]
	그러므로
	\[
		\pi^2\int_0^1xf(x+1)\,dx=2\pi\int_0^1xf^{\prime}(x)\,dx=2\pi\biggl[xf(x)\biggr]_0^1-2\pi\int_0^1f(x)\,dx=2(\pi-2).\qed
	\]
\end{sol}

\newpage
\begin{prob}{2017학년도 수능 가형 21번}{30}
	닫힌구간 $[0$, $1]$에서 증가하는 연속함수 $f(x)$가
	\[
		\int_0^1f(x)\,dx=2,\quad\int_0^1\vert f(x)\vert\,dx=2\sqrt 2
	\]
	를 만족시킨다. 함수 $F(x)$가
	\[
		F(x)=\int_0^x\vert f(t)\vert\,dt\quad (0\leq x\leq 1)
	\]
	일 때, $\displaystyle\int_0^1f(x)F(x)\,dx$의 값은? [4점]
\end{prob}

\begin{sol}
	$\displaystyle\int_0^1f(x)\,dx<\int_0^1\vert f(x)\vert\,dx$이므로 $f(c)=0$을 만족시키는 $c$가 구간 $(0$, $1)$에 존재한다. 함수 $f(x)$는 증가하므로 구간 $(0$, $c)$에서 $f(x)<0$이고 구간 $(c$, $1)$에서 $f(x)>0$이다. 따라서 $\displaystyle\int_0^cf(x)\,dx=A$, $\displaystyle\int_c^1f(x)\,dx=B$라 하면
	\[
		\int_0^1f(x)\,dx=A+B=2,\quad\int_0^1\vert f(x)\vert\,dx=-A+B=2\sqrt 2
	\]
	로부터 $A=1-\sqrt2$, $B=1+\sqrt2$이다. 함수 $F(x)$의 정의로부터
	\[
		F^{\prime}(x)=\vert f(x)\vert=\begin{cases}
			-f(x) & (0\leq x<c)\\
			\phantom{-}f(x) & (c\leq x\leq 1)
		\end{cases}
	\]
	이므로 $F(0)=0$, $F(c)=-A=\sqrt2-1$, $F(1)=2\sqrt2$이다. 그러므로
	\begin{align*}
		\int_0^1f(x)F(x)\,dx&=-\int_0^c\{-f(x)\}F(x)\,dx+\int_c^1f(x)F(x)\,dx\\[3pt]
		&=-\int_0^cF^{\prime}(x)F(x)\,dx+\int_c^1f(x)\,F(x)\,dx\\[3pt]
		&=-\biggl[\frac{1}{2}\{F(x)\}^2\biggr]_0^c+\biggl[\dfrac{1}{2}\{F(x)\}^2\biggr]_c^1\\[3pt]
		&=\frac{1}{2}[\{F(1)\}^2-2\{F(c)\}^2+\{F(0)\}^2]\\[3pt]
		&=1+2\sqrt2.\qed
	\end{align*}
\end{sol}

\newpage
\begin{prob}{2022학년도 수능특강}{31}
	실수 전체의 집합에서 미분가능한 두 함수 $f(x)$, $g(x)$가 다음 조건을 만족시킨다.
	\begin{hint}
		\begin{itemize}
			\item[(가)] 모든 실수 $x$에 대하여 $f(x)>0$, $g(x)>0$이다.
			\item[(나)] 모든 실수 $x$에 대하여 $f^{\prime}(x)g(x)-f(x)g^{\prime}(x)=f(x)g(x)$이다.
		\end{itemize}
	\end{hint}
	$f(1)=g(1)$일 때, $\displaystyle\sum_{n=2}^\infty\frac{g(n)}{f(n)}$의 값은? (단, $n$은 자연수이다.)
\end{prob}

\begin{sol}
	(가)에서 $g(x)>0$이므로 (나)에 주어진 식의 양변을 $\{g(x)\}^2$으로 나누자.
	\begin{align*}
		f^{\prime}(x)g(x)-f(x)g^{\prime}(x)=f(x)g(x)~&\longrightarrow~\frac{f^{\prime}(x)g(x)-f(x)g^{\prime}(x)}{\{g(x)\}^2}=\frac{f(x)}{g(x)}\\[3pt]
		&\longrightarrow~\biggl\{\frac{f(x)}{g(x)}\biggr\}^{\prime}=\frac{f(x)}{g(x)}\\[3pt]
		&\longrightarrow~e^{-x}\biggl[\biggl\{\frac{f(x)}{g(x)}\biggr\}^{\prime}-\frac{f(x)}{g(x)}\biggr]=0\\[3pt]
		&\longrightarrow~\biggl\{e^{-x}\frac{f(x)}{g(x)}\biggr\}^{\prime}=0\\[3pt]
		&\longrightarrow~\frac{f(x)}{g(x)}=Ce^{x}
	\end{align*}
	$f(1)=g(1)$에서 $C=e^{-1}$이다. 따라서 $\displaystyle\sum_{n=2}^\infty\frac{g(n)}{f(n)}=\displaystyle\sum_{n=2}^\infty e^{-n+1}=\frac{e^{-1}}{1-e^{-1}}=\frac{1}{e-1}$이다.\qed
\end{sol}

\newpage
\begin{prob}{2017년 5월 전북교육청 가형 30번}{32}
	열린구간 $\biggl(-\dfrac{\pi}{2}$, $\dfrac{\pi}{2}\biggr)$에서 미분가능하고 $f(0)=1$인 함수 $f(x)$가 $-\dfrac{\pi}{2}<x<\dfrac{\pi}{2}$인 모든 실수 $x$에 대하여 다음 조건을 만족시킨다.
	\begin{hint}
		\begin{itemize}
			\item[(가)] $f(x)>0$
			\item[(나)] $\biggl(\dfrac{1}{f(x)\cos x}\biggr)^{\prime}=\dfrac{x}{\cos x}$
		\end{itemize}
	\end{hint}
	$g(x)=\displaystyle\int_0^x\dfrac{\tan t}{f(t)}\,dt$라 할 때, $g(4)+\dfrac{1}{f(4)}$의 값을 구하시오. [4점]
\end{prob}

\begin{sol}
	\[
		\begin{tikzcd}
			D & I & \\[-12pt]
			\dfrac{1}{f(t)\cos t} \arrow[rd, "+"] & \sin t & \\
			\dfrac{t}{t\cos t} & -\cos t \arrow[r] & -t \\[-12pt] \hline \\[-12pt]
			1 \arrow[rd, "-"] & -t & \\
			0 & -\dfrac{t^2}{2} &
		\end{tikzcd}\quad\begin{aligned}
			g(4)&=\int_0^4\frac{\tan t}{f(t)}\,dt\\[3pt]
			&=\int_0^4\biggl\{\sin t\times \frac{1}{f(t)\cos t}\biggr\}\,dt\\[3pt]
			&=\biggl[-\frac{1}{f(t)}+\frac{1}{2}t^2\biggr]_0^4\\[3pt]
			&=-\frac{1}{f(4)}+\frac{1}{f(0)}+8
		\end{aligned}
	\]
	따라서 $g(4)+\dfrac{1}{f(4)}=\dfrac{1}{f(0)}+8=9$이다.\qed
\end{sol}

\newpage
\begin{prob}{2022학년도 수능특강}{33}
	실수 전체의 집합에서 이계도함수가 연속인 함수 $f(x)$가
	\[
		\int_0^\pi\{f^{\prime\prime}(x)+4f(x)\}\sin 2x\,dx+2\pi^2=0
	\]
	을 만족시킨다. $f(0)=1$일 때, $f(\pi)$의 값은?
\end{prob}

\begin{sol}
	\[
		\begin{tikzcd}
			D & I \\[-12pt]
			\sin 2x \arrow[rd, "+"] & f^{\prime\prime}(x) \\
			2\cos 2x \arrow[rd, "-"] & f^\prime(x) \\
			-4\sin 2x \arrow[r, "+"] & f(x)
		\end{tikzcd}
	\]
	\[
		\begin{aligned}
			&\int_0^\pi f^{\prime\prime}(x)\sin 2x\,dx=\biggl[f^{\prime}(x)\sin2x-2f(x)\cos2x\biggr]_0^\pi-4\int_0^\pi f(x)\sin2x\,dx\\[3pt]
			&\int_0^\pi\{f^{\prime\prime}(x)+4f(x)\}\sin2x\,dx=\biggl[f^{\prime}(x)\sin2x-2f(x)\cos2x\biggr]_0^\pi=2f(0)-2f(\pi)
		\end{aligned}
	\]
	따라서 $f(\pi)=f(0)-\dfrac{1}{2}\displaystyle\int_0^\pi\{f^{\prime\prime}(x)+4f(x)\}\sin2x\,dx=\pi^2+1$이다.\qed
\end{sol}

\newpage
\begin{prob}{2018년 7월 학력평가 가형 20번}{34}
	양의 실수 전체의 집합에서 미분가능한 두 함수 $f(x)$와 $g(x)$가 다음 조건을 만족시킨다.
	\begin{hint}
		\begin{itemize}
			\item[(가)] 모든 양의 실수 $x$에 대하여 $g(x)=\displaystyle\int_1^x\dfrac{f(t^2+1)}{t}\,dt$
			\item[(나)] $\displaystyle\int_2^5f(x)\,dx=16$
		\end{itemize}
	\end{hint}
	$g(2)=3$일 때, $\displaystyle\int_1^2xg(x)\,dx$의 값은? [4점]
\end{prob}

\begin{sol}
	(가)에서 $g(1)=0$이고 $g^\prime(x)=\dfrac{f(x^2+1)}{x}$이다.
	\[
		\begin{tikzcd}
			D & I \\[-12pt]
			g(x) \arrow[rd, "+"] & x \\
			\dfrac{f(x^2+1)}{x} \arrow[r, "-"] & \dfrac{1}{2}x^2
		\end{tikzcd}\quad\begin{aligned}
			\int_1^2xg(x)\,dx&=\biggl[\frac{1}{2}x^2g(x)\biggr]_1^2-\frac{1}{2}\int_1^2xf(x^2+1)\,dx\\[3pt]
			&=2g(2)-\frac{1}{2}g(1)-\frac{1}{2}\int_1^2xf(x^2+1)\,dx
		\end{aligned}
	\]
	$x^2+1=t$라 하면 $\dfrac{dt}{dx}=2x$이므로 $\displaystyle\int_1^2xf(x^2+1)\,dx=\frac{1}{2}\int_2^5f(t)\,dt=8$이고 $\displaystyle\int_1^2xg(x)\,dx=2$이다.\qed
\end{sol}

\newpage
\begin{prob}{2023학년도 수능완성}{35}
	실수 전체의 집합에서 미분가능한 함수 $f(x)$와 도함수 $f^{\prime}(x)$가 다음 조건을 만족시킨다.
	\begin{hint}
		\begin{itemize}
			\item[(가)] 모든 실수 $x$에 대하여 $f(x)>0$, $2x\{f(x)\}^2+f^{\prime}(x)=0$이다.
			\item[(나)] $f(0)=1$, $f^{\prime}(2)=-\dfrac{4}{25}$
		\end{itemize}
	\end{hint}
	$\displaystyle\int_0^2x^3\{f(x)\}^3\,dx=\frac{q}{p}$일 때, $p+q$의 값을 구하시오. (단, $p$와 $q$는 서로소인 자연수이다.) [4점]
\end{prob}

\begin{sol}[풀이 1]
	(가)와 (나)에 $x=2$를 대입하면 $f(2)=\dfrac{1}{5}$이다.
	\[
		\begin{tikzcd}
			D & I & \\[-12pt]
			-\dfrac{1}{2}x^2 \arrow[rd, "+"] & f^\prime(x)f(x) & \\
			-x & \dfrac{1}{2}\{f(x)\}^2 \arrow[r] & -\dfrac{1}{2}x\{f(x)\}^2=\dfrac{1}{4}f^\prime(x) \\[-12pt] \hline \\[-12pt]
			1 \arrow[rd, "-"] & \dfrac{1}{4}f^\prime(x) & \\
			0 & \dfrac{1}{4}f(x) &
		\end{tikzcd}
	\]
	\[
		\int_0^2x^3\{f(x)\}^3\,dx=\int_0^2\biggl\{-\frac{1}{2}x^2f^{\prime}(x)f(x)\biggr\}\,dx=\biggl[-\frac{1}{4}x^2\{f(x)\}^2-\frac{1}{4}f(x)\biggr]_0^2=\frac{4}{25}\qed
	\]
\end{sol}

\begin{sol}[풀이 2]
	\[
		2x\{f(x)\}^2+f^{\prime}(x)=0~\longrightarrow~-\frac{f^{\prime}(x)}{\{f(x)\}^2}=2x~\longrightarrow~\frac{1}{f(x)}=x^2+C~\longrightarrow~f(x)=\frac{1}{x^2+C}
	\]
	$f(0)=1$이므로 $C=1$이고 $f(x)=\dfrac{1}{x^2+1}$이다. $x^2+1=t$라 하면 $\dfrac{dt}{dx}=2x$이므로
	\begin{align*}
		\int_0^2x^3\{f(x)\}^2\,dx&=\int_0^2\frac{x^3}{(x^2+1)^3}\,dx=\frac{1}{2}\int_1^5\frac{t-1}{t^3}\,dt\\[3pt]
		&=\frac{1}{2}\int_1^5\biggl(\frac{1}{t^2}-\frac{1}{t^3}\biggr)dt=\frac{1}{2}\biggl[-\frac{1}{t}+\frac{1}{2t^2}\biggr]_1^5=\frac{4}{25}.\qed
	\end{align*}
\end{sol}

\newpage
\begin{prob}{2023학년도 수능 미적분 29번}{36}
	세 상수 $a$, $b$, $c$에 대하여 함수 $f(x)=ae^{2x}+be^x+c$가 다음 조건을 만족시킨다.
	\begin{hint}
		\begin{itemize}
			\item[(가)] $\lim\limits_{x\,\to\,-\infty}\dfrac{f(x)+6}{e^x}=1$
			\item[(나)] $f(\ln 2)=0$
		\end{itemize}
	\end{hint}
	함수 $f(x)$의 역함수를 $g(x)$라 할 때, $\displaystyle\int_0^{14}g(x)\,dx=p+q\ln2$이다. $p+q$의 값을 구하시오. \\ (단, $p$, $q$는 유리수이고, $\ln 2$는 무리수이다.) [4점]
\end{prob}

\begin{sol}
	(가)에서 극한
	\[
		\lim\limits_{x\,\to-\infty}\dfrac{f(x)+6}{e^x}=\lim\limits_{x\,\to-\infty}\{ae^x+(c+6)e^{-x}+b\}
	\]
	이 수렴하므로 $c=-6$이고, 극한값이 $1$이므로 $b=1$이다. (나)에서
	\[
		f(x)=ae^{2x}+e^x-6~\longrightarrow~f(\ln2)=4a-4=0~\longrightarrow~a=1
	\]
	이므로 $f(x)=e^{2x}+e^x-6$이다. $f(t)=14$라 하면
	\[
		e^{2t}+e^t-20=0~\longrightarrow~(e^t+5)(e^t-4)=0~\longrightarrow~e^t=4~\longrightarrow~b=\ln4.
	\]
	따라서
	\[
		\begin{aligned}
			&\int_{\ln2}^{\ln4}f(x)\,dx=\int_{\ln2}^{\ln4}(e^{2x}+e^x-6)\,dx=\biggl[\frac{1}{2}e^{2x}+e^x-6x\biggr]_{\ln2}^{\ln4}=8-6\ln2,\\[3pt]
			&\int_0^{14}g(x)\,dx=(\ln4)f(\ln4)-(\ln2)f(\ln2)-\int_{\ln2}^{\ln4}f(x)\,dx=-8+34\ln2
		\end{aligned}
	\]
	이므로 $p=-8$, $q=34$, $p+q=26$이다.\qed
\end{sol}

\newpage
\begin{prob}{2020학년도 6월 모의평가 가형 20번}{37}
	실수 전체의 집합에서 미분가능한 함수 $f(x)$가 모든 실수 $x$에 대하여 다음 조건을 만족시킨다.
	\begin{hint}
		\begin{itemize}
			\item[(가)] $f(x)>0$
			\item[(나)] $\ln f(x)+2\displaystyle\int_0^x(x-t)f(t)\,dt=0$
		\end{itemize}
	\end{hint}
	$\langle$보기$\rangle$에서 옳은 것만을 있는 대로 고른 것은? [4점]
	\begin{hint}[보기]
		\begin{itemize}
			\item[ㄱ.] $x>0$에서 함수 $f(x)$는 감소한다.
			\item[ㄴ.] 함수 $f(x)$의 최댓값은 $1$이다.
			\item[ㄷ.] 함수 $F(x)$를 $F(x)=\displaystyle\int_0^xf(t)\,dt$라 할 때, $f(1)+\{F(1)\}^2=1$이다.
		\end{itemize}
	\end{hint}
\end{prob}

\begin{sol}
	주어진 등식에 $x=0$을 대입하면 $f(0)=1$이다.
	\begin{align*}
		\ln f(x)+2\int_0^x(x-t)f(t)\,dt=0~&\longrightarrow~\ln f(x)+2x\int_0^xf(t)\,dt-2\int_0^xtf(t)\,dt=0\\[3pt]
		&\longrightarrow~\frac{f^{\prime}(x)}{f(x)}+2\int_0^xf(t)\,dt+2xf(x)-2f(x)=0\\[3pt]
		&\longrightarrow~f^{\prime}(x)=-2f(x)\int_0^xf(t)\,dt
	\end{align*}
	\begin{itemize}
		\item[ㄱ.] $x>0$에서 $f(x)>0$, $\displaystyle\int_0^xf(t)\,dt>0$이므로 $f^{\prime}(x)<0$, 즉 함수 $f(x)$는 $x>0$에서 감소한다. (참)
		\item[ㄴ.] 같은 논리로 $x<0$에서 $f(x)>0$, $\displaystyle\int_0^xf(t)\,dt<0$이므로 $f^{\prime}(x)>0$, 즉 함수 $f(x)$는 $x<0$에서 증가한다.\\[3pt] 따라서 함수 $f(x)$는 $x=0$에서 극대이자 최대이고, 그 함숫값은 $1$이다. (참)
		\item[ㄷ.] $F^{\prime}(x)=f(x)$이므로
			\[
				f^{\prime}(x)=-2F^{\prime}(x)F(x)~\longrightarrow~f(x)=-\{F(x)\}^2+C~\longrightarrow~f(x)+\{F(x)\}^2=C.
			\]
			$F(0)=0$에서 $C=1$이고 $f(1)+\{F(1)\}^2=1$이다.\qed
	\end{itemize}
\end{sol}

\begin{remark}
	$f(x)$의 식을 닫힌 형태로 구할 수 있다.
	\[
		\begin{aligned}
			f(x)+\{F(x)\}^2=1~&\longrightarrow~F^{\prime}(x)+\{F(x)\}^2=1\longrightarrow~\frac{F^{\prime}(x)}{1-\{F(x)\}^2}=1\\[3pt]
			&\longrightarrow~\frac{F^{\prime}(x)}{1+F(x)}+\frac{F^{\prime}(x)}{1-F(x)}=2\\[3pt]
			&\longrightarrow~\ln\vert 1+F(x)\vert-\ln\vert1-F(x)\vert=2x+C\quad(F(0)=0~\Longrightarrow~C=0)\\[3pt]
			&\longrightarrow~\frac{1+F(x)}{1-F(x)}=e^{2x}\longrightarrow~F(x)=\frac{e^{2x}-1}{e^{2x}+1},\quad f(x)=\frac{4e^{2x}}{(e^{2x}+1)^2}
		\end{aligned}
	\]
\end{remark}

\newpage
\begin{prob}{2019학년도 수능완성}{38}
	구간 $[0$, $\infty)$에서 연속인 함수 $f(x)$가 다음 조건을 만족시킨다.
	\begin{hint}
		\begin{itemize}
			\item[(가)] $\displaystyle\int_0^1f(x)\,dx=8$
			\item[(나)] $\displaystyle\int_0^1x^2f(x^2)\,dx=3$
		\end{itemize}
	\end{hint}
	함수 $F(x)$가 $F(x)=\displaystyle\int_0^xtf(t^2)\,dt$일 때, $\displaystyle\int_0^1F(x)\,dx$의 값은? [4점]
\end{prob}

\begin{sol}
	$F(0)=0$, $F^{\prime}(x)=xf(x^2)$이고, $t^2=u$라 하면 $\dfrac{du}{dt}=2t$이므로
	\[
		F(1)=\int_0^1tf(t^2)\,dt=\frac{1}{2}\int_0^1f(u)\,du=4.
	\]
	\[
		\begin{tikzcd}
			D & I \\[-12pt]
			F(x) \arrow[rd, "+"] & 1 \\
			xf(x^2) \arrow[r, "-"] & x
		\end{tikzcd}\qquad\begin{aligned}
			\int_0^1F(x)\,dx&=\biggl[xF(x)\biggr]_0^1-\int_0^1x^2f(x^2)\,dx\\[3pt]
			&=F(1)-\int_0^1x^2f(x^2)\,dx=1\qed
		\end{aligned}
	\]
\end{sol}

\newpage
\begin{prob}{2022학년도 예비시행 미적분 29번}{39}
	함수 $f(x)=e^x+x-1$과 양수 $t$에 대하여 함수
	\[
		F(x)=\int_0^x\{t-f(s)\}\,ds
	\]
	가 $x=\alpha$에서 최댓값을 가질 때, 실수 $\alpha$의 값을 $g(t)$라 하자. 미분가능한 함수 $g(t)$에 대하여 $\displaystyle\int_{f(1)}^{f(5)}\dfrac{g(t)}{1+e^{g(t)}}\,dt$의 값을 구하시오. [4점]
\end{prob}

\begin{sol}
	$f^{\prime}(x)=e^x+1>0$로부터 함수 $f(x)$는 실수 전체의 집합에서 증가하므로
	\[
		F^{\prime}(x)=t-f(x)~\longrightarrow~t-f(g(t))=0
	\]
	이고 $g(x)$는 $f(x)$의 역함수이다. $t=f(u)$라 하면 $u=g(t)$, $\dfrac{dt}{du}=f^{\prime}(u)=e^u+1$이므로
	\[
		\int_{f(1)}^{f(5)}\dfrac{g(t)}{1+e^{g(t)}}\,dt=\int_1^5\frac{u}{1+e^u}(1+e^u)\,du=\int_1^5u\,du=\biggl[\frac{1}{2}u^2\biggr]_1^5=12.\qed
	\]
\end{sol}

\newpage
\begin{prob}{2020년 7월 학력평가 가형 19번}{40}
	실수 전체의 집합에서 $f(x)>0$이고 도함수가 연속인 함수 $f(x)$가 있다. 실수 전체의 집합에서 함수 $g(x)$가
	\[
		g(x)=\int_0^x\ln f(t)\,dt
	\]
	일 때, 함수 $g(x)$와 $g(x)$의 도함수 $g^{\prime}(x)$는 다음 조건을 만족시킨다.
	\begin{hint}
		\begin{itemize}
			\item[(가)] 함수 $g(x)$는 $x=1$에서 극값 $2$를 갖는다.
			\item[(나)] 모든 실수 $x$에 대하여 $g^{\prime}(-x)=g^{\prime}(x)$이다.
		\end{itemize}
	\end{hint}
	$\displaystyle\int_{-1}^1\dfrac{xf^{\prime}(x)}{f(x)}\,dx$의 값은? [4점]
\end{prob}

\begin{sol}
	\[
		g(0)=0~\Longrightarrow~g^{\prime}(x)=\ln f(x)~\Longrightarrow~g^{\prime\prime}(x)=\dfrac{f^{\prime}(x)}{f(x)}
	\]
	(가)에서 $g(1)=2$, $g^\prime(1)=0$이다. (나)에서 모든 실수 $x$에 대하여 $g^{\prime\prime}(x)=-g^{\prime\prime}(x)$이다.
	\[
		\begin{tikzcd}
			D & I \\[-12pt]
			x \arrow[rd, "+"] & g^{\prime\prime}(x) \\
			1 \arrow[rd, "-"] & g^\prime(x) \\
			0 & g(x)
		\end{tikzcd}\quad\begin{aligned}
			\int_{-1}^1\dfrac{xf^{\prime}(x)}{f(x)}\,dx&=\int_{-1}^1xg^{\prime\prime}(x)\,dx=2\int_0^1xg^{\prime\prime}(x)\,dx\\[3pt]
			&=2\biggl[xg^{\prime}(x)-g(x)\biggr]_0^1=-4\qed
		\end{aligned}
	\]
\end{sol}

\newpage
\begin{prob}{2019학년도 수능 가형 21번}{41}
	실수 전체의 집합에서 미분가능한 함수 $f(x)$가 다음 조건을 만족시킬 때, $f(-1)$의 값은? [4점]
	\begin{hint}
		\begin{itemize}
			\item[(가)] 모든 실수 $x$에 대하여 $2\{f(x)\}^2f^{\prime}(x)=\{f(2x+1)\}^2f^{\prime}(2x+1)$이다.
			\item[(나)] $f\biggl(-\dfrac{1}{8}\biggr)=1$, $f(6)=2$
		\end{itemize}
	\end{hint}
\end{prob}

\begin{sol}
	(가)에서 주어진 식의 양변을 부정적분하면
	\[
		\frac{2}{3}\{f(x)\}^3+C^{\prime}=\frac{1}{6}\{f(2x+1)\}^3~\longrightarrow~4\{f(x)\}^3+C=\{f(2x+1)\}^3~(C=6C^{\prime}).
	\]
	$x=-\dfrac{1}{8}$부터 순차적으로 대입하면
	\[
		\begin{aligned}
			&x=-\frac{1}{8}~\longrightarrow~\biggl\{f\biggl(\frac{3}{4}\biggr)\biggr\}^3=4\biggl\{f\biggl(-\frac{1}{8}\biggr)\biggr\}^3+C=4+C,\\[3pt]
			&x=\phantom{-}\frac{3}{4}~\longrightarrow~\biggl\{f\biggl(\dfrac{5}{2}\biggr)\biggr\}^3=4\biggl\{f\biggl(\dfrac{3}{4}\biggr)\biggr\}^3+C=16+5C,\\[3pt]
			&x=\phantom{-}\frac{5}{2}~\longrightarrow~\{f(6)\}^3=4\biggl\{f\biggl(\dfrac{5}{2}\biggr)\biggr\}^3+C=64+21C
		\end{aligned}
	\]
	이므로
	\[
		f(6)=2~\longrightarrow~8=64+21C~\longrightarrow~C=-\frac{8}{3}.
	\]
	$x=-1$을 대입하면
	\[
		4\{f(-1)\}^3-\frac{8}{3}=\{f(-1)\}^3~\longrightarrow~\{f(-1)\}^3=\frac{8}{9}~\longrightarrow~f(-1)=\frac{2}{\sqrt[3]{9}}=\frac{2\sqrt[3]{3}}{3}.\qed
	\]
\end{sol}

\newpage
\begin{prob}{2021년 10월 학력평가 미적분 27번}{42}
	미분가능한 함수 $f(x)$가 다음 조건을 만족시킨다.
	\begin{hint}
		\begin{itemize}
			\item[(가)] $x_1<x_2$인 임의의 두 실수 $x_1$, $x_2$에 대하여 $f(x_1)>f(x_2)$이다.
			\item[(나)] 닫힌구간 $[-1$, $3]$에서 함수 $f(x)$의 최댓값은 $1$이고 최솟값은 $-2$이다.
		\end{itemize}
	\end{hint}
	$\displaystyle\int_{-1}^3f(x)\,dx=3$일 때, $\displaystyle\int_{-2}^1f^{-1}(x)\,dx$의 값은? [3점]
\end{prob}

\begin{sol}
	(가), (나)에서 $f(-1)=1$, $f(3)=-2$이다.
	\[
		\int_{-2}^1f^{-1}(x)\,dx=(-1)f(-1)-3f(3)-\int_3^{-1}f(x)\,dx=8\qed
	\]
\end{sol}

\newpage
\begin{prob}{2017년 3월 학력평가 가형 21번}{43}
	구간 $[0$, $1]$에서 정의된 연속함수 $f(x)$에 대하여 함수
	\[
		F(x)=\int_0^xf(t)\,dt\quad(0\leq x\leq 1)
	\]
	은 다음 조건을 만족시킨다.
	\begin{hint}
		\begin{itemize}
			\item[(가)] $F(x)=f(x)-x$
			\item[(나)] $\displaystyle\int_0^1F(x)\,dx=e-\frac{5}{2}$
		\end{itemize}
	\end{hint}
	$\langle$보기$\rangle$에서 옳은 것만을 있는 대로 고른 것은? [4점]
	\begin{hint}[보기]
		\begin{itemize}
			\item[ㄱ.] $F(1)=e$
			\item[ㄴ.] $\displaystyle\int_0^1xF(x)\,dx=\frac{1}{6}$
			\item[ㄷ.] $\displaystyle\int_0^1\{F(x)\}^2\,dx=\frac{1}{2}e^2-2e+\frac{11}{6}$
		\end{itemize}
	\end{hint}
\end{prob}

\begin{sol}[풀이 1]
	먼저 $F(0)=0$, $F^{\prime}(x)=f(x)$이다.
	\begin{itemize}
		\item[ㄱ.] $\displaystyle\int_0^1F(x)\,dx=\int_0^1\{f(x)-x\}\,dx=F(1)-\dfrac{1}{2}~\longrightarrow~F(1)=\biggl(e-\frac{5}{2}\biggr)+\frac{1}{2}=e-2$\quad(거짓)
		\item[ㄴ.] \[
				\begin{aligned}
					&\int_0^1xf(x)\,dx=\biggl[xF(x)\biggr]_0^1-\int_0^1F(x)\,dx=\frac{1}{2}\\[3pt]
					\longrightarrow~&\int_0^1xF(x)\,dx=\int_0^1\{xf(x)-x^2\}\,dx=\int_0^1xf(x)\,dx-\frac{1}{3}=\frac{1}{6}\quad\text{(참)}
				\end{aligned}
			\]
		\item[ㄷ.] \[
				\begin{aligned}
					\int_0^1\{F(x)\}^2\,dx&=\int_0^1\{f(x)-x\}F(x)\,dx=\int_0^1f(x)F(x)\,dx-\int_0^1xF(x)\,dx\\[3pt]
					&=\biggl[\frac{1}{2}\{F(x)\}^2\biggr]_0^1-\frac{1}{6}=\frac{1}{2}(e-2)^2-\frac{1}{6}=\frac{1}{2}e^2-2e+\frac{11}{6}\quad\text{(참)}\qed
				\end{aligned}
			\]
	\end{itemize}
\end{sol}

\begin{sol}[풀이 2]
	(가)에서
	\[
		\begin{aligned}
			F(x)=F^{\prime}(x)-x~&\longrightarrow~F^{\prime}(x)-F(x)=x\longrightarrow~e^{-x}\{F^{\prime}(x)-F(x)\}=xe^{-x}\\[3pt]
			&\longrightarrow~\{e^{-x}F(x)\}^{\prime}=xe^{-x}\longrightarrow~e^{-x}F(x)=\int xe^{-x}\,dx=-(x+1)e^{-x}+C.
		\end{aligned}
	\]
	$F(0)=0$이므로 $C=1$이고 $F(x)=e^x-x-1$이다.
	\begin{itemize}
		\item[ㄱ.] $F(1)=e-2$\quad(거짓)
		\item[ㄴ.] $\displaystyle\int_0^1xF(x)\,dx=\int_0^1(xe^x-x^2-x)\,dx=\int_0^1xe^x\,dx-\frac{5}{6}=\biggl[(x-1)e^x\biggr]_0^1-\frac{5}{6}=\frac{1}{6}$\quad(참)
		\item[ㄷ.] \[
				\begin{aligned}
					\int_0^1\{F(x)\}^2\,dx&=\int_0^1(e^x-x-1)^2\,dx=\int_0^1(e^{2x}-2xe^x-2e^x+x^2+2x+1)\,dx\\[3pt]
					&=\biggl[\frac{1}{2}e^{2x}-2(x-1)e^x-2e^x\biggr]_0^1+\frac{7}{3}=\frac{1}{2}e^2-2e+\frac{11}{6}\quad\text{(참)}\qed
				\end{aligned}
			\]
	\end{itemize}
\end{sol}

\begin{remark}
	$F(x)=f(x)-x$에서 곧바로 양변을 미분하여 $f(x)=f^{\prime}(x)-x$라 할 수 없다. $f$의 미분가능성이 보장되지 않았기 때문이다. $F(x)$의 식을 찾고 나서야 $f$가 미분가능함을 알 수 있다.
\end{remark}

\newpage
\begin{prob}{2018년 4월 학력평가 가형 21번}{44}
	$\dfrac{3}{5}<x<4$에서 정의된 미분가능한 함수 $f(x)$가 $f(1)=2$이고
	\[
		f^{\prime}(x)=\frac{1-x^2\{f(x)\}^3}{x^3\{f(x)\}^2}
	\]
	을 만족시킨다. 함수 $f(x)$의 역함수 $g(x)$가 존재하고 미분가능할 때, <보기>에서 옳은 것만을 있는 대로 고른 것은? [4점]
	\begin{hint}[보기]
		\begin{itemize}
			\item[ㄱ.] $g^{\prime}(2)=-\dfrac{4}{7}$
			\item[ㄴ.] $g(x)=\dfrac{1}{3}x^3\{g(x)\}^3-\dfrac{5}{3}$
			\item[ㄷ.] $2<g(1)<\dfrac{5}{2}$
		\end{itemize}
	\end{hint}
\end{prob}

\begin{sol}[풀이 1]
	역함수의 미분법으로부터 $g^{\prime}(x)=\dfrac{1}{f^{\prime}(g(x))}=\dfrac{x^2\{g(x)\}^3}{1-x^3\{g(x)\}^2}$이다.
	\begin{itemize}
		\item[ㄱ.] $g(2)=1~\longrightarrow~g^{\prime}(2)=\dfrac{2^2\{g(2)\}^3}{1-2^3\{g(2)\}^2}=-\dfrac{4}{7}$\quad(참)
		\item[ㄴ.] \[
				\begin{aligned}
					g^{\prime}(x)=\frac{x^2\{g(x)\}^3}{1-x^3\{g(x)\}^2}~&\longrightarrow~g^{\prime}(x)=x^2\{g(x)\}^2\{xg^{\prime}(x)+g(x)\}=\{xg(x)\}^2\{xg(x)\}^{\prime}\\[3pt]
					&\longrightarrow~g(x)=\frac{1}{3}x^3\{g(x)\}^3+C
				\end{aligned}
			\]
		    $g(2)=1$에서 $C=-\dfrac{5}{3}$, $g(x)=\dfrac{1}{3}x^3\{g(x)\}^3-\dfrac{5}{3}$이다. (참)
		\item[ㄷ.] \[
				g(1)=\frac{1}{3}\{g(1)\}^3-\frac{5}{3}~\longrightarrow~\{g(1)\}^3-3g(1)-5=0
			\]
			$h(x)=x^3-3x-5$라 하면 $h^{\prime}(x)=3(x+1)(x-1)$이므로 극댓값 $h(-1)<0$, 극솟값 $h(1)<0$을 갖는다. 따라서 방정식 $h(x)=0$은 오직 하나의 실근 $g(1)$만을 갖는다. 이때 $h(2)<0$, $h\biggl(\dfrac{5}{2}\biggr)>0$이므로 사잇값 정리에 의해 $2<g(1)<\dfrac{5}{2}$이다. (참)\qed
	\end{itemize}
\end{sol}

\begin{sol}[풀이 2]
	\[
		\begin{aligned}
			f^{\prime}(x)=\frac{1-x^2\{f(x)\}^3}{x^3\{f(x)\}^2}~&\longrightarrow~3x^3\{f(x)\}^2f^{\prime}(x)+3x^2\{f(x)\}^3=3\\[3pt]
			&\longrightarrow~x^3\cdot3\{f(x)\}^2f^{\prime}(x)+3x^2\cdot\{f(x)\}^3=3\\[3pt]
			&\longrightarrow~[x^3\{f(x)\}^3]^{\prime}=3\\[3pt]
			&\longrightarrow~x^3\{f(x)\}^3=3x+C\quad(f(1)=2~\Longrightarrow~C=5)\\[3pt]
			&\longrightarrow\{f(x)\}^3=\frac{3}{x^2}+\frac{5}{x^3}\\[3pt]
			&\longrightarrow~f(x)=\sqrt[3]{\frac{3}{x^2}+\frac{5}{x^3}}=\frac{\sqrt[3]{3x+5}}{x}
		\end{aligned}
	\]
	\begin{itemize}
		\item[ㄱ.] 풀이 1과 동일 (참)
		\item[ㄴ.] $x^3\{f(x)\}^3=3x+5~\longrightarrow~x^3\{g(x)\}^3=3g(x)+5~\longrightarrow~g(x)=\dfrac{1}{3}\{g(x)\}^3-\dfrac{5}{3}$\quad(참)
		\item[ㄷ.] $f(x)$는 감소함수이므로 $\{f(2)\}^3>\{f(g(1))\}^3=1>\biggl\{f\biggl(\dfrac{5}{2}\biggr)\biggr\}^3$를 확인하면 된다.
			\[
				\{f(2)\}^3=\frac{3}{4}+\frac{5}{8}=\frac{11}{8}>1,\quad\biggl\{f\biggl(\frac{5}{2}\biggr)\biggr\}^3=\frac{12}{25}+\frac{40}{125}=\frac{100}{125}<1
			\]
			이므로 $2<g(1)<\dfrac{5}{2}$이다. (참)\qed
	\end{itemize}
\end{sol}

\newpage
\begin{prob}{2010학년도 수능 가형 29번}{45}
	실수 전체의 집합에서 이계도함수를 갖는 두 함수 $f(x)$와 $g(x)$에 대하여 정적분
	\[
		\int_0^1\{f^{\prime}(x)g(1-x)-g^{\prime}(x)f(1-x)\}\,dx
	\]
	의 값을 $k$라 하자. 옳은 것만을 <보기>에서 있는 대로 고른 것은? [4점]
	\begin{hint}[보기]
		\begin{itemize}
			\item[ㄱ.] $\displaystyle\int_0^1\{f(x)g^{\prime}(1-x)-g(x)f^{\prime}(1-x)\}\,dx=-k$
			\item[ㄴ.] $f(0)=f(1)$이고 $g(0)=g(1)$이면, $k=0$이다.
			\item[ㄷ.] $f(x)=\ln(1+x^4)$이고 $g(x)=\sin\pi x$이면, $k=0$이다.
		\end{itemize}
	\end{hint}
\end{prob}

\begin{sol}[풀이 1]
	\begin{itemize}
		\item[ㄱ.] 정리 \ref{thm:9.1}에 의해
			\[
				\begin{aligned}
					&k=\int_0^1\{f^{\prime}(x)g(1-x)-g^{\prime}(x)f(1-x)\}\,dx=\int_0^1\{f^{\prime}(1-x)g(x)-g^{\prime}(1-x)f(x)\}\,dx\\[3pt]
		&\longrightarrow~\int_0^1\{f(x)g^{\prime}(1-x)-g(x)f^{\prime}(1-x)\}\,dx=-k.\quad\text{(참)}
				\end{aligned}
			\]
		\item[ㄴ.] \[
				\begin{aligned}
					\int_0^1f(1-x)g^{\prime}(x)\,dx&=\biggl[f(1-x)g(x)\biggr]_0^1+\int_0^1f^{\prime}(1-x)g(x)\,dx\\[3pt]
					&=f(0)g(1)-f(1)g(0)+\int_0^1f^{\prime}(1-x)g(x)\,dx
				\end{aligned}
			\]
			에서 주어진 조건과 정리 \ref{thm:9.1}에 의해
			\[
				f(0)g(1)=f(1)g(0),\quad\int_0^1f^{\prime}(1-x)g(x)\,dx=\int_0^1f^{\prime}(x)g(1-x)\,dx~\longrightarrow~k=0.\quad\text{(참)}
			\]
		\item[ㄷ.] ㄴ에서 $k=f(1)g(0)-f(0)g(1)$이고 $g(0)=g(1)=0$이므로 $k=0$. (참)\qed
	\end{itemize}
\end{sol}

\begin{sol}[풀이 2]
	ㄱ에서
	\[
		\begin{cases}
			\phantom{-}k=\displaystyle\int_0^1\{f^{\prime}(x)g(1-x)-g^{\prime}(x)f(1-x)\}\,dx\\[6pt]
			-k=\displaystyle\int_0^1\{f(x)g^{\prime}(1-x)-g(x)f^{\prime}(1-x)\}\,dx
		\end{cases}
	\]
	이므로 두 식을 서로 빼면
	\[
		\begin{aligned}
			2k&=\int_0^1\{f^{\prime}(x)g(1-x)-f(x)g^{\prime}(1-x)+g(x)f^{\prime}(1-x)-g^{\prime}(x)f(1-x)\}\,dx\\[3pt]
			&=\biggl[f(x)g(1-x)-g(x)f(1-x)\biggr]_0^1=2f(1)g(0)-2f(0)g(1)
		\end{aligned}
	\]
	이 성립하므로 $k=f(1)g(0)-f(0)g(1)$이다.\qed
\end{sol}

\newpage
\begin{prob}{2011학년도 수능 가형 28번}{46}
	실수 전체의 집합에서 미분가능한 함수 $f(x)$가 있다. 모든 실수 $x$에 대하여 $f(2x)=2f(x)f^{\prime}(x)$이고,
	\[
		f(a)=0,\quad\int_{2a}^{4a}\frac{f(x)}{x}\,dx=k\,\,(a>0,\,0<k<1)
	\]
	일 때, $\displaystyle\int_a^{2a}\frac{\{f(x)\}^2}{x^2}\,dx$의 값을 $k$로 나타낸 것은? [3점]
\end{prob}

\begin{sol}
	$f(2a)=2f(a)f^{\prime}(a)=0$. $x=2t$라 하면 $\dfrac{dx}{dt}=2$이므로
	\[
		\begin{tikzcd}
			D & I \\[-12pt]
			\dfrac{1}{t} \arrow[rd, "+"] & 2f(t)f^\prime(t) \\
			-\dfrac{1}{t^2} \arrow[r, "-"] & \{f(t)\}^2
		\end{tikzcd}\quad\begin{aligned}
			k&=\int_{2a}^{4a}\frac{f(x)}{x}\,dx\\[12pt]&=\int_a^{2a}\frac{f(2t)}{t}\,dt=\int_a^{2a}\frac{2f(t)f^{\prime}(t)}{t}\,dt\\[3pt]
			&=\biggl[\frac{\{f(t)\}^2}{t}\biggr]_a^{2a}+\int_a^{2a}\frac{\{f(t)\}^2}{t^2}\,dt\\[3pt]
			&=\int_a^{2a}\frac{\{f(t)\}^2}{t^2}\,dt.\qed
		\end{aligned}
	\]
\end{sol}

\newpage
\begin{prob}{2017학년도 9월 모의평가 가형 21번}{47}
	양의 실수 전체의 집합에서 미분가능한 두 함수 $f(x)$와 $g(x)$가 모든 양의 실수 $x$에 대하여 다음 조건을 만족시킨다.
	\begin{hint}
		\begin{itemize}
			\item[(가)] $\biggl(\dfrac{f(x)}{x}\biggr)^{\prime}=x^2e^{-x^2}$
			\item[(나)] $g(x)=\dfrac{4}{e^2}\displaystyle\int_1^xe^{t^2}f(t)\,dt$
		\end{itemize}
	\end{hint}
	$f(1)=\dfrac{1}{e}$일 때, $f(2)-g(2)$의 값은? [4점]
\end{prob}

\begin{sol}
	\[
		\begin{tikzcd}
			D & I & \\[-12pt]
			\dfrac{f(t)}{t} \arrow[rd, "+"] & te^{t^2} & \\
			t^2e^{-t^2} & \dfrac{1}{2}e^{t^2} \arrow[r] & \dfrac{1}{2}t^2 \\[-12pt] \hline \\[-12pt]
			1 \arrow[rd, "-"] & \dfrac{1}{2}t^2 & \\
			0 & \dfrac{1}{6}t^3
		\end{tikzcd}\qquad\begin{aligned}
			g(2)&=\frac{4}{e^4}\int_1^2e^{t^2}f(t)\,dt=\frac{4}{e^4}\int_1^2te^{t^2}\cdot\frac{f(t)}{t}\,dt\\[3pt]
			&=\frac{4}{e^4}\biggl[\frac{f(t)e^{t^2}}{2t}-\frac{t^3}{6}\biggr]_1^2=f(2)-\frac{20}{3e^4}
		\end{aligned}
	\]
	에서 $f(2)-g(2)=\dfrac{20}{3e^4}$이다.\qed
\end{sol}

\newpage
\begin{prob}{2017년 10월 전북교육청 가형 21번}{48}
	미분가능한 함수 $f(x)$가 다음 조건을 만족시킨다.
	\begin{hint}
		\begin{itemize}
			\item[(가)] 모든 실수 $x$에 대하여 $f(x)=e^x+\displaystyle\int_0^1f(x+t)\,dt$이다.
			\item[(나)] $f(0)=1$, $f(1)=2e+3$
		\end{itemize}
	\end{hint}
	$\displaystyle\int_1^2xf(x)\,dx-\int_0^1xf(x)\,dx$의 값은? [4점]
\end{prob}

\begin{sol}
	$x+t=u$라 하면 $\dfrac{du}{dt}=1$이므로 $\displaystyle\int_0^1f(x+t)\,dt=\int_{x}^{x+1}f(u)\,du$이다.
	\[
		f(x)=e^x+\int_x^{x+1}f(u)\,du~\Longrightarrow~f^{\prime}(x)=e^x+f(x+1)-f(x)
	\]
	\[
		\begin{aligned}
			\int_1^2xf(x)\,dx-\int_0^1xf(x)\,dx&=\int_0^1(x+1)f(x+1)\,dx-\int_0^1xf(x)\,dx\\[3pt]
			&=\int_0^1(x+1)\{f^{\prime}(x)+f(x)-e^x\}\,dx-\int_0^1xf(x)\,dx\\[3pt]
			&=\int_0^1(x+1)f(x)\,dx+\int_0^1(x+1)f^{\prime}(x)\,dx-\int_0^1(x+1)e^x\,dx-\int_0^1xf(x)\,dx\\[3pt]
			&=\int_0^1f(x)\,dx+\biggl[(x+1)f(x)\biggr]_0^1-\int_0^1f(x)\,dx-\biggl[xe^x\biggr]_0^1\\[3pt]
			&=2f(1)-f(0)-e=3e+5.\qed
		\end{aligned}
	\]
\end{sol}

\newpage
\begin{prob}{2022학년도 수능 미적분 30번}{49}
	실수 전체의 집합에서 증가하고 미분가능한 함수 $f(x)$가 다음 조건을 만족시킨다.
	\begin{hint}
		\begin{itemize}
			\item[(가)] $f(1)=1$, $\displaystyle\int_1^2f(x)\,dx=\frac{5}{4}$
			\item[(나)] 함수 $f(x)$의 역함수를 $g(x)$라 할 때, $x\geq1$인 모든 실수 $x$에 대하여 $g(2x)=2f(x)$이다.
		\end{itemize}
	\end{hint}
	$\displaystyle\int_1^8xf^{\prime}(x)\,dx=\frac{q}{p}$일 때, $p+q$의 값을 구하시오. (단, $p$와 $q$는 서로소인 자연수이다.) [4점]
\end{prob}

\begin{sol}
	$g(2)=2f(1)=2~\longrightarrow~f(2)=2$
	\[
		\longrightarrow\int_1^2xf^{\prime}(x)\,dx=\int_{f(1)}^{f(2)}g(x)\,dx=2f(2)-f(1)-\int_1^2f(x)\,dx=\frac{7}{4}
	\]
	$g(4)=2f(2)=4~\longrightarrow~f(4)=4$
	\[
		\longrightarrow\int_2^4xf^{\prime}(x)\,dx=\int_{f(2)}^{f(4)}g(x)\,dx=\int_2^4g(x)\,dx=2\int_1^2g(2x)\,dx=4\int_1^2f(x)\,dx=5
	\]
	$g(8)=2f(8)=8~\longrightarrow~f(8)=8$
	\[
		\begin{aligned}
			\longrightarrow~\int_4^8xf^{\prime}(x)\,dx&=\int_{f(4)}^{f(8)}g(x)\,dx=\int_4^8g(x)\,dx=2\int_2^4g(2x)\,d=4\int_2^4f(x)\,dx\\[3pt]
			&=4\int_{g(2)}^{g(4)} f(x)\,dx=4\biggl\{4g(4)-2g(2)-\int_2^4g(x)\,dx\biggr\}=28
		\end{aligned}
	\]
	\[
		\int_1^8xf^{\prime}(x)\,dx=\frac{7}{4}+5+28=\frac{139}{4}~\longrightarrow~p=4,~q=139~\longrightarrow~p+q=143\qed
	\]
\end{sol}

\newpage
\begin{prob}{2021학년도 수능특강}{50}
	정의역이 $\left\{x\,\middle\vert\,-\dfrac{\pi}{2}<x<\dfrac{\pi}{2}\right\}$인 함수 $f(x)=\displaystyle\int_0^x\tan\theta\,d\theta$가 있다. $0<t<\dfrac{\pi}{2}$인 실수 $t$에 대하여 $0\leq x\leq t$일 때 곡선 $y=f(x)$의 곡선의 길이를 $l(t)$라 하자. $\lim\limits_{t\,\to\,\frac{\pi}{2}-}\biggl\{l(t)+\ln\dfrac{\sin t}{f^{\prime}(t)}\biggr\}$의 값은?
\end{prob}

\begin{sol}
	$f^\prime(x)=\tan x$이므로
	\[
		\begin{aligned}
			l(t)&=\int_0^t\sqrt{1+\{f^{\prime}(u)\}^2}\,du=\int_0^t\sqrt{1+\tan^2u}\,du=\int_0^t\sec u\,du\\[3pt]
			&=\int_0^t\frac{1}{\cos u}\,du=\int_0^t\frac{\cos u}{\cos^2u}\,du=\int_0^t\frac{\cos u}{1-\sin^2u}\,du=\frac{1}{2}\int_0^t\cos u\biggl(\frac{1}{1+\sin u}+\frac{1}{1-\sin u}\biggr)du\\[3pt]
			&=\frac{1}{2}\ln(1+\sin u)-\frac{1}{2}(1-\sin u)=\frac{1}{2}\ln\frac{1+\sin u}{1-\sin u}.
		\end{aligned}
	\]
	따라서
	\[
		\begin{aligned}
			\lim\limits_{t\,\to\,\frac{\pi}{2}-}\biggl\{l(t)+\ln\frac{\sin t}{f^{\prime}(t)}\biggr\}&=\lim\limits_{t\,\to\,\frac{\pi}{2}-}\biggl\{\frac{1}{2}\ln\frac{1+\sin u}{1-\sin u}+\ln\frac{\sin t}{\tan t}\biggr\}\\[3pt]
			&=\lim\limits_{t\,\to\,\frac{\pi}{2}-}\biggl\{\ln\frac{1+\sin u}{\cos t}+\ln(\cos t)\biggr\}\\[3pt]
			&=\lim\limits_{t\,\to\,\frac{\pi}{2}-}\ln(1+\sin u)=\ln 2.\qed
		\end{aligned}
	\]
\end{sol}

\newpage
\begin{prob}{2025학년도 수능특강}{51}
	실수 전체의 집합에서 미분가능한 함수 $f(x)$가 다음 조건을 만족시킬 때, $\displaystyle\int_{-\tfrac{\pi}{2}}^{\tfrac{\pi}{2}}\frac{xf^{\prime}(x)}{1+\pi^{f^{\prime}(x)}}\,dx$의 값은?
	\begin{hint}
		\begin{itemize}
			\item[(가)] 모든 실수 $x$에 대하여 $f(-x)=f(x)$이다.
			\item[(나)] $f\biggl(\dfrac{\pi}{2}\biggr)=12$
			\item[(다)] $\displaystyle\int_0^{\tfrac{\pi}{2}}f(x)\,dx=12$
		\end{itemize}
	\end{hint}
\end{prob}

\begin{sol}
	(가)에서 $f(-x)=f(x)$이므로 $f^\prime(-x)=-f^\prime(x)$이다. $I=\displaystyle\int_{-\tfrac{\pi}{2}}^{\tfrac{\pi}{2}}\frac{xf^{\prime}(x)}{1+\pi^{f^{\prime}(x)}}\,dx$라 하면 정리 \ref{thm:9.1}에 의해
	\[
		I=\int_{-\tfrac{\pi}{2}}^{\tfrac{\pi}{2}}\frac{xf^{\prime}(x)}{1+\pi^{f^{\prime}(x)}}\,dx=\int_{-\tfrac{\pi}{2}}^{\tfrac{\pi}{2}}\frac{-xf^{\prime}(-x)}{1+\pi^{f^{\prime}(-x)}}\,dx=\int_{-\tfrac{\pi}{2}}^{\tfrac{\pi}{2}}\frac{xf^{\prime}(x)}{1+\pi^{-f^{\prime}(x)}}\,dx=\int_{-\tfrac{\pi}{2}}^{\tfrac{\pi}{2}}\frac{xf^{\prime}(x)\pi^{f^{\prime}(x)}}{1+\pi^{f^{\prime}(x)}}\,dx
	\]
	이므로
	\[
		\begin{aligned}
			2I&=\int_{-\tfrac{\pi}{2}}^{\tfrac{\pi}{2}}\frac{xf^{\prime}(x)}{1+\pi^{f^{\prime}(x)}}\,dx+\int_{-\tfrac{\pi}{2}}^{\tfrac{\pi}{2}}\frac{xf^{\prime}(x)\pi^{f^{\prime}(x)}}{1+\pi^{f^{\prime}(x)}}\,dx=\int_{-\tfrac{\pi}{2}}^{\tfrac{\pi}{2}}xf^{\prime}(x)\,dx\\[3pt]
			&=\frac{\pi}{2}f\biggl(\frac{\pi}{2}\biggr)-\biggl(-\frac{\pi}{2}\biggr)f\biggl(-\frac{\pi}{2}\biggr)-\int_{-\tfrac{\pi}{2}}^{\tfrac{\pi}{2}}f(x)\,dx\\[3pt]
			&=\pi f\biggl(\frac{\pi}{2}\biggr)-2\int_0^{\tfrac{\pi}{2}}f(x)\,dx=12\pi-24.
		\end{aligned}
	\]
	따라서 $\displaystyle\int_{-\tfrac{\pi}{2}}^{\tfrac{\pi}{2}}\frac{xf^{\prime}(x)}{1+\pi^{f^{\prime}(x)}}\,dx=I=6\pi-12$이다.\qed
\end{sol}

\newpage
\begin{prob}{2022학년도 수능특강}{52}
	정의역이 $\{x\,\vert\,x>0\}$인 미분가능한 함수 $f(x)$가 모든 양의 실수 $x$에 대하여
	\[
		f(x)>0,\quad\{f(x)\}^2-xf(x)f^{\prime}(x)=x^4e^{-x}
	\]
	을 만족시킨다.
	\[
		\int_1^2\frac{e^{2x}\{f(2x)\}^3}{x^3}\,dx-12\int_2^4f(x)\,dx=\frac{e^4}{m}\{f(4)\}^2-\frac{e^2}{2}\{f(2)\}^3
	\]
	일 때, 자연수 $m$의 값을 구하시오.
\end{prob}

\begin{sol}
	\[
		\begin{aligned}
			\{f(x)\}^2-xf(x)f^{\prime}(x)=x^4e^{-x}~&\longrightarrow~f(x)-xf^{\prime}(x)=\frac{x^4e^{-x}}{f(x)}\\[3pt]
			&\longrightarrow~\frac{xf^{\prime}(x)-f(x)}{x^2}=-\frac{x^2e^{-x}}{f(x)}\\[3pt]
			&\longrightarrow~\biggl(\frac{f(x)}{x}\biggr)^{\prime}=-\frac{x^2e^{-x}}{f(x)}
		\end{aligned}
	\]
	$2x=t$라 하면 $\dfrac{dt}{dx}=2$이므로
	\[
		\begin{tikzcd}
			D & I \\[-12pt]
			\biggl\{\dfrac{f(t)}{t}\biggr\}^2 \arrow[rd, "+"] & 4e^t \\
			3\biggl\{\dfrac{f(t)}{t}\biggr\}\biggl\{-\dfrac{t^2e^{-t}}{f(t)}\biggr\}=-3e^{-t}f(t) \arrow[r, "-"] & 4e^t
		\end{tikzcd}
	\]
	\[
		\begin{aligned}
			\int_1^2\frac{e^{2x}\{f(2x)\}^3}{x^3}\,dx&=\int_2^4\frac{4e^t\{f(t)\}^3}{t^3}\,dt\\[3pt]
			&=\biggl[4e^t\biggl\{\frac{f(t)}{t}\biggr\}^3\biggr]_2^4+12\int_2^4f(t)\,dt\\[3pt]
			&=\frac{e^4}{16}\{f(4)\}^3-\frac{e^2}{2}\{f(2)\}^3+12\int_2^4f(t)\,dt.
		\end{aligned}
	\]
	따라서 $m=16$이다.\qed
\end{sol}

\newpage
\begin{prob}{2020학년도 수능특강}{53}
	양의 실수 전체의 집합에서 미분가능하고, 역함수가 존재하는 함수 $f(x)$가 다음 조건을 만족시킨다.
	\begin{hint}
		\begin{itemize}
			\item[(가)] $1\leq x\leq 8$인 모든 실수 $x$에 대하여 $f^{\prime}(x)>0$, $f^{\prime}(x)f^{-1}(x)=\dfrac{\sqrt{(x-1)(x^2-1)}}{2}$이다.
			\item[(나)] $f(3)=2$, $f(8)=3$
		\end{itemize}
	\end{hint}
	$\displaystyle\int_1^3\frac{f(x)}{f^{\prime}(f^{-1}(x))}\,dx$의 값은?
\end{prob}

\begin{sol}
	(가)에서 $x=1$을 대입하면 $f^{\prime}(1)f^{-1}(1)=0$, $f^{-1}(1)=0$이다. (나)에서 $f^{-1}(3)=8$이다. 합성함수의 미분법에 의해 $(f^{-1})^{\prime}(x)=\dfrac{1}{f^{\prime}(f^{-1}(x))}$이므로
	\[
		\begin{aligned}
			\int_1^3\frac{f(x)}{f^{\prime}(f^{-1}(x))}\,dx&=\int_1^3f(x)(f^{-1})^{\prime}(x)\,dx=\biggl[f(x)f^{-1}(x)\biggr]_1^3-\int_1^3f^{\prime}(x)f^{-1}(x)\,dx\\[3pt]
			&=f(3)f^{-1}(3)-f(1)f^{-1}(1)-\int_1^3\frac{\sqrt{(x-1)(x^2-1)}}{2}\,dx\\[3pt]
			&=16-\int_1^3\frac{\sqrt{(x-1)(x^2-1)}}{2}\,dx,
		\end{aligned}
	\]
	\[
		\begin{aligned}
			\int_1^3\frac{\sqrt{(x-1)(x^2-1)}}{2}\,dx&=\frac{1}{2}\int_1^3(x-1)\sqrt{x+1}\,dx=\frac{1}{2}\int_2^4(x-2)\sqrt{x}\,dx\\[3pt]
			&=\biggl[\frac{1}{5}x^2\sqrt{x}-\frac{2}{3}x\sqrt{x}\biggr]_2^4=\frac{16+8\sqrt2}{15},
		\end{aligned}
	\]
	\[
		\int_1^3\frac{f(x)}{f^{\prime}(f^{-1}(x))}\,dx=16-\frac{16+8\sqrt2}{15}=\frac{8(28-\sqrt2)}{15}.\qed
	\]
\end{sol}

\newpage
\begin{prob}{2024학년도 수능완성}{54}
	실수 전체의 집합에서 연속인 이계도함수를 갖는 함수 $f(x)$가 모든 실수 $x$에 대하여 다음 조건을 만족시킨다.
	\begin{hint}
		\begin{itemize}
			\item[(가)] $f^{\prime}(x)<0$
			\item[(나)] $\{f^{\prime}(x)\}^2+3\displaystyle\int_0^xf(2t)\,dt=9$
		\end{itemize}
	\end{hint}
	$f^{\prime\prime}(0)=0$, $\{f(1)\}^2=\{f^{\prime}(1)\}^2-\{f^{\prime}(0)\}^2$, $\displaystyle\int_0^2f(x)\,dx=-\frac{3}{2}\biggl(e-\frac{1}{e}\biggr)^2$일 때, $\displaystyle\int_0^1\frac{f^{\prime\prime}(x)\times f(x)}{\{f^{\prime}(x)\}^2}\,dx=\frac{k}{e^2+1}$이다. 상수 $k$의 값을 구하시오. [4점]
\end{prob}

\begin{sol}
	\[
		\begin{tikzcd}
			D & I & \\[-12pt]
			f(x) \arrow[rd, "+"] & \dfrac{f^{\prime\prime}(x)}{\{f^\prime(x)\}^2} & \\
			f^\prime(x) & -\dfrac{1}{f^\prime(x)} \arrow[r] & -1 \\[-12pt] \hline \\[-12pt]
			-1 \arrow[rd, "-"] & 1 & \\
			0 & x &
		\end{tikzcd}\quad \int_0^1\frac{f^{\prime\prime}(x)\times f(x)}{\{f^{\prime}(x)\}^2}\,dx=\biggl[-\frac{f(x)}{f^{\prime}(x)}+x\biggr]_0^1=\frac{f(0)}{f^{\prime}(0)}-\frac{f(1)}{f^{\prime}(1)}+1
	\]
	(나)에 $x=0$을 대입하면 $\{f^{\prime}(0)\}^2=9$이다. (나)의 양변을 미분하면 $2f^{\prime}(x)f^{\prime\prime}(x)+3f(2x)=0$이므로 $x=0$을 대입하면 $f(0)=0$이다. $2t=u$라 하면 $\dfrac{du}{dt}=2$이므로
	\[
		\int_0^1f(2t)\,dt=\frac{1}{2}\int_0^2f(u)\,du=-\frac{3}{4}\biggl(e-\frac{1}{e}\biggr)^2.
	\]
	따라서
	\[
		\begin{aligned}
			\{f^{\prime}(1)\}^2&=9-3\int_0^1f(2t)\,dt=9+\frac{9}{4}\biggl(e-\frac{1}{e}\biggr)^2,\\[3pt]
			\{f(1)\}^2&=\{f^{\prime}(1)\}^2-\{f^{\prime}(0)\}^2=\frac{9}{4}\biggl(e-\frac{1}{e}\biggr)^2,\\[3pt]
			\frac{\{f(1)\}^2}{\{f^{\prime}(1)\}^2}&=\frac{\dfrac{9}{4}\biggl(e-\dfrac{1}{e}\biggr)^2}{9+\dfrac{9}{4}\biggl(e-\dfrac{1}{e}\biggr)^2}=\frac{(e^2-1)^2}{4e^2+(e^2-1)^2}=\frac{(e^2-1)^2}{(e^2+1)^2},\\[3pt]
		\end{aligned}
	\]
	\[
		\int_0^1\frac{f^{\prime\prime}(x)\times f(x)}{\{f^{\prime}(x)\}^2}\,dx=\frac{f(0)}{f^{\prime}(0)}-\frac{f(1)}{f^{\prime}(1)}+1=1-\frac{e^2-1}{e^2+1}=\frac{2}{e^2+1}
	\]
	이므로 $k=2$이다.\qed
\end{sol}

\newpage
\begin{prob}{2026학년도 9월 모의평가 미적분 30번}{55}
	실수 전체의 집합에서 미분가능한 함수 $f(x)$와 실수 전체의 집합에서 연속인 함수 $g(x)$는 모든 실수 $x$에 대하여
	\[
		f(x)=\ln\biggl(\frac{g(x)}{1+xf^\prime(x)}\biggr)
	\]
	를 만족시킨다. $f(1)=4\ln2$이고
	\[
		\int_1^2g(x)\,dx=34,\quad\int_1^2xg(x)\,dx=53
	\]
	일 때, $\displaystyle\int_1^2xe^{f(x)}\,dx$의 값을 구하시오. [4점]
\end{prob}

\begin{sol}
	주어진 식을 $g(x)$에 대하여 정리하면
	\[
		f(x)=\ln\biggl(\frac{g(x)}{1+xf^\prime(x)}\biggr)~\longrightarrow~g(x)=e^{f(x)}\{1+xf^\prime(x)\}=(x)^\prime e^{f(x)}+x(e^{f(x)})^\prime=(xe^{f(x)})^\prime.
	\]
	\[
		\int_1^2g(x)\,dx=\int_1^2(xe^{f(x)})^\prime\,dx=\biggl[xe^{f(x)}\biggr]_1^2=2e^{f(2)}-e^{f(1)}=2e^{f(2)}-16
	\]
	이므로 $e^{f(2)}=25$이다. 따라서
	\[
		\int_1^2xg(x)\,dx=\int_1^2x(xe^{f(x)})^\prime\,dx=\biggl[x^2e^{f(x)}\biggr]_1^2-\int_1^2xe^{f(x)}\,dx=84-\int_1^2xe^{f(x)}\,dx
	\]
	에서 $\displaystyle\int_1^2xe^{f(x)}\,dx=31$이다.\qed
\end{sol}
\end{document}
